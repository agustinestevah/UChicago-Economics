\documentclass[10pt, oneside]{article} 
\usepackage{amsmath, amsthm, amssymb, calrsfs, wasysym, verbatim, bbm, color, graphics, geometry, esint, float}
\usepackage{mdframed}



\geometry{tmargin=.75in, bmargin=.75in, lmargin=.75in, rmargin = .75in}  

\newcommand{\bbR}{\mathbb{R}}
\newcommand{\bbC}{\mathbb{C}}
\newcommand{\bbZ}{\mathbb{Z}}
\newcommand{\bbP}{\mathbb{P}}
\newcommand{\bbN}{\mathbb{N}}
\newcommand{\bbQ}{\mathbb{Q}}
\newcommand{\Cdot}{\boldsymbol{\cdot}}
\newcommand{\scA}{\mathscr{A}}
\newcommand{\curl}{\text{curl}}
\newcommand{\Ind}{\text{Ind}}
\newcommand{\Log}{\text{Log}}
\newcommand{\Vol}{\text{Vol}}
\newcommand{\Var}{\text{Var}}
\newcommand{\Cov}{\text{Cov}}
\newcommand{\Corr}{\text{Corr}}
\newcommand{\bbE}{\mathbb{E}}
\newcommand{\bbV}{\mathbb{V}}




\newcommand{\sm}{\setminus}

\theoremstyle{definition}
\newtheorem{exmp}{Example}[section]
\newtheorem{thm}{Theorem}
\newtheorem{defn}{Definition}
\newtheorem{prop}{Proposition}
\newtheorem{conv}{Convention}
\newtheorem{rem}{Remark}
\newtheorem{lem}{Lemma}
\newtheorem{cor}{Corollary}

% Copyright 2021 Paolo Adajar (padajar.com, paoloadajar@mit.edu)
% 
% Permission is hereby granted, free of charge, to any person obtaining a copy of this software and associated documentation files (the "Software"), to deal in the Software without restriction, including without limitation the rights to use, copy, modify, merge, publish, distribute, sublicense, and/or sell copies of the Software, and to permit persons to whom the Software is furnished to do so, subject to the following conditions:
%
% The above copyright notice and this permission notice shall be included in all copies or substantial portions of the Software.
% 
% THE SOFTWARE IS PROVIDED "AS IS", WITHOUT WARRANTY OF ANY KIND, EXPRESS OR IMPLIED, INCLUDING BUT NOT LIMITED TO THE WARRANTIES OF MERCHANTABILITY, FITNESS FOR A PARTICULAR PURPOSE AND NONINFRINGEMENT. IN NO EVENT SHALL THE AUTHORS OR COPYRIGHT HOLDERS BE LIABLE FOR ANY CLAIM, DAMAGES OR OTHER LIABILITY, WHETHER IN AN ACTION OF CONTRACT, TORT OR OTHERWISE, ARISING FROM, OUT OF OR IN CONNECTION WITH THE SOFTWARE OR THE USE OR OTHER DEALINGS IN THE SOFTWARE.

\usepackage{fullpage}
\usepackage{enumitem}
\usepackage{amsfonts, amssymb, amsmath,amsthm}
\usepackage{mathtools}
\usepackage[pdftex, pdfauthor={\name}, pdftitle={\classnum~\assignment}]{hyperref}
\usepackage[dvipsnames]{xcolor}
\usepackage{bbm}
\usepackage{graphicx}
\usepackage{mathrsfs}
\usepackage{pdfpages}
\usepackage{tabularx}
\usepackage{pdflscape}
\usepackage{makecell}
\usepackage{booktabs}
\usepackage{natbib}
\usepackage{caption}
\usepackage{subcaption}
\usepackage{physics}
\usepackage[many]{tcolorbox}
\usepackage{version}
\usepackage{ifthen}
\usepackage{cancel}
\usepackage{listings}
\usepackage{courier}

\usepackage{tikz}
\usepackage{istgame}

\hypersetup{
	colorlinks=true,
	linkcolor=blue,
	filecolor=magenta,
	urlcolor=blue,
}

\setlength{\parindent}{0mm}
\setlength{\parskip}{2mm}

\setlist[enumerate]{label=({\alph*})}
\setlist[enumerate, 2]{label=({\roman*})}

\allowdisplaybreaks[1]

\newcommand{\psetheader}{
	\ifthenelse{\isundefined{\collaborators}}{
		\begin{center}
			{\setlength{\parindent}{0cm} \setlength{\parskip}{0mm}
				
				{\textbf{\classnum~\semester:~\assignment} \hfill \name}
				
				\subject \hfill \href{mailto:\email}{\tt \email}
				
				Instructor(s):~\instructors \hfill Due Date:~\duedate	
				
				\hrulefill}
		\end{center}
	}{
		\begin{center}
			{\setlength{\parindent}{0cm} \setlength{\parskip}{0mm}
				
				{\textbf{\classnum~\semester:~\assignment} \hfill \name\footnote{Collaborator(s): \collaborators}}
				
				\subject \hfill \href{mailto:\email}{\tt \email}
				
				Instructor(s):~\instructors \hfill Due Date:~\duedate	
				
				\hrulefill}
		\end{center}
	}
}

\renewcommand{\thepage}{\classnum~\assignment \hfill \arabic{page}}
\newcommand{\uconv}{\overset{}{\rightrightarrows}}
\makeatletter
\def\points{\@ifnextchar[{\@with}{\@without}}
\def\@with[#1]#2{{\ifthenelse{\equal{#2}{1}}{{[1 point, #1]}}{{[#2 points, #1]}}}}
\def\@without#1{\ifthenelse{\equal{#1}{1}}{{[1 point]}}{{[#1 points]}}}
\makeatother

\newtheoremstyle{theorem-custom}%
{}{}%
{}{}%
{\itshape}{.}%
{ }%
{\thmname{#1}\thmnumber{ #2}\thmnote{ (#3)}}

\theoremstyle{theorem-custom}

\newtheorem{theorem}{Theorem}
\newtheorem{lemma}[theorem]{Lemma}
\newtheorem{example}[theorem]{Example}

\newenvironment{problem}[1]{\color{black} #1}{}

\newenvironment{solution}{%
	\leavevmode\begin{tcolorbox}[breakable, colback=green!5!white,colframe=green!75!black, enhanced jigsaw] \proof[\scshape Solution:] \setlength{\parskip}{2mm}%
	}{\renewcommand{\qedsymbol}{$\heartsuit$} \endproof \end{tcolorbox}}

\newenvironment{reflection}{\begin{tcolorbox}[breakable, colback=black!8!white,colframe=black!60!white, enhanced jigsaw, parbox = false]\textsc{Reflections:}}{\end{tcolorbox}}

\newcommand{\qedh}{\renewcommand{\qedsymbol}{$\blacksquare$}\qedhere}

\definecolor{mygreen}{rgb}{0,0.6,0}
\definecolor{mygray}{rgb}{0.5,0.5,0.5}
\definecolor{mymauve}{rgb}{0.58,0,0.82}

% from https://github.com/satejsoman/stata-lstlisting
% language definition
\lstdefinelanguage{Stata}{
	% System commands
	morekeywords=[1]{regress, reg, summarize, sum, display, di, generate, gen, bysort, use, import, delimited, predict, quietly, probit, margins, test},
	% Reserved words
	morekeywords=[2]{aggregate, array, boolean, break, byte, case, catch, class, colvector, complex, const, continue, default, delegate, delete, do, double, else, eltypedef, end, enum, explicit, export, external, float, for, friend, function, global, goto, if, inline, int, local, long, mata, matrix, namespace, new, numeric, NULL, operator, orgtypedef, pointer, polymorphic, pragma, private, protected, public, quad, real, return, rowvector, scalar, short, signed, static, strL, string, struct, super, switch, template, this, throw, transmorphic, try, typedef, typename, union, unsigned, using, vector, version, virtual, void, volatile, while,},
	% Keywords
	morekeywords=[3]{forvalues, foreach, set},
	% Date and time functions
	morekeywords=[4]{bofd, Cdhms, Chms, Clock, clock, Cmdyhms, Cofc, cofC, Cofd, cofd, daily, date, day, dhms, dofb, dofC, dofc, dofh, dofm, dofq, dofw, dofy, dow, doy, halfyear, halfyearly, hh, hhC, hms, hofd, hours, mdy, mdyhms, minutes, mm, mmC, mofd, month, monthly, msofhours, msofminutes, msofseconds, qofd, quarter, quarterly, seconds, ss, ssC, tC, tc, td, th, tm, tq, tw, week, weekly, wofd, year, yearly, yh, ym, yofd, yq, yw,},
	% Mathematical functions
	morekeywords=[5]{abs, ceil, cloglog, comb, digamma, exp, expm1, floor, int, invcloglog, invlogit, ln, ln1m, ln, ln1p, ln, lnfactorial, lngamma, log, log10, log1m, log1p, logit, max, min, mod, reldif, round, sign, sqrt, sum, trigamma, trunc,},
	% Matrix functions
	morekeywords=[6]{cholesky, coleqnumb, colnfreeparms, colnumb, colsof, corr, det, diag, diag0cnt, el, get, hadamard, I, inv, invsym, issymmetric, J, matmissing, matuniform, mreldif, nullmat, roweqnumb, rownfreeparms, rownumb, rowsof, sweep, trace, vec, vecdiag, },
	% Programming functions
	morekeywords=[7]{autocode, byteorder, c, _caller, chop, abs, clip, cond, e, fileexists, fileread, filereaderror, filewrite, float, fmtwidth, has_eprop, inlist, inrange, irecode, matrix, maxbyte, maxdouble, maxfloat, maxint, maxlong, mi, minbyte, mindouble, minfloat, minint, minlong, missing, r, recode, replay, return, s, scalar, smallestdouble,},
	% Random-number functions
	morekeywords=[8]{rbeta, rbinomial, rcauchy, rchi2, rexponential, rgamma, rhypergeometric, rigaussian, rlaplace, rlogistic, rnbinomial, rnormal, rpoisson, rt, runiform, runiformint, rweibull, rweibullph,},
	% Selecting time-span functions
	morekeywords=[9]{tin, twithin,},
	% Statistical functions
	morekeywords=[10]{betaden, binomial, binomialp, binomialtail, binormal, cauchy, cauchyden, cauchytail, chi2, chi2den, chi2tail, dgammapda, dgammapdada, dgammapdadx, dgammapdx, dgammapdxdx, dunnettprob, exponential, exponentialden, exponentialtail, F, Fden, Ftail, gammaden, gammap, gammaptail, hypergeometric, hypergeometricp, ibeta, ibetatail, igaussian, igaussianden, igaussiantail, invbinomial, invbinomialtail, invcauchy, invcauchytail, invchi2, invchi2tail, invdunnettprob, invexponential, invexponentialtail, invF, invFtail, invgammap, invgammaptail, invibeta, invibetatail, invigaussian, invigaussiantail, invlaplace, invlaplacetail, invlogistic, invlogistictail, invnbinomial, invnbinomialtail, invnchi2, invnF, invnFtail, invnibeta, invnormal, invnt, invnttail, invpoisson, invpoissontail, invt, invttail, invtukeyprob, invweibull, invweibullph, invweibullphtail, invweibulltail, laplace, laplaceden, laplacetail, lncauchyden, lnigammaden, lnigaussianden, lniwishartden, lnlaplaceden, lnmvnormalden, lnnormal, lnnormalden, lnwishartden, logistic, logisticden, logistictail, nbetaden, nbinomial, nbinomialp, nbinomialtail, nchi2, nchi2den, nchi2tail, nF, nFden, nFtail, nibeta, normal, normalden, npnchi2, npnF, npnt, nt, ntden, nttail, poisson, poissonp, poissontail, t, tden, ttail, tukeyprob, weibull, weibullden, weibullph, weibullphden, weibullphtail, weibulltail,},
	% String functions 
	morekeywords=[11]{abbrev, char, collatorlocale, collatorversion, indexnot, plural, plural, real, regexm, regexr, regexs, soundex, soundex_nara, strcat, strdup, string, strofreal, string, strofreal, stritrim, strlen, strlower, strltrim, strmatch, strofreal, strofreal, strpos, strproper, strreverse, strrpos, strrtrim, strtoname, strtrim, strupper, subinstr, subinword, substr, tobytes, uchar, udstrlen, udsubstr, uisdigit, uisletter, ustrcompare, ustrcompareex, ustrfix, ustrfrom, ustrinvalidcnt, ustrleft, ustrlen, ustrlower, ustrltrim, ustrnormalize, ustrpos, ustrregexm, ustrregexra, ustrregexrf, ustrregexs, ustrreverse, ustrright, ustrrpos, ustrrtrim, ustrsortkey, ustrsortkeyex, ustrtitle, ustrto, ustrtohex, ustrtoname, ustrtrim, ustrunescape, ustrupper, ustrword, ustrwordcount, usubinstr, usubstr, word, wordbreaklocale, worcount,},
	% Trig functions
	morekeywords=[12]{acos, acosh, asin, asinh, atan, atanh, cos, cosh, sin, sinh, tan, tanh,},
	morecomment=[l]{//},
	% morecomment=[l]{*},  // `*` maybe used as multiply operator. So use `//` as line comment.
	morecomment=[s]{/*}{*/},
	% The following is used by macros, like `lags'.
	morestring=[b]{`}{'},
	% morestring=[d]{'},
	morestring=[b]",
	morestring=[d]",
	% morestring=[d]{\\`},
	% morestring=[b]{'},
	sensitive=true,
}

\lstset{ 
	backgroundcolor=\color{white},   % choose the background color; you must add \usepackage{color} or \usepackage{xcolor}; should come as last argument
	basicstyle=\footnotesize\ttfamily,        % the size of the fonts that are used for the code
	breakatwhitespace=false,         % sets if automatic breaks should only happen at whitespace
	breaklines=true,                 % sets automatic line breaking
	captionpos=b,                    % sets the caption-position to bottom
	commentstyle=\color{mygreen},    % comment style
	deletekeywords={...},            % if you want to delete keywords from the given language
	escapeinside={\%*}{*)},          % if you want to add LaTeX within your code
	extendedchars=true,              % lets you use non-ASCII characters; for 8-bits encodings only, does not work with UTF-8
	firstnumber=0,                % start line enumeration with line 1000
	frame=single,	                   % adds a frame around the code
	keepspaces=true,                 % keeps spaces in text, useful for keeping indentation of code (possibly needs columns=flexible)
	keywordstyle=\color{blue},       % keyword style
	language=Octave,                 % the language of the code
	morekeywords={*,...},            % if you want to add more keywords to the set
	numbers=left,                    % where to put the line-numbers; possible values are (none, left, right)
	numbersep=5pt,                   % how far the line-numbers are from the code
	numberstyle=\tiny\color{mygray}, % the style that is used for the line-numbers
	rulecolor=\color{black},         % if not set, the frame-color may be changed on line-breaks within not-black text (e.g. comments (green here))
	showspaces=false,                % show spaces everywhere adding particular underscores; it overrides 'showstringspaces'
	showstringspaces=false,          % underline spaces within strings only
	showtabs=false,                  % show tabs within strings adding particular underscores
	stepnumber=2,                    % the step between two line-numbers. If it's 1, each line will be numbered
	stringstyle=\color{mymauve},     % string literal style
	tabsize=2,	                   % sets default tabsize to 2 spaces
%	title=\lstname,                   % show the filename of files included with \lstinputlisting; also try caption instead of title
	xleftmargin=0.25cm
}



\title{UChicago Econometrics Notes: 20510}
\author{Notes by Agustín Esteva, Lectures by Murilo Ramos}
\date{Academic Year 2024-2025}

\begin{document}

\maketitle
\tableofcontents

\vspace{.25in}


\newpage
\section{Lectures}

\subsection*{Lecture 1: The Envelope Theorem}
\begin{thm} (Inverse Function Theorem)
    Suppose $f: \bbR^n \to \bbR$ and $g_j: \bbR^n \to \bbR$ is $C^1$ for $j \in [m]$ (i.e, $g: \bbR^n \to \bbR^m$). Suppose $f$ has a local extremum at $x_0$ with the constraint $g(x) =0$ and $\text{rank}(Dg_{x_0}) = m.$ Then there are real numbers, called \textbf{Lagrange Multipliers} $\lambda_j \in \bbR$ such that 
    \[\nabla f(x_0) = \sum_{j=1}^m \lambda_j \nabla g_j(x_0)\]
\end{thm}

We will prove this for the $m = 1,$ $n=2$ case, which states that if $f: \bbR^2 \to \bbR$ and $g: \bbR^2 \to \bbR$ is $C^1,$ then if $f$ has a local extremum at $x_0$ with the constraint of $g(x) = 0,$ and $\nabla g(x_0) \neq 0,$ then there exists some $\lambda \in \bbR$ such that 
\[\nabla f(x_0) = \lambda \nabla g(x_0).\]
\begin{proof}
    By the implicit function theorem, we have that $g(x_1, x_2) = 0$ can locally be written as either 
    \[g(x_1, \gamma(x_2)) =0, \qquad \text{or}\qquad g(\beta(x_2), x_2) = 0.\] Let's concentrate on the first. Then for all $x_1 \in U,$ we write $g(x_1, \gamma(x_2)) = 0.$ By assumption, we have that that by the chain rule
    \begin{align}
        0 = \frac{d}{dx_1}g(x_1, \gamma(x_1)) = \langle 
    \nabla g(x_1, \gamma(x_1)), \begin{pmatrix}
        1 \\ \gamma'(x)
    \end{pmatrix}
    \rangle.
    \end{align} We have that the function $\hat{f}: U \to \bbR$ with $\hat{f}(x_1) = f(x_1, \gamma(x_1))$ has a local extremum at $x_{0_1}.$ Thus, we have that 
    \[0 = \hat{f}'(x_{0_{1}}) = \langle \nabla f(x_{0_1}, \gamma(x_{0_1})), \begin{pmatrix}
        1 \\ \gamma'(x_{0_1})
    \end{pmatrix}\rangle\] Because (1) holds for any $x\in U,$ then we have the $\nabla f$ and $\nabla g$ are orthagonal to each other, and so there must exist some real $\lambda$ such that 
    \[\nabla f = \lambda \nabla g.\]
\end{proof}
\begin{thm} (Envelope Theorem)
    Suppose $f(x, \alpha), g_j(x,\alpha): \bbR^{n + \ell} \to \bbR,$ for $j = 1, 2,\dots, \ell$ and $g_j(x,\alpha) \geq 0$ for all $j$. If we call 
    \[x^*(\alpha) = \max_xf(x, \alpha),\] and assume $\nabla g(x^*(\alpha), \alpha) \neq 0$ and define
    \[V(\alpha)= f(x^*(\alpha), \alpha),\] then for each $j \in [k],$ we have 
    \[\frac{\partial V(\alpha)}{\partial \alpha_j} = \frac{\partial \mathcal{L}(x^*(\alpha), \lambda^*, \alpha)}{\partial \alpha_j},\] where 
    \[\mathcal{L}(x, \lambda, \alpha) = f(x, \alpha) - \lambda g(x, \alpha)\] and $\lambda^*$ are the Lagrange multipliers. 
\end{thm}
\begin{proof}
    We use the chain rule to see that 
    \begin{align*}
      \frac{\partial V(\alpha)}{\partial \alpha_k} &= \frac{\partial f(x^*(\alpha), \alpha)}{\partial \alpha_k}\\
      &= \nabla_x f(x^*(\alpha), \alpha) \cdot \frac{\partial x^*(\alpha)}{\partial \alpha_k}+\frac{\partial f(x^*(\alpha), \alpha)}{\partial \alpha_k}\\ &= -\sum \lambda_j^* \nabla_x g(x^*(\alpha), \alpha) \cdot \frac{\partial x^*(\alpha)}{\partial \alpha_k}+\frac{\partial f(x^*(\alpha), \alpha)}{\partial \alpha_j}\\
      &= -\sum \lambda_j^*\frac{\partial g_j(x^*(\alpha), \alpha)}{\partial \alpha_k} + \frac{\partial f(x^*(\alpha), \alpha)}{\partial \alpha_k}\\
      &= \frac{\partial}{ \partial \alpha_j} (f(x^*(\alpha), \alpha) - \sum_{j=1}^\ell\lambda_j^* g_j(x^*(\alpha), \alpha))\\
      &= \frac{\partial \mathcal{L}(x^*(\alpha), \lambda^*(\alpha), \alpha}{\partial \alpha_j}
    \end{align*}
\end{proof}

\begin{exmp}
    Let $p_x$ and $p_y$ be the prices of good $x$ and good $y.$ Suppose a consumer wants to maximize 
    \[\max_{x,y}\,U(x,y)\quad \text{s.t.}\quad p_xx  + p_yy \leq m.\] That is, she wants to maximize the utility from buying $x$ and $y$ with an income of $m.$ Then if $V(m) = U(x^*,y^*)$ subject to $g(x,y,m) =: p_xx + p_yy - m = 0.$ Then the Lagrangian is given by 
    \[\mathcal{L}((x, y),\lambda, m)= U(x,y) - \lambda g(x,y,m)\] and thus by the Envelope Theorem, 
    \begin{align}
    \frac{\partial V}{\partial m}(x^*, y^*, m) &= \frac{\partial \cal L}{\partial m}(x^*, y^*, m) = -\lambda^*(-1) = \lambda^*
    \end{align}
     
\end{exmp}

\newpage
\subsection*{Lecture 2: Scarcity: The Budget Constraint}
\begin{defn}
    Let $x_1, \dots, x_n$ be goods with non-zero prices $p_{x_1}, \dots, p_{x_n}$ respectively. We say that the \textbf{budget set} is the set of all bundles of goods a consumer can buy given a budget/income $m$. That is, 
    \[B(p_{x_1}, \dots, p_{x_n}, m) = \{(x_1, \dots, x_n)\in \bbR^n \mid \sum_{j=1}^n  p_{x_j}x_j \leq m\}\]
\end{defn}
\begin{rem}
    Notice that the optimal bundles are given by the boundary line 
    \[\sum_{j=1}^n  p_{x_j}x_j = m\] Implying that
    \[x_k = \frac{1}{p_{x_k}} \left(-\sum_{j=1, j \neq k}^n[p_{x_j}x_j] + m\right)  \] and thus 
    \[\frac{\partial x_k}{\partial x_i}=  -\frac{p_{x_i}}{p_{x_k}}\] is the rate at which the market allows the consumer to trade good $x_i$  to $x_k.$ 
\end{rem}
\begin{defn}
    \begin{itemize}
    Let $\succeq$ be a relation.
        \item We say that a consumer \textbf{prefers} $x_1$ to $x_2$ if $x_2 \succeq x_1$ and \textbf{strictly prefers} $x_1$ to $x_2$ if $x_1 \succ x_2.$  
        \item  We say that a consumer's preference is \textbf{complete} if for any $x_1, x_2$ consumption bundles, either $x_1 \succeq x_2$ or $x_2 \succeq x_1.$
        \item We say that a consumer's preference is \textbf{transitive} if  when $x_1 \succeq x_2$ and $x_2 \succeq x_3,$ then $x_1 \succeq x_3.$
        \item We say that a consumer's preference is \textbf{rational} if it is both complete, symmetric, and rational.
        \item We say that a consumer's preference is \textbf{continuous} if when $x_1 \succ x_2,$ there exist neighborhoods $N_1, N_2$ such that for all $x \in N_1$ we have $x \succ x_2$ and symmetrically for $N_2.$
    \end{itemize}
\end{defn}
\begin{thm}
    (Utility Representation Theorem) If a preference ordering is rational, then it admits a utility function representation that is unique up to a monotonically increasing transformation.
\end{thm}

\newpage
\subsection*{Lecture 3: Utility Maximization}
\begin{defn}
    The \textbf{marginal rate of substitution} (MRS) is defined to be 
    \[\frac{dy}{dx} = -\frac{U_x}{U_y},\] and it measures an individual's willingness to pay for $x$ in terms of $y.$ 
\end{defn}

\begin{rem}
In the setting of Example 1.1, we wish to
maximize \[dU = U_x dx + U_ydy\] such that $p_x dx + p_ydy= 0$ (equivalently, $dy = -\frac{p_x}{p_y}dx$). Plugging in gives 
\begin{align}
    dU = U_xdx + U_y(-\frac{p_x}{p_y}dx) = \left[U_x - U_y\frac{p_x}{p_y}\right]dx
\end{align}
\begin{enumerate}
    \item If $U_x/U_y > p_x/p_y,$ set $dx>0$ so that 
    \[\left[\frac{U_x}{U_y} - \frac{p_x}{p_y}\right]U_y dx\] is \underline{taking advantage of all trading opportunities}
    \item If $U_x/p_x > U_y/p_y,$ set $dx>0$ so that 
    \[\left[\frac{U_x}{p_x} - \frac{U_y}{p_y}\right]p_x dx\] is the \underline{bang for your buck}.
    \item If $U_x > U_y \cdot p_x/p_y,$ then setting $dx>0,$ 
    \[\left[U_x - U_y\frac{p_x}{p_y}\right]dx\] is the \underline{trade until the marginal cost equals the marginal benefit}
\end{enumerate}
Recall from (2) that \[\frac{\partial V}{\partial m}(x^*, y^*, m) = \lambda^*.\] Using the Envelope Theorem, we see that 
\begin{align}
    \frac{\partial V^*}{\partial p_x} = \frac{\partial \cal L^*}{\partial p_x} = -\lambda^* x^*
\end{align}

\end{rem}

\subsection*{Lecture 4: Expenditure Minimization} 
We seek to find the bundle that minimizes the expenditure that achieves a certain utility $\underline{U}.$ That is 
\[e(x,y,\bar U) = \min_{x,y} p_x x + p_yy \qquad U(x,y) = \bar U.\] Solving gives the Lagrangian
\[\mathcal{L}(x,y,\bar U) = e(x,y,\bar U) - \lambda (U(x,y) - \bar U) = p_x x + p_yy- \lambda (U(x,y) - \bar U)\]
with first order conditions 
\[0=\frac{\partial \cal L}{\partial x}(x^*, y^*, \bar U) = p_x - \lambda^* U_x(x,y) \implies p_x = \lambda^* U_x(x,y)\]
\[0 = \frac{\partial \cal L}{\partial x}(x^*, y^*, \bar U) = p_y - \lambda^* U_y(x,y) \implies p_y = \lambda^* U_y(x,y)\]
\[U(x^*, y^*)= \bar U\]



\end{document}