\documentclass[11pt]{article}

% NOTE: Add in the relevant information to the commands below; or, if you'll be using the same information frequently, add these commands at the top of paolo-pset.tex file. 
\newcommand{\name}{Agustín Esteva}
\newcommand{\email}{aesteva@uchicago.edu}
\newcommand{\classnum}{20210}
\newcommand{\subject}{Econometric}
\newcommand{\instructors}{Murilo Ramos}
\newcommand{\assignment}{Problem Set 1}
\newcommand{\semester}{Summer 2025}
\newcommand{\duedate}{\today}
\newcommand{\bA}{\mathbf{A}}
\newcommand{\bB}{\mathbf{B}}
\newcommand{\bC}{\mathbf{C}}
\newcommand{\bD}{\mathbf{D}}
\newcommand{\bE}{\mathbf{E}}
\newcommand{\bF}{\mathbf{F}}
\newcommand{\bG}{\mathbf{G}}
\newcommand{\bH}{\mathbf{H}}
\newcommand{\bI}{\mathbf{I}}
\newcommand{\bJ}{\mathbf{J}}
\newcommand{\bK}{\mathbf{K}}
\newcommand{\bL}{\mathbf{L}}
\newcommand{\bM}{\mathbf{M}}
\newcommand{\bN}{\mathbf{N}}
\newcommand{\bO}{\mathbf{O}}
\newcommand{\bP}{\mathbf{P}}
\newcommand{\bQ}{\mathbf{Q}}
\newcommand{\bR}{\mathbf{R}}
\newcommand{\bS}{\mathbf{S}}
\newcommand{\bT}{\mathbf{T}}
\newcommand{\bU}{\mathbf{U}}
\newcommand{\bV}{\mathbf{V}}
\newcommand{\bW}{\mathbf{W}}
\newcommand{\bX}{\mathbf{X}}
\newcommand{\bY}{\mathbf{Y}}
\newcommand{\bZ}{\mathbf{Z}}
\newcommand{\Vol}{\text{Vol}}
\newcommand{\MSE}{\text{MSE}}
\newcommand{\Bias}{\text{Bias}}
\newcommand{\Var}{\text{Var}}
\newcommand{\Cov}{\text{Cov}}
\newcommand{\Corr}{\text{Corr}}



%% blackboard bold math capitals
\newcommand{\bbA}{\mathbb{A}}
\newcommand{\bbB}{\mathbb{B}}
\newcommand{\bbC}{\mathbb{C}}
\newcommand{\bbD}{\mathbb{D}}
\newcommand{\bbE}{\mathbb{E}}
\newcommand{\bbF}{\mathbb{F}}
\newcommand{\bbG}{\mathbb{G}}
\newcommand{\bbH}{\mathbb{H}}
\newcommand{\bbI}{\mathbb{I}}
\newcommand{\bbJ}{\mathbb{J}}
\newcommand{\bbK}{\mathbb{K}}
\newcommand{\bbL}{\mathbb{L}}
\newcommand{\bbM}{\mathbb{M}}
\newcommand{\bbN}{\mathbb{N}}
\newcommand{\bbO}{\mathbb{O}}
\newcommand{\bbP}{\mathbb{P}}
\newcommand{\bbQ}{\mathbb{Q}}
\newcommand{\bbR}{\mathbb{R}}
\newcommand{\bbS}{\mathbb{S}}
\newcommand{\bbT}{\mathbb{T}}
\newcommand{\bbU}{\mathbb{U}}
\newcommand{\bbV}{\mathbb{V}}
\newcommand{\bbW}{\mathbb{W}}
\newcommand{\bbX}{\mathbb{X}}
\newcommand{\bbY}{\mathbb{Y}}
\newcommand{\bbZ}{\mathbb{Z}}

%% script math capitals
\newcommand{\sA}{\mathscr{A}}
\newcommand{\sB}{\mathscr{B}}
\newcommand{\sC}{\mathscr{C}}
\newcommand{\sD}{\mathscr{D}}
\newcommand{\sE}{\mathscr{E}}
\newcommand{\sF}{\mathscr{F}}
\newcommand{\sG}{\mathscr{G}}
\newcommand{\sH}{\mathscr{H}}
\newcommand{\sI}{\mathscr{I}}
\newcommand{\sJ}{\mathscr{J}}
\newcommand{\sK}{\mathscr{K}}
\newcommand{\sL}{\mathscr{L}}
\newcommand{\sM}{\mathscr{M}}
\newcommand{\sN}{\mathscr{N}}
\newcommand{\sO}{\mathscr{O}}
\newcommand{\sP}{\mathscr{P}}
\newcommand{\sQ}{\mathscr{Q}}
\newcommand{\sR}{\mathscr{R}}
\newcommand{\sS}{\mathscr{S}}
\newcommand{\sT}{\mathscr{T}}
\newcommand{\sU}{\mathscr{U}}
\newcommand{\sV}{\mathscr{V}}
\newcommand{\sW}{\mathscr{W}}
\newcommand{\sX}{\mathscr{X}}
\newcommand{\sY}{\mathscr{Y}}
\newcommand{\sZ}{\mathscr{Z}}


\renewcommand{\emptyset}{\O}

\newcommand{\abs}[1]{\lvert #1 \rvert}
\newcommand{\norm}[1]{\lVert #1 \rVert}
\newcommand{\sm}{\setminus}


\newcommand{\sarr}{\rightarrow}
\newcommand{\arr}{\longrightarrow}

% NOTE: Defining collaborators is optional; to not list collaborators, comment out the line below.
%\newcommand{\collaborators}{Alyssa P. Hacker (\texttt{aphacker}), Ben Bitdiddle (\texttt{bitdiddle})}

% Copyright 2021 Paolo Adajar (padajar.com, paoloadajar@mit.edu)
% 
% Permission is hereby granted, free of charge, to any person obtaining a copy of this software and associated documentation files (the "Software"), to deal in the Software without restriction, including without limitation the rights to use, copy, modify, merge, publish, distribute, sublicense, and/or sell copies of the Software, and to permit persons to whom the Software is furnished to do so, subject to the following conditions:
%
% The above copyright notice and this permission notice shall be included in all copies or substantial portions of the Software.
% 
% THE SOFTWARE IS PROVIDED "AS IS", WITHOUT WARRANTY OF ANY KIND, EXPRESS OR IMPLIED, INCLUDING BUT NOT LIMITED TO THE WARRANTIES OF MERCHANTABILITY, FITNESS FOR A PARTICULAR PURPOSE AND NONINFRINGEMENT. IN NO EVENT SHALL THE AUTHORS OR COPYRIGHT HOLDERS BE LIABLE FOR ANY CLAIM, DAMAGES OR OTHER LIABILITY, WHETHER IN AN ACTION OF CONTRACT, TORT OR OTHERWISE, ARISING FROM, OUT OF OR IN CONNECTION WITH THE SOFTWARE OR THE USE OR OTHER DEALINGS IN THE SOFTWARE.

\usepackage{fullpage}
\usepackage{enumitem}
\usepackage{amsfonts, amssymb, amsmath,amsthm}
\usepackage{mathtools}
\usepackage[pdftex, pdfauthor={\name}, pdftitle={\classnum~\assignment}]{hyperref}
\usepackage[dvipsnames]{xcolor}
\usepackage{bbm}
\usepackage{graphicx}
\usepackage{mathrsfs}
\usepackage{pdfpages}
\usepackage{tabularx}
\usepackage{pdflscape}
\usepackage{makecell}
\usepackage{booktabs}
\usepackage{natbib}
\usepackage{caption}
\usepackage{subcaption}
\usepackage{physics}
\usepackage[many]{tcolorbox}
\usepackage{version}
\usepackage{ifthen}
\usepackage{cancel}
\usepackage{listings}
\usepackage{courier}

\usepackage{tikz}
\usepackage{istgame}

\hypersetup{
	colorlinks=true,
	linkcolor=blue,
	filecolor=magenta,
	urlcolor=blue,
}

\setlength{\parindent}{0mm}
\setlength{\parskip}{2mm}

\setlist[enumerate]{label=({\alph*})}
\setlist[enumerate, 2]{label=({\roman*})}

\allowdisplaybreaks[1]

\newcommand{\psetheader}{
	\ifthenelse{\isundefined{\collaborators}}{
		\begin{center}
			{\setlength{\parindent}{0cm} \setlength{\parskip}{0mm}
				
				{\textbf{\classnum~\semester:~\assignment} \hfill \name}
				
				\subject \hfill \href{mailto:\email}{\tt \email}
				
				Instructor(s):~\instructors \hfill Due Date:~\duedate	
				
				\hrulefill}
		\end{center}
	}{
		\begin{center}
			{\setlength{\parindent}{0cm} \setlength{\parskip}{0mm}
				
				{\textbf{\classnum~\semester:~\assignment} \hfill \name\footnote{Collaborator(s): \collaborators}}
				
				\subject \hfill \href{mailto:\email}{\tt \email}
				
				Instructor(s):~\instructors \hfill Due Date:~\duedate	
				
				\hrulefill}
		\end{center}
	}
}

\renewcommand{\thepage}{\classnum~\assignment \hfill \arabic{page}}
\newcommand{\uconv}{\overset{}{\rightrightarrows}}
\makeatletter
\def\points{\@ifnextchar[{\@with}{\@without}}
\def\@with[#1]#2{{\ifthenelse{\equal{#2}{1}}{{[1 point, #1]}}{{[#2 points, #1]}}}}
\def\@without#1{\ifthenelse{\equal{#1}{1}}{{[1 point]}}{{[#1 points]}}}
\makeatother

\newtheoremstyle{theorem-custom}%
{}{}%
{}{}%
{\itshape}{.}%
{ }%
{\thmname{#1}\thmnumber{ #2}\thmnote{ (#3)}}

\theoremstyle{theorem-custom}

\newtheorem{theorem}{Theorem}
\newtheorem{lemma}[theorem]{Lemma}
\newtheorem{example}[theorem]{Example}

\newenvironment{problem}[1]{\color{black} #1}{}

\newenvironment{solution}{%
	\leavevmode\begin{tcolorbox}[breakable, colback=green!5!white,colframe=green!75!black, enhanced jigsaw] \proof[\scshape Solution:] \setlength{\parskip}{2mm}%
	}{\renewcommand{\qedsymbol}{$\heartsuit$} \endproof \end{tcolorbox}}

\newenvironment{reflection}{\begin{tcolorbox}[breakable, colback=black!8!white,colframe=black!60!white, enhanced jigsaw, parbox = false]\textsc{Reflections:}}{\end{tcolorbox}}

\newcommand{\qedh}{\renewcommand{\qedsymbol}{$\blacksquare$}\qedhere}

\definecolor{mygreen}{rgb}{0,0.6,0}
\definecolor{mygray}{rgb}{0.5,0.5,0.5}
\definecolor{mymauve}{rgb}{0.58,0,0.82}

% from https://github.com/satejsoman/stata-lstlisting
% language definition
\lstdefinelanguage{Stata}{
	% System commands
	morekeywords=[1]{regress, reg, summarize, sum, display, di, generate, gen, bysort, use, import, delimited, predict, quietly, probit, margins, test},
	% Reserved words
	morekeywords=[2]{aggregate, array, boolean, break, byte, case, catch, class, colvector, complex, const, continue, default, delegate, delete, do, double, else, eltypedef, end, enum, explicit, export, external, float, for, friend, function, global, goto, if, inline, int, local, long, mata, matrix, namespace, new, numeric, NULL, operator, orgtypedef, pointer, polymorphic, pragma, private, protected, public, quad, real, return, rowvector, scalar, short, signed, static, strL, string, struct, super, switch, template, this, throw, transmorphic, try, typedef, typename, union, unsigned, using, vector, version, virtual, void, volatile, while,},
	% Keywords
	morekeywords=[3]{forvalues, foreach, set},
	% Date and time functions
	morekeywords=[4]{bofd, Cdhms, Chms, Clock, clock, Cmdyhms, Cofc, cofC, Cofd, cofd, daily, date, day, dhms, dofb, dofC, dofc, dofh, dofm, dofq, dofw, dofy, dow, doy, halfyear, halfyearly, hh, hhC, hms, hofd, hours, mdy, mdyhms, minutes, mm, mmC, mofd, month, monthly, msofhours, msofminutes, msofseconds, qofd, quarter, quarterly, seconds, ss, ssC, tC, tc, td, th, tm, tq, tw, week, weekly, wofd, year, yearly, yh, ym, yofd, yq, yw,},
	% Mathematical functions
	morekeywords=[5]{abs, ceil, cloglog, comb, digamma, exp, expm1, floor, int, invcloglog, invlogit, ln, ln1m, ln, ln1p, ln, lnfactorial, lngamma, log, log10, log1m, log1p, logit, max, min, mod, reldif, round, sign, sqrt, sum, trigamma, trunc,},
	% Matrix functions
	morekeywords=[6]{cholesky, coleqnumb, colnfreeparms, colnumb, colsof, corr, det, diag, diag0cnt, el, get, hadamard, I, inv, invsym, issymmetric, J, matmissing, matuniform, mreldif, nullmat, roweqnumb, rownfreeparms, rownumb, rowsof, sweep, trace, vec, vecdiag, },
	% Programming functions
	morekeywords=[7]{autocode, byteorder, c, _caller, chop, abs, clip, cond, e, fileexists, fileread, filereaderror, filewrite, float, fmtwidth, has_eprop, inlist, inrange, irecode, matrix, maxbyte, maxdouble, maxfloat, maxint, maxlong, mi, minbyte, mindouble, minfloat, minint, minlong, missing, r, recode, replay, return, s, scalar, smallestdouble,},
	% Random-number functions
	morekeywords=[8]{rbeta, rbinomial, rcauchy, rchi2, rexponential, rgamma, rhypergeometric, rigaussian, rlaplace, rlogistic, rnbinomial, rnormal, rpoisson, rt, runiform, runiformint, rweibull, rweibullph,},
	% Selecting time-span functions
	morekeywords=[9]{tin, twithin,},
	% Statistical functions
	morekeywords=[10]{betaden, binomial, binomialp, binomialtail, binormal, cauchy, cauchyden, cauchytail, chi2, chi2den, chi2tail, dgammapda, dgammapdada, dgammapdadx, dgammapdx, dgammapdxdx, dunnettprob, exponential, exponentialden, exponentialtail, F, Fden, Ftail, gammaden, gammap, gammaptail, hypergeometric, hypergeometricp, ibeta, ibetatail, igaussian, igaussianden, igaussiantail, invbinomial, invbinomialtail, invcauchy, invcauchytail, invchi2, invchi2tail, invdunnettprob, invexponential, invexponentialtail, invF, invFtail, invgammap, invgammaptail, invibeta, invibetatail, invigaussian, invigaussiantail, invlaplace, invlaplacetail, invlogistic, invlogistictail, invnbinomial, invnbinomialtail, invnchi2, invnF, invnFtail, invnibeta, invnormal, invnt, invnttail, invpoisson, invpoissontail, invt, invttail, invtukeyprob, invweibull, invweibullph, invweibullphtail, invweibulltail, laplace, laplaceden, laplacetail, lncauchyden, lnigammaden, lnigaussianden, lniwishartden, lnlaplaceden, lnmvnormalden, lnnormal, lnnormalden, lnwishartden, logistic, logisticden, logistictail, nbetaden, nbinomial, nbinomialp, nbinomialtail, nchi2, nchi2den, nchi2tail, nF, nFden, nFtail, nibeta, normal, normalden, npnchi2, npnF, npnt, nt, ntden, nttail, poisson, poissonp, poissontail, t, tden, ttail, tukeyprob, weibull, weibullden, weibullph, weibullphden, weibullphtail, weibulltail,},
	% String functions 
	morekeywords=[11]{abbrev, char, collatorlocale, collatorversion, indexnot, plural, plural, real, regexm, regexr, regexs, soundex, soundex_nara, strcat, strdup, string, strofreal, string, strofreal, stritrim, strlen, strlower, strltrim, strmatch, strofreal, strofreal, strpos, strproper, strreverse, strrpos, strrtrim, strtoname, strtrim, strupper, subinstr, subinword, substr, tobytes, uchar, udstrlen, udsubstr, uisdigit, uisletter, ustrcompare, ustrcompareex, ustrfix, ustrfrom, ustrinvalidcnt, ustrleft, ustrlen, ustrlower, ustrltrim, ustrnormalize, ustrpos, ustrregexm, ustrregexra, ustrregexrf, ustrregexs, ustrreverse, ustrright, ustrrpos, ustrrtrim, ustrsortkey, ustrsortkeyex, ustrtitle, ustrto, ustrtohex, ustrtoname, ustrtrim, ustrunescape, ustrupper, ustrword, ustrwordcount, usubinstr, usubstr, word, wordbreaklocale, worcount,},
	% Trig functions
	morekeywords=[12]{acos, acosh, asin, asinh, atan, atanh, cos, cosh, sin, sinh, tan, tanh,},
	morecomment=[l]{//},
	% morecomment=[l]{*},  // `*` maybe used as multiply operator. So use `//` as line comment.
	morecomment=[s]{/*}{*/},
	% The following is used by macros, like `lags'.
	morestring=[b]{`}{'},
	% morestring=[d]{'},
	morestring=[b]",
	morestring=[d]",
	% morestring=[d]{\\`},
	% morestring=[b]{'},
	sensitive=true,
}

\lstset{ 
	backgroundcolor=\color{white},   % choose the background color; you must add \usepackage{color} or \usepackage{xcolor}; should come as last argument
	basicstyle=\footnotesize\ttfamily,        % the size of the fonts that are used for the code
	breakatwhitespace=false,         % sets if automatic breaks should only happen at whitespace
	breaklines=true,                 % sets automatic line breaking
	captionpos=b,                    % sets the caption-position to bottom
	commentstyle=\color{mygreen},    % comment style
	deletekeywords={...},            % if you want to delete keywords from the given language
	escapeinside={\%*}{*)},          % if you want to add LaTeX within your code
	extendedchars=true,              % lets you use non-ASCII characters; for 8-bits encodings only, does not work with UTF-8
	firstnumber=0,                % start line enumeration with line 1000
	frame=single,	                   % adds a frame around the code
	keepspaces=true,                 % keeps spaces in text, useful for keeping indentation of code (possibly needs columns=flexible)
	keywordstyle=\color{blue},       % keyword style
	language=Octave,                 % the language of the code
	morekeywords={*,...},            % if you want to add more keywords to the set
	numbers=left,                    % where to put the line-numbers; possible values are (none, left, right)
	numbersep=5pt,                   % how far the line-numbers are from the code
	numberstyle=\tiny\color{mygray}, % the style that is used for the line-numbers
	rulecolor=\color{black},         % if not set, the frame-color may be changed on line-breaks within not-black text (e.g. comments (green here))
	showspaces=false,                % show spaces everywhere adding particular underscores; it overrides 'showstringspaces'
	showstringspaces=false,          % underline spaces within strings only
	showtabs=false,                  % show tabs within strings adding particular underscores
	stepnumber=2,                    % the step between two line-numbers. If it's 1, each line will be numbered
	stringstyle=\color{mymauve},     % string literal style
	tabsize=2,	                   % sets default tabsize to 2 spaces
%	title=\lstname,                   % show the filename of files included with \lstinputlisting; also try caption instead of title
	xleftmargin=0.25cm
}

% NOTE: To compile a version of this pset without problems, solutions, or reflections, uncomment the relevant line below.

%\excludeversion{problem}
%\excludeversion{solution}
%\excludeversion{reflection}

\begin{document}	
	
	% Use the \psetheader command at the beginning of a pset. 
	\psetheader

\section*{Problem 1}
Suppose $X\sim\text{Bernoulli}(p).$ 
\begin{enumerate}
    \item Show $\bbE[X^3] = p$
\begin{solution}
Using the following part, we let $k = 3$ and conclude
\end{solution}
\item Show $\bbE[X^k]= p$
\begin{solution}
    By definition, 
    \begin{align*}
        \bbE[X^k] &= 1^k \bbP\{X = 1\} + 0^k\bbP\{X = 0\}\\
        &= \boxed{p}
    \end{align*}
\end{solution}
\item Suppose that p = 0.3. Compute the mean, variance, skewness, and kurtosis of X.
\begin{solution}
    From previous part, $\bbE[X] = p = \boxed{0.3}.$ $\Var[X] = p(1-p) = \boxed{0.21}.$ To calculate the skewness, we compute 
    \begin{align*}
        \bbE\left[\left(\frac{X - \bbE[X]}{\sqrt{\bbV[X]}}\right)^3\right] &= \frac{1}{\bbV[X]^\frac{3}{2}}\bbE[(X - \mu)^3]\\
        &= \frac{1}{\bbV[X]^\frac{3}{2}}(\bbE[X^3] - 3\bbE[X^2]\bbE[X] + 2\bbE[X]^3\\
        &= \frac{p - 3p^2 + 2p^3}{(p(1-p))^\frac{3}{2}}\\
        &= \frac{p(1-p)(1-2p)}{(p(1-p))^\frac{3}{2}}\\
        &= \frac{1-2p}{\sqrt{1-p}}\\
        &\approx \boxed{0.87}
    \end{align*}
    Using similar algebra, we skip a few steps:
    \[\bbE\left[\left(\frac{X - \bbE[X]}{\sqrt{\bbV[X]}}\right)^4\right] = \frac{1-6p(1-p)}{p(1-p)} +3= \boxed{1.762}\]
\end{solution}
\end{enumerate}
In a population, $\mu_Y = 100$ and $\sigma_Y^2 = 43$. Use the Central Limit Theorem to answer the following questions:

\begin{enumerate}
    \item[(a)] In a random sample of size $n = 100$, find 
    \[
    \Pr\left(\overline{Y} \leq 101\right).
    \]
\begin{solution}
    We know by the CLT
    \[\frac{\sqrt{100}(\overline{Y} - 100)}{\sqrt{43}} \sim N(0,1)\] Hence 
    \[\bbP\{\overline{Y} \leq 101\} = \bbP\{\frac{10(\overline{Y} - 100)}{\sqrt{43}} \leq\frac{10(101 - 100)}{\sqrt{43}}\} = \bbP\{Z \leq \frac{10}{\sqrt{43}}\} \approx \boxed{0.94}\]
\end{solution}

    \item[(b)] In a random sample of size $n = 165$, find 
    \[
    \Pr\left(\overline{Y} > 98\right).
    \]
\begin{solution}
    Using similar logic to the above problem, we find that 
    \[Z = \frac{\sqrt{165}(\overline{Y} - 100)}{\sqrt{43}} \sim N(0,1)\] so then 
    \[\bbP\{Z \geq  \frac{\sqrt{165}(98 - 100)}{\sqrt{43}}\}\approx \boxed{1}\]

\end{solution}

    \item[(c)] In a random sample of size $n = 64$, find 
    \[
    \Pr\left(101 \leq \overline{Y} \leq 103\right).
    \]
    \begin{solution}
\[Z = \frac{8(\overline{Y} - 100)}{\sqrt{43}}\sim N(0,1)\] and thus 
\[\bbP\{\frac{8(101 - 100)}{\sqrt{43}} \leq Z \leq \frac{8(103 - 100)}{\sqrt{43}}\}
 = \boxed{0.1111}\]
    \end{solution}
\end{enumerate}


\newpage
\section*{Problem 2}
\begin{problem}
    Show that $\bbE[Y\mid X]$ is minimizes 
    \[\min_{g(X)} \bbE[(Y - g(X))^2]\]
\end{problem}
\begin{solution}
We claim that $Y - \bbE[Y \mid X] \perp \bbE[Y \mid X] - g(X).$ To see this, we note that 
\begin{align*}
 \langle Y - \bbE[Y \mid X], \bbE[Y \mid X] - g(X)\rangle &= \langle Y, \bbE[Y \mid X]\rangle - \langle Y, g(X)\rangle -\langle \bbE[Y \mid X], \bbE[Y \mid X] \rangle + \langle \bbE[Y \mid X], g(X)\rangle\\
 &= \bbE[Y\bbE[Y \mid X]] - \bbE[Y g(X)] - \bbE[\bbE[Y \mid X]^2] + \bbE[\bbE[Y \mid X]g(X)]\\
 &= \bbE[\bbE[Y \bbE[Y \mid X] \mid X]] - \bbE[\bbE[Y g(X) \mid X]] - \bbE[\bbE[Y \mid X]^2] + \bbE[\bbE[Y \mid X]g(X)]\\
 &= \bbE[\bbE[Y \mid X]^2] - \bbE[g(X) \bbE[Y \mid X]] - \bbE[\bbE[Y \mid X]^2] +  \bbE[g(X)\bbE[Y \mid X]]\\
 &= 0
\end{align*}
Here, we use the fact that $g(X)$ is $X-$measurable, and thus we pull it out of the conditional expectation. We also made heavy use of LIE. By orthogonality, we can use the Pythagorean theorem


Computing, 
\begin{align*}
    \bbE[(Y - g(X))^2] &= \bbE[(Y - \bbE[Y \mid X]+ \bbE[Y \mid X] - g(X))^2]\\
    &= \|(Y - \bbE[Y \mid X])+ (\bbE[Y \mid X] - g(X))\|^2\\
    &= \|Y - \bbE[Y \mid X]\|^2 + \|\bbE[Y \mid X] - g(X)\|^2
\end{align*}
Clearly, the left hand side will be minimized when \[\|\bbE[Y \mid X] - g(X)\|^2 = 0\iff \bbE[Y \mid X] - g(X) = 0 \iff g(X) = \bbE[Y \mid X]\]
\end{solution}

\newpage
\section*{Problem 3}
\begin{problem}
    Prove that 
    \[\Cov(X,Y) = \bbE[XY] - \bbE[X]\bbE[Y]\]
\end{problem}
\begin{solution}
    By definition,
    \begin{align*}
        \Cov(X,Y) &\equiv \bbE[(X - \bbE[X])(Y - \bbE[Y])]\\
        &= \bbE[XY - Y\bbE[X] - X\bbE[Y] + \bbE[X]\bbE[Y]\\
        &= \bbE[XY] - \bbE[Y \bbE[X]] - \bbE[X \bbE[Y]] + \bbE[\bbE[X]\bbE[Y]]\\
        &= \bbE[XY] - \bbE[Y]\bbE[X] + (-\bbE[X]\bbE[Y] + \bbE[X]\bbE[Y])\\
        &= \bbE[XY] - \bbE[Y]\bbE[X]
    \end{align*}
    Here we make heavy use of the fact that $\bbE[Z]$ is a constant and can thus be pulled out of the expectation.
\end{solution}

\newpage
\section*{Problem 4}
\begin{problem}
    Suppose that in the State of Illinois the written exam for a drivers license consists of 10 multiple-choice
 questions. Each question has 4 possible choices, only one of which is correct. Passing requires answering
 at least 5 questions correctly. What is the probability a student-driver passes his exam by “guessing
 randomly” on each question?
\end{problem}
\begin{solution}
    Let $X \sim \text{Binomial}(10, \frac{1}{4}).$ It should be clear that the solution to this problem is the same as finding 
\begin{align*}
    \bbP[X \geq 5] &= 1 - (\sum_{k=0}^4 \binom{10}{k}(\frac{1}{4})^k(\frac{3}{4})^{10-k}) = \boxed{0.0781}
\end{align*}
where the numerical answer was found using software.

\end{solution}

\newpage
\section*{Problem 5}
\begin{problem}
    Let $X \sim \text{Bernoulli}(p).$ Define $Z = 3^X -1.$ 
\begin{itemize}
    \item Is $Z$ a random variable. Why?
    \begin{solution}
        Yes, functions of random variabls are clearly random variables. To be precise, we assum that $(\Omega, \mathcal{F}, \bbP)$ is probability space, then $X: \Omega \to \bbR$ is $\mathcal{F}$ measurable. Since $Y = g(X)$ where $g$ is continuous, then clearly $g$ is $\cal F$ measurable and thus a random variable.
    \end{solution}
    \item Show that $\bbE[Z] = 2p.$
    \begin{solution}
        \begin{align*}
            \bbE[Z] &= \bbE[3^X -1]\\
            &= \bbE[3^X] -1\\
            &= 3^1 p + 3^0 (1-p) - 1\\
            &= 3p + 1 - p -1 \\
            &= 1+2p -1 \\
            &= \boxed{2p}
        \end{align*}
    \end{solution}
    \item Show that $\bbE[Z^2] = 4p$
    \begin{solution}
    We use work from the previous part to find that
        \begin{align*}
            \bbE[Z^2] &= \bbE[(3^X -1)^2]\\
            &= \bbE[3^{2X} - 2\cdot 3^X + 1]\\
            &= \bbE[3^{2X}] - 2\bbE[3^X] +1\\
            &= 1+2p- 2(1+2p) + 1\\
            &= \boxed{4p}
        \end{align*}
    \end{solution}
    \item 
    Find $\Var(Z)$
    \begin{solution}
        We use the identity and the previous problems to find that
        \[\Var(Z) = \bbE[Z^2] - \bbE[Z]^2 = \boxed{4p - 4p^2}\]
    \end{solution}
\end{itemize}
\end{problem}

\newpage
\section*{Problem 6}
\begin{problem}
    Let \texttt{GPA} denote a random variable for a college student’s grade point average, and \texttt{SAT} denote a
random variable for the college student’s SAT score. Suppose that there is the following relationship
between \texttt{GPA} and \texttt{SAT}:
\[
\mathbb{E}[\texttt{GPA} \mid \texttt{SAT}] = 0.70 + 0.002 \cdot \texttt{SAT}.
\]

\begin{enumerate}
    \item[(a)] What is the expected GPA when \texttt{SAT} = 750? What is the expected GPA when \texttt{SAT} = 1500?
\begin{solution}
    Plugging in, 
\[\bbE[\texttt{GPA} \mid 750] = 0.70 + 0.002 \cdot 750 = \boxed{2.20}\]
\[\bbE[\texttt{GPA} \mid 1500] = 0.70 + 0.002 \cdot 1500 = \boxed{3.70}\]
\end{solution}
    \item[(b)] If \( \mathbb{E}[\texttt{SAT}] = 1000 \), what is \( \mathbb{E}[\texttt{GPA}] \)?
\begin{solution}
    Using the Law of Total Expectation:
    \[\bbE[\texttt{GPA}] = \bbE[\bbE[\texttt{GPA}\mid \texttt{SAT}] = \bbE[0.70 + 0.002 \cdot \texttt{SAT}] = 0.70 + 0.002\bbE[\texttt{SAT}] = \boxed{2.70}\]
\end{solution}
\end{enumerate}

\end{problem}


\newpage
\section*{Problem 7}
\begin{problem}
    Suppose \( X \sim \text{Unif}(-1,1) \), and let \( Y = X^2 \). Show that \( \operatorname{Cov}[X,Y] = 0 \), but that \( X \) and \( Y \) are not independent.
\end{problem}
\begin{solution}
    By Problem 3,
\[\Cov(X,Y) = \bbE[XY] - \bbE[X]\bbE[Y] = \bbE[X^3] - \bbE[X]\bbE[X^2]\] We compute, noting that the pdf of $X$ is 
\[f_X(x) = \frac{1}{2}\]
\[\bbE[X] = \frac{-1 + 1}{2} = 0\]
\[\bbE[X^2] = \int_{\bbR} X^2\,d\bbP = \int_{-1}^1 \frac{1}{2}x^2 \,dx =  \frac{1}{3}\]
\[\bbE[X^3] = \int_{\bbR} X^3\,d\bbP = \int_{-1}^1 \frac{1}{2}x^3 \,dx = \frac{1}{4}x^4 \bigg|_{-1}^1 = \frac{1}{4} - \frac{1}{4} = 0\]

Hence, the covariance is zero. To see that $X$ and $Y$ are not independent, simply compare the joint densities. We know that $X$ and $Y$ are independent iff
\begin{align}
f_{X,Y}(x,y) = f_X(x)f_Y(y)    
\end{align}

We know that $f_X(x) = \frac{1}{2}.$ Moreover, supposing that $y \in (-1,1),$
\[F_Y(y) = \bbP\{Y \leq y\} = \bbP\{X^2 \leq y\} = \bbP\{-\sqrt{y} \leq X \leq \sqrt{y}\} = \bbP\{X \leq \sqrt{y}\} - \bbP\{X \leq -\sqrt{y}\} =
\sqrt{y}.\] Hence, $f_Y(y) = \frac{1}{2\sqrt{y}}.$ Thus the right hand side of (1) is $\frac{1}{4\sqrt{y}}.$ To compute the left hand side, first we compute the conditional density 
\[f_{Y \mid X}(y)\] 
\end{solution}

\newpage
\section*{Problem 8}
\begin{problem}
    Let \( X \) be a random variable such that
\[
\mathbb{P}(X = -1) = \mathbb{P}(X = 0) = \mathbb{P}(X = 1) = \frac{1}{3}.
\]

Let \( Y \) be another random variable such that
\[
\mathbb{E}[Y \mid X] = 3 + 3X \quad \text{and} \quad \operatorname{Var}(Y \mid X) = 3.
\]
For each question below, provide a numerical answer and show your work.

\begin{enumerate}[label=(\alph*)]
    \item What is \( \mathbb{E}[X] \)?
\begin{solution}
    \[\bbE[X] = -1 \cdot \frac{1}{3} + 0 \cdot \frac{1}{3} + 1 \cdot \frac{1}{3} = \boxed{0}\]
\end{solution}
    \item What is \( \mathbb{E}[X^2] \)?
\begin{solution}
    \[\bbE[X^2] = 2(1\cdot \frac{1}{3}) + 0 \cdot \frac{1}{3} = \boxed{\frac{2}{3}}\]
\end{solution}
    \item What is \( \mathbb{E}[Y] \)?
    \begin{solution}
        
\[\bbE[Y] = \bbE[\bbE[Y \mid X]] = \bbE[3 + 3X]] = 3 + 3\bbE[X] = \boxed{3}\]
    \end{solution}

    \item What is \( \operatorname{Var}[2 + \mathbb{E}[X]] \)?
    \begin{solution}
        Bro what the variance of a number is just zero.
    \end{solution}
    \item What is \( \operatorname{Var}[X] \)?
    \begin{solution}
        \[\Var(X) = \bbE[X^2] - \bbE[X]^2 =\boxed{ \frac{2}{3}}\]
    \end{solution}
    \item What is \( \operatorname{Var}[Y] \)?
    \begin{solution}
\begin{align*}
\Var(Y) &= \bbE[\Var(Y \mid X)] + \Var(\bbE[Y \mid X])    \\
&= \bbE[3] + \Var(3 + 3X)\\
&= 3 + 9\cdot\Var(X)\\
&= \boxed{9}
\end{align*}
    \end{solution}
    \item What is \( \operatorname{Cov}[X, Y] \)?
\begin{solution}
\begin{align*}
\Cov(X,Y) &= \bbE[XY] - \bbE[X] \bbE[Y] \\
&= \bbE[3X + 3X^2] - \bbE[X]\bbE[Y]\\
&=  3\bbE[X^2] \\
&= \boxed{2}
\end{align*}

\end{solution}
    \item What is \( \operatorname{Corr}[X, Y] \)?
    \begin{solution}
\[\Corr(X,Y) = \frac{\Cov(X,Y)}{\sqrt{\Var(X)}\sqrt{\Var(Y)}} = \frac{2}{\sqrt{6}} \]
    \end{solution}
    \item Is \( Y \) mean independent of \( X \)? Explain briefly.
\begin{solution}
Nah. Assume, for the sake of contradiction, that it is mean independent. Then $\Corr(X,Y) = 0.$ This is a contradiction to part $h$.
\end{solution}
    \item Is \( Y \) independent of \( X \)? Explain briefly.
\begin{solution}
    Nah, same reason as part i.
\end{solution}
\end{enumerate}
\end{problem}


\end{document}