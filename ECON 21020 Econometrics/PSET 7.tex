\documentclass[11pt]{article}
\usepackage{float}

% NOTE: Add in the relevant information to the commands below; or, if you'll be using the same information frequently, add these commands at the top of paolo-pset.tex file. 
\newcommand{\name}{Agustín Esteva \heart}
\newcommand{\email}{aesteva@uchicago.edu}
\newcommand{\classnum}{20210}
\newcommand{\subject}{Econometrics \heart}
\newcommand{\instructors}{Murilo Ramos}
\newcommand{\assignment}{Problem Set 4}
\newcommand{\semester}{Summer 2025}
\newcommand{\duedate}{\today}
\newcommand{\bA}{\mathbf{A}}
\newcommand{\bB}{\mathbf{B}}
\newcommand{\bC}{\mathbf{C}}
\newcommand{\bD}{\mathbf{D}}
\newcommand{\bE}{\mathbf{E}}
\newcommand{\bF}{\mathbf{F}}
\newcommand{\bG}{\mathbf{G}}
\newcommand{\bH}{\mathbf{H}}
\newcommand{\bI}{\mathbf{I}}
\newcommand{\bJ}{\mathbf{J}}
\newcommand{\bK}{\mathbf{K}}
\newcommand{\bL}{\mathbf{L}}
\newcommand{\bM}{\mathbf{M}}
\newcommand{\bN}{\mathbf{N}}
\newcommand{\bO}{\mathbf{O}}
\newcommand{\bP}{\mathbf{P}}
\newcommand{\bQ}{\mathbf{Q}}
\newcommand{\bR}{\mathbf{R}}
\newcommand{\bS}{\mathbf{S}}
\newcommand{\bT}{\mathbf{T}}
\newcommand{\bU}{\mathbf{U}}
\newcommand{\bV}{\mathbf{V}}
\newcommand{\bW}{\mathbf{W}}
\newcommand{\bX}{\mathbf{X}}
\newcommand{\bY}{\mathbf{Y}}
\newcommand{\bZ}{\mathbf{Z}}
\newcommand{\Vol}{\text{Vol}}
\newcommand{\MSE}{\text{MSE}}
\newcommand{\Bias}{\text{Bias}}
\newcommand{\Var}{\text{Var}}
\newcommand{\Cov}{\text{Cov}}
\newcommand{\Corr}{\text{Corr}}



%% blackboard bold math capitals
\newcommand{\bbA}{\mathbb{A}}
\newcommand{\bbB}{\mathbb{B}}
\newcommand{\bbC}{\mathbb{C}}
\newcommand{\bbD}{\mathbb{D}}
\newcommand{\bbE}{\mathbb{E}}
\newcommand{\bbF}{\mathbb{F}}
\newcommand{\bbG}{\mathbb{G}}
\newcommand{\bbH}{\mathbb{H}}
\newcommand{\bbI}{\mathbb{I}}
\newcommand{\bbJ}{\mathbb{J}}
\newcommand{\bbK}{\mathbb{K}}
\newcommand{\bbL}{\mathbb{L}}
\newcommand{\bbM}{\mathbb{M}}
\newcommand{\bbN}{\mathbb{N}}
\newcommand{\bbO}{\mathbb{O}}
\newcommand{\bbP}{\mathbb{P}}
\newcommand{\bbQ}{\mathbb{Q}}
\newcommand{\bbR}{\mathbb{R}}
\newcommand{\bbS}{\mathbb{S}}
\newcommand{\bbT}{\mathbb{T}}
\newcommand{\bbU}{\mathbb{U}}
\newcommand{\bbV}{\mathbb{V}}
\newcommand{\bbW}{\mathbb{W}}
\newcommand{\bbX}{\mathbb{X}}
\newcommand{\bbY}{\mathbb{Y}}
\newcommand{\bbZ}{\mathbb{Z}}

%% script math capitals
\newcommand{\sA}{\mathscr{A}}
\newcommand{\sB}{\mathscr{B}}
\newcommand{\sC}{\mathscr{C}}
\newcommand{\sD}{\mathscr{D}}
\newcommand{\sE}{\mathscr{E}}
\newcommand{\sF}{\mathscr{F}}
\newcommand{\sG}{\mathscr{G}}
\newcommand{\sH}{\mathscr{H}}
\newcommand{\sI}{\mathscr{I}}
\newcommand{\sJ}{\mathscr{J}}
\newcommand{\sK}{\mathscr{K}}
\newcommand{\sL}{\mathscr{L}}
\newcommand{\sM}{\mathscr{M}}
\newcommand{\sN}{\mathscr{N}}
\newcommand{\sO}{\mathscr{O}}
\newcommand{\sP}{\mathscr{P}}
\newcommand{\sQ}{\mathscr{Q}}
\newcommand{\sR}{\mathscr{R}}
\newcommand{\sS}{\mathscr{S}}
\newcommand{\sT}{\mathscr{T}}
\newcommand{\sU}{\mathscr{U}}
\newcommand{\sV}{\mathscr{V}}
\newcommand{\sW}{\mathscr{W}}
\newcommand{\sX}{\mathscr{X}}
\newcommand{\sY}{\mathscr{Y}}
\newcommand{\sZ}{\mathscr{Z}}


\renewcommand{\emptyset}{\O}

\newcommand{\abs}[1]{\lvert #1 \rvert}
\newcommand{\norm}[1]{\lVert #1 \rVert}
\newcommand{\sm}{\setminus}



\newcommand{\sarr}{\rightarrow}
\newcommand{\arr}{\longrightarrow}

% NOTE: Defining collaborators is optional; to not list collaborators, comment out the line below.
%\newcommand{\collaborators}{Alyssa P. Hacker (\texttt{aphacker}), Ben Bitdiddle (\texttt{bitdiddle})}

% Copyright 2021 Paolo Adajar (padajar.com, paoloadajar@mit.edu)
% 
% Permission is hereby granted, free of charge, to any person obtaining a copy of this software and associated documentation files (the "Software"), to deal in the Software without restriction, including without limitation the rights to use, copy, modify, merge, publish, distribute, sublicense, and/or sell copies of the Software, and to permit persons to whom the Software is furnished to do so, subject to the following conditions:
%
% The above copyright notice and this permission notice shall be included in all copies or substantial portions of the Software.
% 
% THE SOFTWARE IS PROVIDED "AS IS", WITHOUT WARRANTY OF ANY KIND, EXPRESS OR IMPLIED, INCLUDING BUT NOT LIMITED TO THE WARRANTIES OF MERCHANTABILITY, FITNESS FOR A PARTICULAR PURPOSE AND NONINFRINGEMENT. IN NO EVENT SHALL THE AUTHORS OR COPYRIGHT HOLDERS BE LIABLE FOR ANY CLAIM, DAMAGES OR OTHER LIABILITY, WHETHER IN AN ACTION OF CONTRACT, TORT OR OTHERWISE, ARISING FROM, OUT OF OR IN CONNECTION WITH THE SOFTWARE OR THE USE OR OTHER DEALINGS IN THE SOFTWARE.

\usepackage{fullpage}
\usepackage{enumitem}
\usepackage{amsfonts, amssymb, amsmath,amsthm}
\usepackage{mathtools}
\usepackage[pdftex, pdfauthor={\name}, pdftitle={\classnum~\assignment}]{hyperref}
\usepackage[dvipsnames]{xcolor}
\usepackage{bbm}
\usepackage{graphicx}
\usepackage{mathrsfs}
\usepackage{pdfpages}
\usepackage{tabularx}
\usepackage{pdflscape}
\usepackage{makecell}
\usepackage{booktabs}
\usepackage{natbib}
\usepackage{caption}
\usepackage{subcaption}
\usepackage{physics}
\usepackage[many]{tcolorbox}
\usepackage{version}
\usepackage{ifthen}
\usepackage{cancel}
\usepackage{listings}
\usepackage{courier}

\usepackage{tikz}
\usepackage{istgame}

\hypersetup{
	colorlinks=true,
	linkcolor=blue,
	filecolor=magenta,
	urlcolor=blue,
}

\setlength{\parindent}{0mm}
\setlength{\parskip}{2mm}

\setlist[enumerate]{label=({\alph*})}
\setlist[enumerate, 2]{label=({\roman*})}

\allowdisplaybreaks[1]

\newcommand{\psetheader}{
	\ifthenelse{\isundefined{\collaborators}}{
		\begin{center}
			{\setlength{\parindent}{0cm} \setlength{\parskip}{0mm}
				
				{\textbf{\classnum~\semester:~\assignment} \hfill \name}
				
				\subject \hfill \href{mailto:\email}{\tt \email}
				
				Instructor(s):~\instructors \hfill Due Date:~\duedate	
				
				\hrulefill}
		\end{center}
	}{
		\begin{center}
			{\setlength{\parindent}{0cm} \setlength{\parskip}{0mm}
				
				{\textbf{\classnum~\semester:~\assignment} \hfill \name\footnote{Collaborator(s): \collaborators}}
				
				\subject \hfill \href{mailto:\email}{\tt \email}
				
				Instructor(s):~\instructors \hfill Due Date:~\duedate	
				
				\hrulefill}
		\end{center}
	}
}

\renewcommand{\thepage}{\classnum~\assignment \hfill \arabic{page}}
\newcommand{\uconv}{\overset{}{\rightrightarrows}}
\makeatletter
\def\points{\@ifnextchar[{\@with}{\@without}}
\def\@with[#1]#2{{\ifthenelse{\equal{#2}{1}}{{[1 point, #1]}}{{[#2 points, #1]}}}}
\def\@without#1{\ifthenelse{\equal{#1}{1}}{{[1 point]}}{{[#1 points]}}}
\makeatother

\newtheoremstyle{theorem-custom}%
{}{}%
{}{}%
{\itshape}{.}%
{ }%
{\thmname{#1}\thmnumber{ #2}\thmnote{ (#3)}}

\theoremstyle{theorem-custom}

\newtheorem{theorem}{Theorem}
\newtheorem{lemma}[theorem]{Lemma}
\newtheorem{example}[theorem]{Example}

\newenvironment{problem}[1]{\color{black} #1}{}

\newenvironment{solution}{%
	\leavevmode\begin{tcolorbox}[breakable, colback=green!5!white,colframe=green!75!black, enhanced jigsaw] \proof[\scshape Solution:] \setlength{\parskip}{2mm}%
	}{\renewcommand{\qedsymbol}{$\heartsuit$} \endproof \end{tcolorbox}}

\newenvironment{reflection}{\begin{tcolorbox}[breakable, colback=black!8!white,colframe=black!60!white, enhanced jigsaw, parbox = false]\textsc{Reflections:}}{\end{tcolorbox}}

\newcommand{\qedh}{\renewcommand{\qedsymbol}{$\blacksquare$}\qedhere}

\definecolor{mygreen}{rgb}{0,0.6,0}
\definecolor{mygray}{rgb}{0.5,0.5,0.5}
\definecolor{mymauve}{rgb}{0.58,0,0.82}

% from https://github.com/satejsoman/stata-lstlisting
% language definition
\lstdefinelanguage{Stata}{
	% System commands
	morekeywords=[1]{regress, reg, summarize, sum, display, di, generate, gen, bysort, use, import, delimited, predict, quietly, probit, margins, test},
	% Reserved words
	morekeywords=[2]{aggregate, array, boolean, break, byte, case, catch, class, colvector, complex, const, continue, default, delegate, delete, do, double, else, eltypedef, end, enum, explicit, export, external, float, for, friend, function, global, goto, if, inline, int, local, long, mata, matrix, namespace, new, numeric, NULL, operator, orgtypedef, pointer, polymorphic, pragma, private, protected, public, quad, real, return, rowvector, scalar, short, signed, static, strL, string, struct, super, switch, template, this, throw, transmorphic, try, typedef, typename, union, unsigned, using, vector, version, virtual, void, volatile, while,},
	% Keywords
	morekeywords=[3]{forvalues, foreach, set},
	% Date and time functions
	morekeywords=[4]{bofd, Cdhms, Chms, Clock, clock, Cmdyhms, Cofc, cofC, Cofd, cofd, daily, date, day, dhms, dofb, dofC, dofc, dofh, dofm, dofq, dofw, dofy, dow, doy, halfyear, halfyearly, hh, hhC, hms, hofd, hours, mdy, mdyhms, minutes, mm, mmC, mofd, month, monthly, msofhours, msofminutes, msofseconds, qofd, quarter, quarterly, seconds, ss, ssC, tC, tc, td, th, tm, tq, tw, week, weekly, wofd, year, yearly, yh, ym, yofd, yq, yw,},
	% Mathematical functions
	morekeywords=[5]{abs, ceil, cloglog, comb, digamma, exp, expm1, floor, int, invcloglog, invlogit, ln, ln1m, ln, ln1p, ln, lnfactorial, lngamma, log, log10, log1m, log1p, logit, max, min, mod, reldif, round, sign, sqrt, sum, trigamma, trunc,},
	% Matrix functions
	morekeywords=[6]{cholesky, coleqnumb, colnfreeparms, colnumb, colsof, corr, det, diag, diag0cnt, el, get, hadamard, I, inv, invsym, issymmetric, J, matmissing, matuniform, mreldif, nullmat, roweqnumb, rownfreeparms, rownumb, rowsof, sweep, trace, vec, vecdiag, },
	% Programming functions
	morekeywords=[7]{autocode, byteorder, c, _caller, chop, abs, clip, cond, e, fileexists, fileread, filereaderror, filewrite, float, fmtwidth, has_eprop, inlist, inrange, irecode, matrix, maxbyte, maxdouble, maxfloat, maxint, maxlong, mi, minbyte, mindouble, minfloat, minint, minlong, missing, r, recode, replay, return, s, scalar, smallestdouble,},
	% Random-number functions
	morekeywords=[8]{rbeta, rbinomial, rcauchy, rchi2, rexponential, rgamma, rhypergeometric, rigaussian, rlaplace, rlogistic, rnbinomial, rnormal, rpoisson, rt, runiform, runiformint, rweibull, rweibullph,},
	% Selecting time-span functions
	morekeywords=[9]{tin, twithin,},
	% Statistical functions
	morekeywords=[10]{betaden, binomial, binomialp, binomialtail, binormal, cauchy, cauchyden, cauchytail, chi2, chi2den, chi2tail, dgammapda, dgammapdada, dgammapdadx, dgammapdx, dgammapdxdx, dunnettprob, exponential, exponentialden, exponentialtail, F, Fden, Ftail, gammaden, gammap, gammaptail, hypergeometric, hypergeometricp, ibeta, ibetatail, igaussian, igaussianden, igaussiantail, invbinomial, invbinomialtail, invcauchy, invcauchytail, invchi2, invchi2tail, invdunnettprob, invexponential, invexponentialtail, invF, invFtail, invgammap, invgammaptail, invibeta, invibetatail, invigaussian, invigaussiantail, invlaplace, invlaplacetail, invlogistic, invlogistictail, invnbinomial, invnbinomialtail, invnchi2, invnF, invnFtail, invnibeta, invnormal, invnt, invnttail, invpoisson, invpoissontail, invt, invttail, invtukeyprob, invweibull, invweibullph, invweibullphtail, invweibulltail, laplace, laplaceden, laplacetail, lncauchyden, lnigammaden, lnigaussianden, lniwishartden, lnlaplaceden, lnmvnormalden, lnnormal, lnnormalden, lnwishartden, logistic, logisticden, logistictail, nbetaden, nbinomial, nbinomialp, nbinomialtail, nchi2, nchi2den, nchi2tail, nF, nFden, nFtail, nibeta, normal, normalden, npnchi2, npnF, npnt, nt, ntden, nttail, poisson, poissonp, poissontail, t, tden, ttail, tukeyprob, weibull, weibullden, weibullph, weibullphden, weibullphtail, weibulltail,},
	% String functions 
	morekeywords=[11]{abbrev, char, collatorlocale, collatorversion, indexnot, plural, plural, real, regexm, regexr, regexs, soundex, soundex_nara, strcat, strdup, string, strofreal, string, strofreal, stritrim, strlen, strlower, strltrim, strmatch, strofreal, strofreal, strpos, strproper, strreverse, strrpos, strrtrim, strtoname, strtrim, strupper, subinstr, subinword, substr, tobytes, uchar, udstrlen, udsubstr, uisdigit, uisletter, ustrcompare, ustrcompareex, ustrfix, ustrfrom, ustrinvalidcnt, ustrleft, ustrlen, ustrlower, ustrltrim, ustrnormalize, ustrpos, ustrregexm, ustrregexra, ustrregexrf, ustrregexs, ustrreverse, ustrright, ustrrpos, ustrrtrim, ustrsortkey, ustrsortkeyex, ustrtitle, ustrto, ustrtohex, ustrtoname, ustrtrim, ustrunescape, ustrupper, ustrword, ustrwordcount, usubinstr, usubstr, word, wordbreaklocale, worcount,},
	% Trig functions
	morekeywords=[12]{acos, acosh, asin, asinh, atan, atanh, cos, cosh, sin, sinh, tan, tanh,},
	morecomment=[l]{//},
	% morecomment=[l]{*},  // `*` maybe used as multiply operator. So use `//` as line comment.
	morecomment=[s]{/*}{*/},
	% The following is used by macros, like `lags'.
	morestring=[b]{`}{'},
	% morestring=[d]{'},
	morestring=[b]",
	morestring=[d]",
	% morestring=[d]{\\`},
	% morestring=[b]{'},
	sensitive=true,
}

\lstset{ 
	backgroundcolor=\color{white},   % choose the background color; you must add \usepackage{color} or \usepackage{xcolor}; should come as last argument
	basicstyle=\footnotesize\ttfamily,        % the size of the fonts that are used for the code
	breakatwhitespace=false,         % sets if automatic breaks should only happen at whitespace
	breaklines=true,                 % sets automatic line breaking
	captionpos=b,                    % sets the caption-position to bottom
	commentstyle=\color{mygreen},    % comment style
	deletekeywords={...},            % if you want to delete keywords from the given language
	escapeinside={\%*}{*)},          % if you want to add LaTeX within your code
	extendedchars=true,              % lets you use non-ASCII characters; for 8-bits encodings only, does not work with UTF-8
	firstnumber=0,                % start line enumeration with line 1000
	frame=single,	                   % adds a frame around the code
	keepspaces=true,                 % keeps spaces in text, useful for keeping indentation of code (possibly needs columns=flexible)
	keywordstyle=\color{blue},       % keyword style
	language=Octave,                 % the language of the code
	morekeywords={*,...},            % if you want to add more keywords to the set
	numbers=left,                    % where to put the line-numbers; possible values are (none, left, right)
	numbersep=5pt,                   % how far the line-numbers are from the code
	numberstyle=\tiny\color{mygray}, % the style that is used for the line-numbers
	rulecolor=\color{black},         % if not set, the frame-color may be changed on line-breaks within not-black text (e.g. comments (green here))
	showspaces=false,                % show spaces everywhere adding particular underscores; it overrides 'showstringspaces'
	showstringspaces=false,          % underline spaces within strings only
	showtabs=false,                  % show tabs within strings adding particular underscores
	stepnumber=2,                    % the step between two line-numbers. If it's 1, each line will be numbered
	stringstyle=\color{mymauve},     % string literal style
	tabsize=2,	                   % sets default tabsize to 2 spaces
%	title=\lstname,                   % show the filename of files included with \lstinputlisting; also try caption instead of title
	xleftmargin=0.25cm
}

% NOTE: To compile a version of this pset without problems, solutions, or reflections, uncomment the relevant line below.

%\excludeversion{problem}
%\excludeversion{solution}
%\excludeversion{reflection}

\begin{document}	

	% Use the \psetheader command at the beginning of a pset. 
	\psetheader
\section*{Problem 1}

\section*{Problem 1}
Stock and Watson, Exercises 12.4, 12.5, 12.8, 12.9, *E12.1
\begin{enumerate}
    \item (12.4) Consider TSLS estimation with a single included endogenous variable and a single instrument. Then the predicted value from the first-stage regression is 
\[
\hat{X}_i = \hat{\pi}_0 + \hat{\pi}_1 Z_i.
\]
Use the definition of the sample variance and covariance to show that 
\[
s_{\hat{X}Y} = \hat{\pi}_1 s_{ZY} \quad \text{and} \quad s_{\hat{X}}^2 = \hat{\pi}_1^2 s_Z^2.
\]
Use this result to fill in the steps of the derivation of Equation (12.4) in Appendix 12.2.
\begin{solution}
    We see that 
    \begin{align*}
        s_{\hat X Y} &= \frac{1}{n}\sum (\hat X_i - \bar{\hat X})(Y_i - \bar Y)\\
        &=\frac{1}{n}\sum \hat X_i(Y_i - \bar Y)\\
        &= \frac{1}{n}\sum (\hat \pi_0 + \hat \pi_1 Z_i)(Y_i - \bar Y)\\
        &= \hat\pi_0\frac{1}{n}\sum (Y_i - \bar Y) + \hat\pi_1 \frac{1}{n}\sum Z_i(Y_i - \bar Y)\\
        &= \hat\pi_1\frac{1}{n}\sum (Z_i - \bar Z)(Y_i - \bar Y)\\
        &= \hat\pi_1 s_{ZY}
    \end{align*} and
    \begin{align*}
        s^2_{\hat X} &= \frac{1}{n}\sum (\hat X_i - \bar{\hat X})^2\\
        &= \frac{1}{n}\sum (\hat X_i - \bar{\hat X})\hat X_i\\
        &= \frac{1}{n}\sum (\pi_0 + \hat \pi_1 Z_i - \frac{1}{n}\sum (\hat \pi_0 + \hat \pi_1 Z_i))(\hat \pi_0 + \hat \pi_1 Z_i)\\
        &= \frac{1}{n}\sum (\hat \pi_1 Z_i - \hat \pi_1 \bar Z_i)(\hat \pi_0 + \hat \pi_1 Z_i)\\
        &= \frac{1}{n}\sum (\hat \pi_1 Z_i - \hat \pi_1 \bar Z_i)^2\\
        &= \hat\pi_1^2 \frac{1}{n}\sum (Z_i - \bar Z)^2\\
        &= (\hat \pi_1)^2 s^2_Z
    \end{align*}
    To see that $\hat \beta_1^{2SLS} = \hat\beta_1^{IV},$
    \begin{align*}
        \hat \beta_1^{2SLS} &= \frac{s_{\hat X Y}}{s^2_{\hat X}}\\
        &= \frac{\hat\pi_1 s_{ZY}}{(\hat \pi_1)^2 s^2_Z}\\
        &= \frac{s_{ZY}}{\hat \pi_1 s^2_Z}
    \end{align*}
    Hence, it suffice to show that the denominator is $s^2_{ZX}:$
    \begin{align*}
        \hat\pi_1s^2_Z &=\frac{s_{ZX}}{s^2_{Z}}s^2_Z = s_{ZX}
    \end{align*}
\end{solution}
\item Consider the IV regression model
\[
Y_i = \beta_0 + \beta_1 X_i + \beta_2 W_i + u_i,
\]
where \( X_i \) is correlated with \( u_i \), and \( Z_i \) is an instrument. Suppose that the first three assumptions in Key Concept 12.4 are satisfied. Which IV assumption is \textbf{not} satisfied when

\begin{enumerate}[label=\textbf{\alph*.}]
    \item \( Z_i \) is independent of \( (Y_i, X_i, W_i) \)?
    \begin{solution}
        Relevance
    \end{solution}
    \item \( Z_i = W_i \)?
    \begin{solution}
        $Z$ cannot affect $Y$ thru anything other than $X.$
    \end{solution}
    \item \( W_i = 1 \) for all \( i \)?
    \begin{solution}
        Co-linearity
    \end{solution}
    \item \( Z_i = X_i \)?
    \begin{solution}
        Exogeneity
    \end{solution}
\end{enumerate}
\item Consider a product market with a supply function \( Q_i^s = \beta_0 + \beta_1 P_i + u_i^s \),  
a demand function \( Q_i^d = \gamma_0 + u_i^d \), and a market equilibrium condition \( Q_i^s = Q_i^d \), where \( u_i^s \) and \( u_i^d \) are mutually independent i.i.d. random variables, both with a mean of 0.  

\begin{enumerate}
    \item Show that \( P_i \) and \( u_i^s \) are correlated.
    \begin{solution}
        \begin{align*}
            \Cov(P, u^s) &= \Cov(\frac{1}{\beta_1}(\gamma_0 + u^d - \beta_0 - u^s), u^s)\\
            &= \frac{1}{\beta_1}\Cov(u^d - u^s, u^s)\\
            &= -\frac{1}{\beta_1}\Var(u^s)\\
            &\neq 0
        \end{align*}
    \end{solution}
    \item Show that the OLS estimator of \( \beta_1 \) is inconsistent.
    \begin{solution}
        We know that 
        \[\hat\beta_1 = \beta_1 + \frac{\sum (P_i - \bar P)u_i^s}{\sum (P_i - \bar P)}\xrightarrow[\bbP]{}\beta_1 +\frac{\Cov(P, u^s)}{\Var(P)}\] yikes! We showed in (i) that the right term isn't zero.
    \end{solution}
    \item How would you estimate \( \beta_0, \beta_1 \), and \( \gamma_0 \)?
    \begin{solution}
        With instrument variables! Introduce a supply shifter.
    \end{solution}
\end{enumerate}
\item A researcher is interested in the effect of military service on human capital.  
He collects data from a random sample of 4000 workers aged 40 and runs the OLS regression \( Y_i = \beta_0 + \beta_1 X_i + u_i \), where \( Y_i \) is a worker’s annual earnings and \( X_i \) is a binary variable that is equal to 1 if the person served in the military and is equal to 0 otherwise.

\begin{enumerate}
    \item Explain why the OLS estimates are likely to be unreliable. (Hint: Which variables are omitted from the regression? Are they correlated with military service?)
    \begin{solution}
        There might be something like education thrown into the mix of $U$ that is omitted, and since someone with military experience might be less likely to attend higher education, they are definitely correlated. 
    \end{solution}
    \item During the Vietnam War, there was a draft in which priority for the draft was determined by a national lottery. (The days of the year were randomly reordered 1 through 365. Those with birth dates ordered first were drafted before those with birth dates ordered second, and so forth.)  
    Explain how the lottery might be used as an instrument to estimate the effect of military service on earnings. (For more about this issue, see Joshua D. Angrist (1990).)
    \begin{solution}
        The lottery, $Z,$ can be used because it directly affects $Z$ (relevance) and it is exogenous to $U.$ 
    \end{solution}
\end{enumerate}
\end{enumerate}

\newpage
\section*{Problem 2}
Suppose the first stage regression model:
\[
X = \pi_0 + \pi_1 Z_1 + \pi_2 Z_2 + \pi_3 W + V
\]
where \( X \) is an endogenous variable, \( Z_1 \) and \( Z_2 \) are instruments, and \( W \) is an exogenous control. The reported \( R^2 \) of this regression is 0.8 and the sample size is 100.

Also, consider the model:
\[
X = \pi_0 + \pi_3 W + V
\]
where the variables are defined as in the previous model, assuming \( \pi_1 \) and \( \pi_2 \) equal to zero. The reported \( R^2 \) is 0.79 and the sample size is 100.

\begin{enumerate}[(a)]
    \item Test at the 5\% significance level the joint hypothesis that \( \pi_1 = \pi_2 = 0 \). You can find the critical value using the F-table. If the desired degrees of freedom do not appear in the table, use the closest values.
    \begin{solution}
        Yes Sir! 
        \[F = \frac{\frac{1}{q}(R^2_{ur} - R^2_r)}{\frac{1}{n-K_{ur} - 1}(1 - R^2_{ur})} = \frac{\frac{1}{2}(0.01)}{\frac{1}{100-4 - 1}(0.2)} = 2.375\] this implies a critical value of about $3.07$
    \end{solution}
    \item Test whether the instruments are weak using the rule of thumb.
    \begin{solution}
        $2.375 < 10,$ thus WEAK
    \end{solution}
    \item Discuss why weak instruments are problematic.
    \begin{solution}
        Because $SE(\hat\beta)$ blows tf up and you can't do any good asymptotic results. 
    \end{solution}
\end{enumerate}

\newpage
\section*{Problem 3}
Describe the fundamental problem of causal inference.
\begin{solution}
    Ok, so imagine you're me and I'm me. There's two me! Sounds like a fundamental problem. I can just kill myself now. 
\end{solution}

\newpage
\section*{Problem 4}
Consider the following expressions:
\[
E[Y | D = 1] - E[Y | D = 0] = ATT + \left(E[Y_0 | D = 1] - E[Y_0 | D = 0]\right)
\]
\[
E[Y | D = 1] - E[Y | D = 0] = ATU + \left(E[Y_1 | D = 1] - E[Y_1 | D = 0]\right)
\]
\[
ATE = \Pr(D = 1) \cdot ATT + \Pr(D = 0) \cdot ATU
\]

Discuss the necessary conditions such that the naive estimator \( E[Y | D = 1] - E[Y | D = 0] \) coincides with the ATT, ATU, and ATE.
\begin{solution}
    In an experiment with randomization such that $y_0^i, y_1^i \perp D^i,$ we see that 
    \[SB_0 = \bbE[y^i_0 \mid D_i = 1] - \bbE[y^i_0 \mid D_i = 0] = \bbE[y_0^i] - \bbE[y_0^i] = 0\] and same for $SB_1.$

    Moreover, 
    \begin{align*}
    ATE &= \bbE[y_1^i - y_0^i]\\
    &= \bbE[\bbE[y_1^i - y_0^i \mid D^i]]\\
    &= p (ATT) + (1-p)(ATU)\\
    &= p(\theta - SB_0)  + (1-p)(\theta - SB_1)\\
    &= \theta
\end{align*}
Thus, in a randomized experiment, 
\boxed{ATT = ATU= ATE = \theta}
\end{solution}


\end{document}