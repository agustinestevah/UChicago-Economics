\documentclass[10pt, oneside]{article} 
\usepackage{amsmath, amsthm, amssymb, calrsfs, wasysym, verbatim, bbm, color, graphics, geometry, esint, float}
\usepackage{mdframed}



\geometry{tmargin=.75in, bmargin=.75in, lmargin=.75in, rmargin = .75in}  

\newcommand{\bbR}{\mathbb{R}}
\newcommand{\bbC}{\mathbb{C}}
\newcommand{\bbZ}{\mathbb{Z}}
\newcommand{\bbP}{\mathbb{P}}
\newcommand{\bbN}{\mathbb{N}}
\newcommand{\bbQ}{\mathbb{Q}}
\newcommand{\Cdot}{\boldsymbol{\cdot}}
\newcommand{\scA}{\mathscr{A}}
\newcommand{\curl}{\text{curl}}
\newcommand{\Ind}{\text{Ind}}
\newcommand{\Log}{\text{Log}}
\newcommand{\Vol}{\text{Vol}}
\newcommand{\Var}{\text{Var}}
\newcommand{\Cov}{\text{Cov}}
\newcommand{\Corr}{\text{Corr}}
\newcommand{\bbE}{\mathbb{E}}
\newcommand{\bbV}{\mathbb{V}}




\newcommand{\sm}{\setminus}

\theoremstyle{definition}
\newtheorem{exmp}{Example}[section]
\newtheorem{thm}{Theorem}
\newtheorem{defn}{Definition}
\newtheorem{prop}{Proposition}
\newtheorem{conv}{Convention}
\newtheorem{rem}{Remark}
\newtheorem{lem}{Lemma}
\newtheorem{cor}{Corollary}

\input{paolo-pset.tex}



\title{UChicago Econometrics Notes: 20510}
\author{Notes by Agustín Esteva, Lectures by Murilo Ramos}
\date{Academic Year 2024-2025}

\begin{document}

\maketitle
\tableofcontents

\vspace{.25in}


\newpage
\section{Lectures}

\subsection{Monday, June 16: Intro to Probability}
\begin{lem}
    (Jensen) Suppose $X$ is a random variable. Let $g: \bbR \to \bbR$ be convex. Then
    \[g(\bbE[X]) \leq \bbE[g(X)]\]
\end{lem}
\begin{proof}
    Since $g$ is convex, then for any $x,y \in \bbR,$ and for any $t\in (0,1)$
    \[g(tx + (1-t)y) \leq tg(x) + (1-t)g(y).\] Let $x = X$ and $y = \bbE[X],$ we see that 
    \[g(tX + (1-t)\bbE[X]) \leq tg(X) + (1-t)g(\bbE[X]).\] Taking expected value,

    \[\bbE[g(tX + (1 - t)\bbE[X]] \leq t\bbE[g(X)] + (1-t)g(\bbE[X])\]
\end{proof}

\begin{defn}
    Let $X$ be a random variable. We say that the \textbf{k\textit{th} moment} of $X$ is $\bbE[X^k].$ We say that the \textbf{k\textit{th} centered moment} of $X$ is $\bbE[(X - \bbE[X])^k].$ We say that the \textbf{k\textit{th} standardized moment} of $X$ is 
    \[\bbE\left[\left(\frac{X - \bbE[X]}{\sqrt{\bbV[X]}}\right)^k\right]\]
\end{defn}

\begin{defn}
    We say that the \textbf{skewness} of $X$ is the third standardized moment of $X.$ We say that the \textbf{kurtosis} of $X$ is the fourth standardized moment of $X.$
\end{defn}

\begin{lemma}
    Let $s \leq t.$ Let $X$ be a positive random variable. If $\bbE[X^t] < \infty,$ then $\bbE[X^s] < \infty.$
\end{lemma}
\begin{proof}
    We can split up $X$ into 
    \[X^t = \mathbbm{1}_{\{X^t \geq 1\}} + \mathbbm{1}_{\{X^t <1\}}.\] It should now be clear that $X^s < X^t + 1$ almost surely. Taking expectations we are done.
\end{proof}

\begin{defn}
    Let $X$ and $Y$ be random variables. We define the \textbf{covariance} of $X$ and $Y$ to be 
    \[\Cov(X,Y) = \bbE[(X - \bbE[X])(Y - \bbE[Y])]\]
\end{defn}



\begin{prop}
    Let $X,Y,Z$ be random variables. Let $a,b,c \in \bbR.$
    \begin{enumerate}
        \item \[\Cov(X,Y) = \Cov(Y,X)\]
        \item \[\Cov(X,Y) = \bbE[XY] - \bbE[X]\bbE[Y]\]
        \[\Cov(X,a) = 0\]
        \item Covariance is bilinear in terms of random variables.
        \item \[\Cov(a + bX, cY) = bc\Cov(X,Y)\]
    \end{enumerate}
\end{prop}

\begin{lemma}
    \[\Var(aX + bY) = a^2\Var(X) + b^2\Var(Y) + 2ab\Cov(X,Y)\]
\end{lemma}
\begin{defn}
    We define the \textbf{correlation} of $X$ and $Y$ to be 
    \[\Corr(X,Y) = \frac{\Cov(X,Y)}{\sqrt{\Var(X)}\sqrt{ \Var(Y)}}\]
\end{defn}
\begin{rem}
    If $\Corr(X,Y)= 0,$ we say that $X$ and $Y$ are uncorrelated, and infer that there is no linear association between $X$ and $Y$. 
\end{rem}

\begin{lemma}
    (C-S Lemma) 
    \[(x,y) \leq \|x\|\|y\|\]
\end{lemma}
\begin{rem}
    Using the $L^2$ norm, we see that 
    \[\bbE[XY] \leq \sqrt{\bbE[X^2]}\sqrt{\bbE[Y^2]}\]
\end{rem}
\begin{proof}
    We prove it for the case of expected value. Let $a\in \bbR.$ Then 
    \begin{align*}
        0 \leq \bbE[(X - aY)^2]= \bbE[X^2] - 2a\bbE[XY] + a^2 \bbE[Y^2]
    \end{align*}
    is a quadratic function with respect to $a.$ Optimizing, 
    \[0 = -2\bbE[XY] + 2a^* \bbE[Y^2]= 0 \implies a^* = \frac{\bbE[XY]}{\bbE[Y^2]}\] Plugging back into (1), 
    \[0 \leq \bbE[X^2] - 2\frac{\bbE[XY]^2}{\bbE[Y^2]} + \frac{\bbE[XY]^2}{\bbE[Y^2]} = \bbE[X^2] - \frac{\bbE[XY]^2}{\bbE[Y^2]}\]
\end{proof}
\begin{rem}
    We see that if $X = aY,$ then by (1), equality in C-S happens. This is an iff.
\end{rem}

\begin{thm}
    For any $X,Y$ random variables, 
    \[|\Corr(X,Y)| \leq 1\] with equality if and only if $Y = a + bX$ almost surely for some $a,b \in \bbR.$
\end{thm}
\begin{proof}
    It suffices to show that $0\leq (\Corr(X,Y))^2 \leq 1.$ But 
    \begin{align*}
        (\Corr(X,Y))^2 &= \frac{\Cov(X,Y)^2}{\Var(X)\Var(Y)}
    \end{align*}
    By definition, it suffices to show that
    \[\Cov(X,Y)^2 = \bbE[(X - \bbE[X])(Y - \bbE[Y])]^2 \leq \bbE[(X - \bbE[X])^2]\bbE[(Y - \bbE[Y])^2]= \Var(X)\Var(Y) \] and so we are done by C-S inequality. Equality comes from equality in C-S.
\end{proof}
\begin{defn}
    Recall that the \textbf{conditional probability} of $X$ given $Y$ is defined to be 
    \[f_{X \mid Y}(x \mid y) = \frac{f_{XY}(x,y)}{f_X(x)}\] The \textbf{conditional expectaion} of $X$ given $Y$ is defined to be 
    \[\bbE[X \mid Y = y] = \int_\bbR x f_{X \mid Y}(x \mid y) = \frac{\int_\bbR x f_{XY}(x,y)}{\int_\bbR  f_Y(y)}\]
\end{defn}

\begin{defn}
    Recall that the \textbf{mean squared error} of $\hat X$ is 
    \[\text{MSE}(\hat{X})= \bbE[(X- \hat X)^2]\]
\end{defn}

\begin{thm}
    $\bbE[Y \mid X]$ is the best predictor for $Y$ given $X$ in an MSE sense. That is, it is the best estimator in the sense that it minimizes the MSE. In other words, 
    \[\bbE[Y \mid X] = \min_{g(X)} \bbE[(Y - g(X))^2]\]
\end{thm}


\begin{thm}
\begin{center}
\fbox{
  \parbox{0.9\textwidth}{
   $\bbE[Y \mid X]$ is the best predictor for $Y$ given $X$ in an MSE sense: That is,  
    \[\bbE[Y \mid X] = \min_{g(X)} \bbE[(Y - g(X))^2]\]
  }
}
\end{center}    
\end{thm}

\begin{prop}
    Let $X,Y,Z$ be random variables, let $g,f$ be functions, and let $a,b \in \bbR$ Then the following hold:
    \begin{enumerate}
        \item $\bbE[g(X) + h(X)Y \mid X] = g(X) + g(X) \bbE[Y \mid X]$
        \item $\bbE[aY = bZ + c \mid X] = a \bbE[Y \mid X] + b \bbE[Z \mid X] + c$
        \item If $Y \leq Z$ almost surely, then $\bbE[Y \mid X] \leq \bbE[Z \mid X]$
        \item  \begin{center}
            
       \fbox{(Tower Law) $\bbE[X] = \bbE[\bbE[X \mid Y]]$} \end{center}
    \end{enumerate}
\end{prop}


\begin{defn}
    We say that $X$ is \textbf{mean independent} of $Y$ if $\bbE[X \mid Y] = c$ almost surely. 
\end{defn}
\begin{rem}
    Note that this notion is not symmetric.
\end{rem}
\begin{lemma}
    If $X$ is mean independent of $Y,$ then
    \begin{itemize}
        \item $\bbE[X \mid Y] = \bbE[X]$
        \item $\bbE[YX] = \bbE[Y]\bbE[X]$
        \item $\Corr(Y,X) = 0$
    \end{itemize}
\end{lemma}
\begin{proof}
Easy!
    \begin{itemize}
        \item $\bbE[X] = \bbE[\bbE[X \mid Y]] = \bbE[c] = c = \bbE[X \mid Y]$
        \item $\bbE[XY] = \bbE[\bbE[XY \mid Y]] = \bbE[Y \bbE[X \mid Y]] = c\bbE[Y] = \bbE[X]\bbE[Y]$
        \item Clear from ii and the fact that $\Cov(X,Y) = \bbE[XY]-\bbE[X]\bbE[Y]$
    \end{itemize}
\end{proof}
\begin{rem}
    Independence implies mean independence implies zero covariance. The converses are in general false. 
    \begin{itemize}
        \item Zero covariance but not mean independent: Let $X$ be a random variable taking values $\{-1, 0, 1\}$ with equal probability:
\begin{align*}
    \Paren{X = -1} &= 1/3 \\
    \Paren{X = 0} &= 1/3 \\
    \Paren{X = 1} &= 1/3
\end{align*}
Let $Y = X^2$.
\item Let $X$ be a random variable taking values $\{-1, 1\}$ with:
\begin{align*}
    \Paren{X = -1} &= 0.5 \\
    \Paren{X = 1} &= 0.5
\end{align*}
Let $Y$ be defined such that:
\begin{itemize}
    \item If $X=1$, $Y$ takes values $\{-1, 1\}$ with $\Paren{Y=-1|X=1}=0.5$ and $\Paren{Y=1|X=1}=0.5$.
    \item If $X=-1$, $Y$ takes values $\{-2, 2\}$ with $\Paren{Y=-2|X=-1}=0.5$ and $\Paren{Y=2|X=-1}=0.5$.
\end{itemize}
    \end{itemize}
\end{rem}

\begin{defn}
    We say that the \textbf{conditional variance} of $Y$ given $X$ is 
    \[\Var(Y \mid X) = \bbE[(Y - \bbE[Y \mid X])^2 \mid X]\]
\end{defn}

\begin{lemma}
    Let $X$ and $Y$ be r.v. and $g,h$ be functions. Then 
    \begin{enumerate}
        \item $\Var(Y \mid X) = \bbE[Y^2 \mid X] - \bbE[Y \mid X]^2$
        \item $\Var(g(X) + h(X) Y \mid X) = \Var(h(X) Y \mid X) = h^2(X)\Var(Y \mid X)$
        \item (Law of Total Variance)
        $\Var(Y) = \bbE[\Var(Y \mid X)] + \Var(\bbE[Y \mid X])$
    \end{enumerate}
\end{lemma}
\begin{proof}
First, 
    \begin{align*}
        \bbE[\Var(Y \mid X)] &= \bbE[\bbE[Y^2 \mid X] - \bbE[Y \mid X]^2]\\
        &= \bbE[\bbE[Y^2 \mid X]] - \bbE[\bbE[Y \mid X]^2]\\
        &= \bbE[Y^2] - \bbE[\bbE[Y \mid X]^2]
    \end{align*}
For the second term,
\begin{align*}
    \Var(\bbE[Y \mid X]) &= \bbE[\bbE[Y \mid X]^2] - \bbE[\bbE[Y \mid X]]^2\\
    &= \bbE[\bbE[Y \mid X]^2] - \bbE[Y]^2\\
\end{align*}
Combining we conclude.
\end{proof}

\newpage
\subsection*{Wednesday, June 18: Intro to Statistics}


We will assume that if we are sampling without replacement with simple random samples, then for a large population, we will treat it as i.i.d. samples. 
\begin{defn}
    Recall that an \textbf{estimator} is a function of the sample such that 
    \[\hat{\theta}_n = \hat{\theta}_n(X_1, \dots, X_n) \]
\end{defn}
\begin{rem}
    Note that an estimator is a random variable, as compared to parameters (e.g, means of populations or variances of populations), which are numbers. 

    The sample mean is a the most frequently used estimator. 

    Recall the analogy principle, where we use $\frac{1}{n}\sum \cdot$ to mim $\bbE[\cdot].$ For example, if $\theta = \Var(X) = \bbE[(X - \bbE(X))^2],$ then 
    \[\hat{\theta}_n = \frac{1}{n} \sum_{i=1}^n (X_i - \overline{X}_n)^2\] As another example, consider 
    \[\theta = \bbP\{X \leq x\} = \bbE[\mathbbm{1}_{X \leq x}]\] then 
    \[\hat{\theta}_n = \frac{1}{n} \sum_{i=1}^n \mathbbm{1}_{X_i \leq x}\]
\end{rem}

\begin{defn}
    Let $\hat{\theta}$ be an estimator for $\theta.$ We define the \textbf{bias} to be 
    \[\text{Bias}(\hat{\theta}) = \bbE[\hat{\theta}] - \theta\]
\end{defn}
\begin{exmp}
    The sample mean is unbiased: 
    \[\bbE[\overline{X}_n] = \bbE[\frac{1}{n}\sum X_i] = \frac{1}{n} \sum \bbE[X_i] = \bbE[X]\]
\end{exmp}
\begin{exmp}
    Consider $\theta = \Var (X)$ with $\hat{\theta} = \frac{1}{n}\sum (X_i - \overline{X}_n)^2.$ Then consider that by the previous example,
    \begin{align*}
        \hat{\theta_n} &= \frac{1}{n}\sum (X_i - \overline{X}_n)^2\\
        &= \frac{1}{n}\sum ((X_i - \bbE[X_i]) - (\overline{X}_n + \bbE[\overline{X}_n]))^2\\
        &= \frac{1}{n}\sum (X_i - \bbE[X_i])^2 - (\overline{X}_n + \bbE[\overline{X}_n])^2\\
        \bbE[\hat{\theta}_n] &= \bbE\left[\frac{1}{n}\sum (X_i - \bbE[X_i])^2 - (\overline{X}_n + \bbE[\overline{X}_n])^2\right]\\
        &= \frac{1}{n}\sum \Var(X_i) - \frac{1}{n}\Var(\overline{X}_n)\\
        &= \Var(X) - \frac{1}{n}\Var(X) = \frac{(n-1)}{n}\Var(X)
    \end{align*}
    Normalize $\frac{n}{n-1}$ to make it unbiased. 

    That is a stupid ass proof. Convince yourself of the following steps:
    \begin{align*}
        \bbE[\hat{\theta}_n] &= \bbE[\frac{1}{n} \sum (X_i - \bar X)^2]\\
        &= \frac{1}{n}\sum \bbE [X_i^2]- \frac{1}{n}\sum \bbE[(\bar X)^2]\\
        &= \bbE[X^2] - \bbE[(\bar X)^2]\\
        &= \Var (X) - \bbE[X]^2 - (\Var(\bar X) - \bbE[\bar X]^2)\\
        &= \Var (X) - \frac{1}{n}\Var(X)\\
        &= \frac{n-1}{n}\Var(X)
    \end{align*}
\end{exmp}

\newpage
\subsection{Final, June 20: Estimator Theory}

\begin{rem}
    Recall that linear combination of normal variables is normal, and thus if $X_1, \dots, X_n \sim N(\mu, \sigma^2),$ then 
    \[\overline{X}_n \sim N(\mu, \frac{\sigma^2}{n}) \implies \frac{\sqrt{n}(\overline{X} - \mu)}{\sigma}\sim N(0,1)\]
\end{rem}
\begin{defn}
    We say that an estimator $\hat{\theta}_n$ \textbf{consistent} if it converges in probability. That is, for any $\epsilon>0,$ 
    \[\lim_{n\to \infty}\bbP\{|\hat{\theta}_n - \theta| >\epsilon\}  = 0\] and we write 
    \[\hat{\theta}_n \xrightarrow[\bbP]{} \theta\]
\end{defn}

\begin{lemma}
    (Chebyshev). If $1\leq p < \infty,$ then for any $\lambda>0,$ we have that 
    \[\bbP\{|X| \geq  \lambda\} \leq \frac{\bbE[|X|^p]}{\lambda^p}\]
\end{lemma}
\begin{proof}
    Letting $A = \{\omega \mid |X| \geq \lambda\},$ we see that 
    \[\bbE[|X|^p] = \int_\Omega |X|^p d\bbP \geq \int_A |X|^p d\bbP \geq \lambda^p \bbP\{A\}\]
\end{proof}

\begin{thm}
\begin{center}
\fbox{
  \parbox{0.9\textwidth}{
    \textbf{Weak Law of Large Numbers.} 
    Let \( X_1, \dots, X_n \sim F \) be i.i.d. Suppose \( \mathbb{E}[X_1^2] < \infty \), then
    \[
    \overline{X}_n \xrightarrow{\bbP} \mathbb{E}[X_1].
    \]
  }
}
\end{center}
\end{thm}


\begin{proof}
    Note that 
    \[\Var(\overline{X}_n) = \bbE[(\overline{X}_n - \bbE[\overline{X}_n])^2] = \bbE[(\overline{X}_n - n\bbE[X_1])^2].\] Hence, 
    \begin{align*}
        \bbP\{|\overline{X}_n - \bbE[X_1]|>\epsilon\} &= \bbP\{(\overline{X}_n - \bbE[X_1])^2 >\epsilon^2\}\\
         &\leq \frac{\Var(\overline{X}_n)}{\epsilon^2}\\
         &= \frac{\Var(X_1)}{n\epsilon^2} \\
         &\to 0
    \end{align*}
    where we use the fact that $\bbE[X_1^2] < \infty$ to say that $\Var(X_1) < \infty.$
\end{proof}



\begin{prop}
    Let $X_1, \dots, X_n \sim F$ be i.i.d. Then $\overline{X}_n$ is consistent.
\end{prop}
\begin{proof}
    As $n\to \infty,$ we know by the law of large numbers that $\overline{X}_n \to \mu$ in probability, and so we are done.
\end{proof}

\begin{thm}
\begin{center}
\fbox{
  \parbox{0.9\textwidth}{
    \textbf{Continuous Mapping Theorem.} \begin{enumerate}
        \item \text{Suppose } $X_n \to x$ in probability and $g$ is continuous. Then 
        \[g(X_n) \xrightarrow[\bbP]{} g(x)\] 
        \item Suppose $X_n \to X$ in distribution and $g$ is continuous. Then 
        \[g(X_n) \xrightarrow[\cal D]{} g(X)\]
    \end{enumerate}}}
\end{center}
\end{thm}

\begin{prop}
        Let $X_1, \dots, X_n \sim F$ be i.i.d. Then 
        \[\hat{\sigma}_x^2 = \frac{1}{n}\sum (X_i - \overline{X})^2\]is consistent.
\end{prop}
\begin{proof}
Note that 
\[\hat{\sigma}_x^2 = \frac{1}{n}\sum(X_i - \overline{X}_n)^2 = \left[\frac{1}{n}\sum X_i^2\right] - (\overline{X}_n)^2 \to \bbE[X^2] - \bbE[X]^2\] by the weak law of large numbers and the continuous mapping theorem using $g(w,x) = w- z^2$ and so we are done. To see the big step, we open up the parenthesis:
\begin{align*}
    \hat{\sigma}_x^2 &= \frac{1}{n}\sum(X_i - \overline{X}_n)^2 \\
    &= \frac{1}{n}\sum X_i^2 - 2X_i\overline{X} + \overline{X}^2\\
    &= \frac{1}{n} X_i^2 - \frac{1}{n}2\overline{X}\sum X_i + \overline{X}^2\\
    &= \frac{1}{n}X_i^2 +\overline{X}^2
\end{align*}
\end{proof}

\begin{exmp}
    Let $(X_1, Y_1), \dots \sim (X,Y)$ be i.i.d with $X,Y \in L^2.$ Let $\theta = \bbE[(X - \mu_X)(Y - \mu_Y)].$ By the anaology principle, 
    \[\hat{\theta}_n = \frac{1}{n}\sum (X_i - \overline{X})(Y_i - \overline{Y})\] Letting $g(w,z,t) = w - zt$ and noting that 
    \[\hat{\theta}_n = \frac{1}{n}\sum X_iY_i - \overline{X}\overline{Y},\] we can use the CMT and the WLLN to show that $\hat{\theta}_n$ is consistent.
\end{exmp}

\begin{defn}
    We say that $X_n$ \textbf{converges in distribution} to $X$ if 
    $F_{X_n} \to F_X(x)$
\end{defn}

\begin{thm}
\begin{center}
\fbox{
  \parbox{0.9\textwidth}{
    \textbf{Central Limit Theorem.}     Let $X_1, \dots, X_n \sim F$ be i.i.d. with mean $\mu$ and $\bbE[X^2] < \infty$ and variance $\sigma^2.$ Then 
    \[\frac{S_n}{\sqrt{n}}\to N(\mu, \sigma^2)\] in distribution
    
    }
    }
\end{center}
\end{thm}


In other words, we have that for large $n,$ 
\[\frac{\sqrt{n}(\bar X - \mu)}{\sigma}\sim N(0,1)\]


\begin{lem}
\begin{center}
\fbox{
  \parbox{0.9\textwidth}{
    \textbf{Slutsky}          Suppose $X_n \to X$ in distribution and $Y_n\to y$ in probability. Then 
    \begin{enumerate}
        \item $X_nY_n \to Xy$ in distribution
        \item $X_n + Y_n \to X + y$ in distribution
        \item $\frac{X_n}{Y_n} \to \frac{X}{y}$ if $y\neq 0$
        \item If $g$ is continuous, then $g(X_n, Y_n)\to g(X, y)$
    \end{enumerate}}}
\end{center}
\end{lem}

\begin{exmp}
    Suppose $X_1, \dots, X_n \sim X$ i.i.d. with $\bbE[X^2] < \infty$ and $\sigma_X^2 >0.$ Recall that the CLT implies that 
    \[\frac{\sqrt{n}(\overline{X}_n - \mu_X)}{\sigma_X} \xrightarrow[d]{} N(0,1).\] In general, we don't observe $\sigma_X^2,$ so we use an estimate $\hat{\sigma}_X^2 \xrightarrow[\bbP]{}\sigma_X^2$ and thus by the CMT 
    \[\hat{\sigma}_X \to \sigma_X.\] Hence, a more feasible statistic for hypothesis tests 
    is 
    \[\frac{\sqrt{n}(\overline{X}_n - \mu_X)}{\hat{\sigma}_X} = \left(\frac{\sqrt{n}(\overline{X}_n - \mu_X)}{\sigma_X}\right)\frac{\sigma_X }{\hat{\sigma}_X} \to N(0,1)\] by Slutsky. 
\end{exmp}

\begin{rem}
\begin{center}
\fbox{
  \parbox{0.9\textwidth}{
    \textbf{Hypothesis Testing}              \begin{enumerate}
        \item (\textit{Step 1}) State $H_0$ and $H_a.$
        \item (\textit{Step 2}) Test statistic and call it 
        \[T_n = g(X_1, \dots, X_n)\] a function of the data.
        \begin{itemize}
            \item $Z$ score could be 
            \[Z = \frac{\sqrt{n}(\overline{X}_n - \mu_X)}{\hat{\sigma}_X}\]
        \end{itemize}
        \item (\textit{Step 3}) Outline rejection region $R$ and critical values. I.e, $\alpha  =0.05$.
        \item (\textit{Step 4}) Conclude (Reject or fail to reject $H_0$)
    \end{enumerate}
    
    }}
\end{center}
\end{rem}


\begin{defn}
    We say that a \textbf{Type I Error} is when the null hypothesis is falsely rejected ($H_0$ is true but it is rejected). We say that a \textbf{Type II Error} is when the failed to be failed to be rejected ($H_0$ is false but it was failed to be rejected)
\end{defn}

\begin{rem}
    The convention is to choose some $\alpha\in \bbR$ such that 
    \[\bbP\{\text{Type I error}\} = \bbP\{T_n \in R \mid H_0\} = \alpha.\] We call $\alpha$ our significance level.
\end{rem}

\begin{exmp}
    (Two sided) Suppose $0 < \Var(X) < \infty$ and $H_0: \bbE[X] =  \mu_0$ and $H_a: \bbE[X] \neq \mu_0$. We let 
    \[T_n = \frac{\sqrt{n}(\overline{X}_n - \mu_0 )}{\hat{\sigma}_X} \xrightarrow[d]{}N(0,1)\] where the convergence happens under the null. We set $\alpha = 0.05,$ and thus 
    \[\bbP\{\left|\frac{\sqrt{n}(\overline{X}_n - \mu_0 )}{\hat{\sigma}_X}\right| \geq c \mid H_0\} = \alpha = 2\left(1 - \Phi(c)\right)  \] by the symmetric of the normal distribution. Solving, 
    \[c = \Phi^{-1}(1 - \frac{\alpha}{2})\]
\end{exmp}

\begin{defn}
    We define the \textbf{$p-$value} to be the smallest $\alpha$ for which we reject $H_0.$
\end{defn}
\begin{exmp}
    $H_0: \bbE[X] \geq 10,$ $H_a: \bbE[X] < 10.$ Found $T_n = -1.5.$ Then the $p-$value is 
    \[\bbP\{Z \leq -1.5\} = p.\] We reject if $p<\alpha$ Suppose now $H_0: \bbE[X] = 10$ and $H_a: \bbE[X]\neq 10.$ Then 
    \[2\bbP\{Z \leq -1.5\} = p.\] More generally, we saw in a $2-$sided test that $c = \Phi^{-1}(1 - \frac{\alpha}{2})$ and we reject if $|T_n| >c,$ and thus reject if
    \[\alpha > 2 (1 - \Phi(|T_n|)) = 2\bbP\{|T_n|\} = p\]
\end{exmp}


\newpage
\subsection{Monday, June 23: Introducing the SLR}


\begin{defn}
    (SLR Model) We say that $y$ is a simple linear regression if 
    \[Y_i = \beta_0 + \beta_1X_i + \sigma U_i,\] where we call $\beta_0$ to be our intercept parameter, $\beta_1$ to be our slope parameter, and $U_i$ is the error term. 
\end{defn}
\begin{rem}
There are three ways to interpret the regressors, and an analysis of these interpretations will yield some insight in why we assume some things:
\begin{enumerate}
    \item (Linear Conditional Expectation)     Suppose that for some $Y$ and $X$ r.v,
    \[\bbE[Y \mid X] = \beta_0 + \beta_1 X.\] We can define 
    \[U = Y - \bbE[Y \mid X].\] Hence, by definition, 
    \[Y = \bbE[Y \mid X] + U = \beta_0 + \beta_1 X + U\] Thus, we see that 
    \[\bbE[U \mid X] = \bbE[Y - \bbE[Y \mid X] \mid X] = 0,\] implying that $U$ is mean independent of $X$ and thus 
    \[\Cov(U,X) = 0\] and moreover, 
    \[\bbE[U] = \bbE[\bbE[U \mid X]] = 0\]
    \item     (Best Linear Predictor (BLP)). Suppose $Y = \beta_0 + \beta_1 X + U = \text{BLP}(Y \mid X) + U$
    \begin{itemize}
        \item Suppose we want to find the best linear predictor for $Y$ as a function of $X$ in the sense that in minimizes $\text{MSE}.$ That is, 
        \[\text{BLP}(Y \mid X) = \min_{(b_0, b_1) \in \bbR^2} \bbE[(Y - b_0 - b_1 X)^2]\] Taking FOC, we find that 
        \[\text{BLP}_1 = (\beta_0, \beta_1)\]
        \item Suppose we want to find the best linear predictor for $\bbE[Y \mid X].$  We want to find 
        \[\text{BLP}_2 = \min_{(b_0, b_1)\in \bbR^2} \bbE[ (\bbE[Y \mid X] - b_0 - b_1 X)^2]\]
    \end{itemize}
    We claim that $\text{BLP}_1 = \text{BLP}_2.$ 
    \begin{proof}
Computing, 
\begin{align*}
    \bbE[(Y - b_0 - b_1 X)^2] &= \bbE[Z^2]\\
    &= \bbE[((Y - \bbE[Y \mid X) + (\bbE[Y \mid X] - b_0 + b_1 X))^2]\\
    &= \bbE[(Y - \bbE[Y \mid X])^2] + \bbE[(\bbE[Y \mid X] - b_0 + b_1 X))^2]
\end{align*}
where we can use orthogonality since 
\begin{align*}
   & \bbE[(Y - \bbE[Y \mid X) (\bbE[Y \mid X] - b_0 + b_1 X)] =\\
    &= \bbE[Y \bbE[Y \mid X]] - \bbE[\bbE[Y \mid X]^2] - b_0\bbE[V] - b_1 \bbE[VX]\\
    &= 0 - 0 - b_1\bbE[\bbE[VX \mid X]] = 0
\end{align*}
Hence, we minimize by taking derivatives and the first term drops out, yielding our result.
    \end{proof}
So we minimized $\bbE[(Y - b_0 - b_1 X)^2] = \bbE[Z^2]$ and to do this explicitly, 
\[\frac{\partial f}{\partial b_0} = -2\bbE[Y - b_0 - b_1 X] \implies \bbE[U] = 0 \qquad \frac{\partial f}{\partial b_1} = -2\bbE[X(Y - b_0 - b_1X)] \implies \bbE[UX] = 0\] Thus, 
\[\bbE[U] = 0\] and 
\[\Cov(U,X) = \bbE[UX] - \bbE[U]\bbE[X] = 0\] and thus the $\text{BLP}$ satisfies the conditions in the previous example.
\item (Causal Interpretation) Suppose our BLP is of the form 
    \[Y = \beta_0 + \beta_1 X + U\] where \textit{we assume} that $\bbE[U] = 0$ and $\Cov(X,U) = \bbE[XU] = 0.$ 
    Then the causal model is of the form
    \[Y = \gamma_0 + \gamma_1 X + V,\] where $V$ is called the causal error (alive!) and can be explained by everything that causes $Y$ which is not encoded in $X$, implying that $\Cov(X,V)\neq 0.$ We define $\gamma_1$ to be 
    \[\frac{\partial Y}{\partial X} \bigg |_{\text{keeping everything constant}} = \gamma_1.\] 

    It is easy to estimate $\beta_0, \beta_1,$ but it is much harder to compute $\gamma_0, \gamma_1.$
\end{enumerate}

\end{rem}


\newpage
\subsection{Wednesday, June 25: SLR Coefficient Theory}
\begin{lem}
    The following equalities hold:
    \begin{enumerate}
        \item \[\sum (X_i - \bar X)(Y_i - \bar Y) = \sum (X_i - \bar X)Y_i\]
        \item \[\sum (X_i - \bar X)^2 = \sum (X_i - \bar X)X_i\]
        \item \[\sum (X_i - \bar X) = 0\]
    \end{enumerate}
\end{lem}

\begin{rem}
\begin{center}
\fbox{
  \parbox{0.9\textwidth}{
    \textbf{SLR Setup in the population}              Let $X,Y,U$ be r.v. such that 
    \[Y = \beta_0 + \beta_1 X + U\] and assume 
    \begin{enumerate}
        \item {$\bbE[U] = 0$}
        \item {$\bbE[XU] = \Cov(X,U) = 0$}
        \item ${0 < \Var(X) < \infty}$
        \item {$(X_1, Y_1), \dots, (X_n, Y_n)\sim (X,Y)$ i.i.d.}
    \end{enumerate}
    
    }}
\end{center}

    From (a), we have that 
    \begin{align*}
        0 &= \bbE[U]\\
        &= \bbE[Y - \beta_0 - \beta_1 X]\\
        &= \bbE[Y] - \beta_0 - \beta_1\bbE[X]
    \end{align*}
    From (b), we have that 
    \begin{align*}
        0 &=\bbE[X(Y - \beta_0 - \beta_1 X)]\\
        &= \bbE[X(Y - \bbE[Y]) - \beta_1(X - \bbE[X])]
    \end{align*} and hence
    \[\bbE[(X - \bbE[X])(Y - \bbE(Y))]=\bbE[X(Y - \bbE[Y])] = \beta_1 \bbE[X(X - \bbE[X])] = \beta_1 \bbE[(X - \bbE[X])(X - \bbE(X))]\]
    Thus, 
    \begin{align}
        \boxed{\beta_1 = \frac{\Cov(X,Y)}{\Var(X)}}
    \end{align}
    \begin{align}
                \boxed{\beta_0 = \bbE[Y] - \frac{\Cov(X,Y)}{\Var(X)}\bbE[X]}
                \end{align}

\end{rem}

\begin{exmp}
        Consider the special case when $X$ is Bernoulli so that $X_1 \sim \text{Bernoulli}(p).$ Note that to compute $\Cov(X,Y) = \bbE[XY] - \bbE[X]\bbE[Y],$ we compute
    \[\bbE[XY] = \bbE[ X\bbE[Y \mid X]]= p \bbE[Y \mid X = 1]\]
    \[\bbE[X]\bbE[Y] = p\bbE[Y] = p\bbE[\bbE[Y \mid X]] = p\left(p\bbE[Y\mid X = 1] + (1-p)\bbE[Y \mid X  =0]\right)\] Thus, we have that 
    \begin{align*}
        \Cov(X,Y) &= p \bbE[Y \mid X = 1] - p^2 \bbE[Y \mid X =1] - p(1-p)\bbE[Y\mid X = 0]\\
        &= p(1-p)\left(\bbE[Y \mid X = 1] - \bbE[Y \mid X = 0]\right)
    \end{align*}
    Hence, 
    \[\beta_1 = \bbE[Y \mid X =1] - \bbE[Y \mid X = 0]\] Computing, we see that 
    \begin{align*}
        \beta_0 &= \bbE[Y] - \beta_1\bbE[X]\\
        &= \bbE[\bbE[Y \mid X]] - \beta_1 p\\
        &= p(\bbE[Y \mid X = 1]) + (1-p)\bbE[Y\mid X = 0] - \left(\bbE[Y \mid X =1] - \bbE[Y \mid X = 0]\right)p\\
        &= \bbE[Y \mid X =0]
    \end{align*}
    Tautological, we have that 
    \[\bbE[Y \mid X] = \bbE[Y \mid X = 0] + \left(\bbE[Y \mid X = 1] - \bbE[Y \mid X = 0]\right)\] implying that by definition, 
    \[\bbE[Y \mid X] = \beta_0 + \beta_1 X.\] 
    \boxed{$Thus, if $X$ is Bernoulli, then$ $\bbE[Y \mid X]$ is linear in $X$ and thus mean independent}.
\end{exmp}

\begin{prop}
\begin{center}
\fbox{
  \parbox{0.9\textwidth}{
    \textbf{SLR Setup in the Sample}              Let $X,Y,U$ be r.v. such that 
    \[Y = \beta_0 + \beta_1 X + U\] and assume 
    \begin{enumerate}
        \item {$\bbE[U] = 0$}
        \item {$\bbE[XU] = \Cov(X,U) = 0$}
        \item ${0 < \Var(X) < \infty}$
        \item {$(X_1, Y_1), \dots, (X_n, Y_n)\sim (X,Y)$ i.i.d.}
    \end{enumerate}
    Then 
    \begin{align}
        \hat \beta_1 &= \frac{\hat\sigma_{XY}}{\hat\sigma_X} = \frac{\sum (X_i - \bar X)(Y_i - \bar Y)}{\sum (X_i - \bar X)^2}
    \end{align}
    \begin{align}
        \hat\beta_2 &= \bar Y - \hat\beta_1 \bar X
    \end{align}
    }}
\end{center}

\end{prop}


\begin{proof}
This is clear using the analogy principle on (1) and (2). 

    For another derivation, recall that 
    \[\bbE[Y - \beta_0 - \beta_1 X] = 0 \qquad \bbE[(Y - \beta_0 - \beta_1 X)X]= 0\] are the first order conditions for $\min_{(b_0, b_1)} \bbE[(Y - b_0 - b_1 X)^2].$ Within the sample, See full derivation in PSET
\end{proof}


\begin{defn}
    Consider a sample regression model such that 
    \[Y_i = \hat{\beta}_0 + \hat{\beta}_1 X_i + \hat{U}_i.\] We call $\hat{U}_i$ the \textbf{residual}, and note that $\hat{\beta}_i$ are both random variables. We define the \textbf{residual} to be 
    \[\hat{U}_i = Y_i - \hat{Y}_i,\] where $\hat{Y}$ is the \textbf{fitted value} such that 
    \[\hat{Y}_i = \hat{\beta}_0 + \hat{\beta}_1 X_i\]
\end{defn}


\begin{rem}
    Recall conditions (a) and (b) in the basic setup. We showed in the above proof the sample equivalents of them for the first order conditions. That is, 
    \begin{align}
    \boxed{\frac{1}{n}\sum \hat{U_i} = 0 }
    \end{align}
    \begin{align}
    \boxed{\frac{1}{n}\sum X_i\hat U_i = 0}
    \end{align}
    Notice that these hold \underline{always} in the OLS, since they are major assumptions. These should not hold in general in causal models.
\end{rem}


\begin{exmp}
    Suppose $Y_i = \hat{\beta}_0 + \hat{\beta}_1 X_i + \hat{U}_i$ where $X_i$ is Bernoulli. Calling $n_0$ the number of times $X_i$ fails and $n_1$ the number of successes, then $n = n_0 + n_1$ is the number in the sample. We call 
    \[\bar Y_0 = \frac{\sum_{i=1}^n Y_i(1 - X_i)}{\sum_{i=1}^n (1 - X_i)} = \frac{1}{n_0} \sum_{i:X_i = 0} Y_i\]
    \[\bar Y_1 = \frac{\sum_{i=1}^n Y_i X_i}{\sum_{i=1}^n X_i} = \frac{\sum_{i : X_i = 1}Y_i}{n_1}\] Thus, we find that (see PSET)
    \[\hat{\beta}_0 = \bar Y_0 \qquad \hat{\beta}_1 = \bar Y_1 - \bar Y_0\]
\end{exmp}

\begin{defn}
    We say that $R^2$ is the \textbf{measure of fit} if it is 
    \[R^2 = \frac{\text{ESS}}{\text{TSS}} = 1 - \frac{\text{SST}}{\text{TSS}}\] where 
    \[\text{Total Sum of Squares (TSS)} = \sum_{i=1}^n (y_i - \bar y)^2\]
    \[\text{Explained Sum of Squares (ESS)} = \sum_{i=1}^n (\hat y_i - \bar y)^2\]
    \[\text{Sum of Squared Residuals (SSR)} = \sum_{i=1}^n \hat{u}_i^2\]
\end{defn}
\begin{prop}
The following hold, 
\begin{enumerate}
    \item \[\text{TSS} = \text{ESS} + \text{SSR}\]
    \item \[R^2 =  1 - \frac{\text{SST}}{\text{TSS}}\]
    \item $R^2 \in [0,1].$
\end{enumerate}
\end{prop}
\begin{proof}
    (a) We compute from the RHS,
    \begin{align*}
        \sum (\hat{y}_i - \bar y)^2  + \sum \hat u_1^2 &= \sum (\hat{y}_i - \bar y)^2  + \sum (y_i- \hat y_i)^2\\
        &= \sum (\hat{y}_i - \bar y - \hat y_i + y_i)^2   - 2\sum (\hat y_i - \bar y)(y_i - \hat y_i)\\
        &= \sum (y_i - \bar y_i)^2 - 2\sum (\hat y_i - \bar y)\hat u_i\\
        &= \sum (y_i - \bar y_i)^2 - 2(\sum \hat y_i u_i - \bar y\sum\hat u_i)\\
        &= \sum (y_i - \bar y_i)^2 - 2(\sum (\beta_0 + \beta_1 x_i) u_i - \bar y\sum\hat u_i)\\
        &= \sum (y_i - \bar y_i)^2\\
        &= \text{TSS}
    \end{align*}
    Where we use Remark 12

    (b) Dividing by TSS in (a), we see that 
    \[1 = R^2 + \frac{\text{SSR}}{\text{TSS}}\]

    (c) From (b), it suffices to see that TSS $\geq$ SSR, but this follows directly from (a)
\end{proof}




\begin{rem}
    Suppose $R^2  = 0,$ then $\text{ESS} = 0$ and $\text{SSR}  = \text{TSS}.$ That is, $\hat{Y}_i = \bar Y.$ Terrible model!!\\

    Suppose $R^2 = 1,$ then $\text{ESS}  = \text{TSS}.$ and $\text{SSR}  =0$ and thus $\hat{u}_i = 0$ and $\hat{y}_i = y_i.$ Goated model.\\

    \underline{$R^2$ does NOT IMPLY CAUSATION.}
\end{rem}

\begin{prop}
\begin{center}
\fbox{
  \parbox{0.9\textwidth}{
    \textbf{(Properties of $\hat\beta$)}              Let $X,Y,U$ be r.v. such that 
    \[Y = \beta_0 + \beta_1 X + U\] and assume 
    \begin{enumerate}
        \item {$\bbE[U] = 0$}
        \item {$\bbE[XU] = \Cov(X,U) = 0$}
        \item ${0 < \Var(X) < \infty}$
        \item {$(X_1, Y_1), \dots, (X_n, Y_n)\sim (X,Y)$ i.i.d.}
    \end{enumerate}
    Then the following hold
    \begin{enumerate}
        \item If $\bbE[U \mid X] = 0$ (alternatively, we have shown that this condition is equivalent to $X$ being binary or to $\bbE[Y \mid X]$ being linear in $X$), then 
        \[\bbE[\hat{\beta_0}] = \beta_0 \qquad \bbE[\hat\beta_1] = \beta_1\]
        \item If $\bbE[X^2] < \infty$ and $\bbE[Y^2] < \infty,$ then 
        \[\hat\beta_0 \xrightarrow[\bbP]{}\beta_0 \qquad \hat\beta_1 \xrightarrow[\bbP]{}\beta_1\]
        \item If $\bbE[X^4] < \infty$ and $\bbE[Y^4] < \infty,$ then 
        \[\sqrt{n}(\hat\beta_1 - \beta_1) \xrightarrow[\cal D]{} N(0, \sigma_1^2)\]
        
    \end{enumerate}
    }}
\end{center}
\end{prop}

\begin{proof}
    \begin{enumerate}
        \item We will first show all these results for $\hat\beta_1.$ Note that 
        \begin{align*}
            \hat\beta_1 &= \frac{\hat\sigma_{XY}}{\hat\sigma_X}\\
            &= \frac{\sum(X_i - \bar X)Y_i}{\sum (X_i - \bar X)^2}\\
            &= \frac{\sum (X_i - \bar X)(\beta_0 + \beta_1 X_i + U_i)}{\sum(X_i - \bar X)^2}\\
            &= \beta_1 + \frac{\sum (X_i - \bar X)U_i}{\hat \sigma_X^2}
        \end{align*}
        We note that 
        \begin{align}
            \hat \beta_1 = \beta_1 + \frac{\sum (X_i - \bar X)U_i}{\sum (X_i - \bar X)^2}
        \end{align}
        Taking $\bbE[\hat\beta_1 \mid X_1, \dots, X_n]$ in (7) and using the assumption that $\bbE[U \mid X] = 0$ and then LIE we conclude. Moreover, 
        \begin{align*}
            \bbE[\hat\beta_0] &= \bbE[\bar Y - \hat\beta_1 \bar X]\\
            &= \bbE[Y] - \beta_1 \bbE[X]\\
            &= \beta_0
        \end{align*}
        \item Under the condition of the second moments, we have showed (Proposition 4 and Example 1.3) that the estimators for covariance and variance are consistent. Thus, using (d) and the CMT for $g(s,t) = \frac{s}{t},$ we see that 
        \[\hat\beta_1 = g(\hat\sigma_{XY}, \hat\sigma_X) \xrightarrow[\bbP]{}g(\sigma_{XY}, \sigma_X) = \beta_1\] Moreover, we use the CMT again with $g(w,s,t) = w - st$ to show that
        \[\hat\beta_0 = g(\bar Y, \hat\beta_1, \bar X) \xrightarrow[\bbP]{} g(\bbE[Y], \beta_1, \bbE[X]) = \beta_0\]
        \item From (7), we see that 
        \begin{align*}
            \sqrt{n}(\hat\beta_1 - \beta_1) &= \frac{\frac{1}{\sqrt{n}}\sum (X_i - \bar X)U_i}{\frac{1}{n}\sum (X_i - \bar X)^2}\\
            &\xrightarrow[\bbP]{} \frac{1}{\sigma_X^2} \left[\frac{1}{\sqrt{n}}\sum (X_i - \bar X)U_i\right]\\
            &=  \frac{1}{\sigma_X^2} \left[\frac{1}{\sqrt{n}}\sum (X_i - \bbE[X] + \bbE[X] - \bar X)U_i\right]\\
            &= \frac{1}{\sigma_X^2}\left[\left(\frac{1}{\sqrt{n}}\sum (X_i - \bbE[X])U_i \right) + \frac{1}{\sqrt{n}} \sum (\bbE[X] - \bar X)U_i\right]\\
            &\xrightarrow[\cal D]{} \frac{1}{\sigma_X^2} N(0, \Var((X - \bbE[X])U))\\
            &= N(0, \frac{1}{(\sigma_X^2)^2}\Var((X - \bbE[X])U))
        \end{align*}
        where we use Slutsky's Lemma for the last convergence, noting that we use the CLT for the first term and the convergence of $\bar X\to \mu_X$ in probability for the second.
    \end{enumerate}
\end{proof}




\newpage
\subsection{Friday, June 27: OVB, Homo/heteroskedasticity, and Inference}
\begin{exmp}
    (Omitted Variable Bias) Causal Model:
\[\text{wages}_i = \gamma_0 + \gamma_1\text{educ}_i + V_i\] where $V_i$ is alive and $\Cov(V_i, X_i) \neq 0.$ 

BLP Model:
\[\text{wages}_i = \beta_0 + \beta_1 \text{educ}_i + U_i\] such that $\Cov(X_i, U_i) = 0.$ Thus, $\gamma_1 \neq \beta_1$ and $\gamma_0 \neq \beta_0.$ 

Does $\beta_1$ over/underestimate $\gamma_1$? Compare to (7), and we see that
\begin{align*}
    \hat\beta_1 &\xrightarrow[\bbP]{} \beta_1\\
    &= \frac{\sigma_{XY}}{\sigma^2_X}\\
    &= \frac{\Cov(X, \gamma_0  + \gamma_1X + V)}{\Var(x)}\\
    &= \frac{\Cov(X, \gamma_0)+ \gamma_1\Cov(X, X) + \Cov(X,V)}{\Var(X)}\\
    &= \gamma_1 + \frac{\Cov(X,V)}{\Var(X)}\\
\end{align*}

\begin{center}
\fbox{
  \parbox{0.9\textwidth}{
    \textbf{(OVB)}              If $\Cov(X,V) >0,$ then $\hat{\beta}_1$ overestimates $\gamma_1.$ If $\Cov(X,V) <0,$ then it underestimates. If $\Cov(X,V) = 0,$ then $\hat{\beta}_1 \to \gamma_1$ in probability.
    }}
\end{center}
\end{exmp}



\begin{rem} In samples, it is often unfeasable to know what $\sigma_1^2$ is. Thus, we often don't use (c) in proposition 7. We estimate using 
\[\hat{\sigma}_1^2 = A\hat\Var(\hat\beta_1)  = \frac{\frac{1}{n}\sum (X_i - \bar X) \hat U_i^2}{(\hat\sigma_X^2)^2}\] and we know that $\hat\sigma_1^2\xrightarrow[\bbP]{} \sigma_1^2.$
\end{rem}


\begin{defn}
    If $U$ is \textbf{homoskedastic}, then $\bbE[U \mid X] = 0$ and $\Var(U \mid X) = \Var(U)$. If $U$ is heteroskedastic, then $\bbE[U  \mid X] = h(X)$
\end{defn}

\begin{prop}
    Suppose $U$ is homoskedastic, then $\sigma_1^2 = \frac{\Var(U)}{\Var(X)}$
\end{prop}

\begin{proof}
    We have that 
\begin{align*}
    \Var((X - \bbE[X])U) &= \bbE[(X - \bbE[X])^2 U^2]\\
    &= \bbE[(X - \bbE[X])^2 \bbE[U^2 \mid X]]\\
    &= \bbE[(X - \bbE[X])^2 \left(\bbE[U^2 \mid X] - \bbE[U \mid X]^2\right)]\\
    &= \bbE[(X - \bbE[X])^2 \Var(U \mid X)]\\
    &= \bbE[(X - \bbE[X])^2 \Var(U)]\\
    &= \Var(U)\Var(X)
\end{align*}
\end{proof}
\begin{exmp}
    (Hetero or Homo?) Suppose $Y$ is Bernoulli($p$) and $\bbE[U \mid X] = 0$ and $Y = \beta_0 +\beta_1 X + U.$ Recall that we have showed that  $\bbE[Y \mid X] = \beta_0 + \beta_1 X.$ First, Note that $Y^2 = Y.$ Next, note that 
    \[\Var(Y \mid X) = \bbE[Y^2 \mid X] - \bbE[Y \mid X]^2 = \bbE[Y \mid X] - \bbE[Y \mid X]^2 = \bbE[Y \mid X][1 - \bbE[Y \mid X]]\] Thus, 
    \[\Var(Y \mid X) = (\beta_0 + \beta_1 X)(1 - \beta_0 - \beta_1 X)\] But we also have that 
    \[\Var(Y \mid X) = \Var(\beta_0 + \beta_1 X +U  \mid X) = \Var(U \mid X)\] which depends on $X,$ and so the error term $U$ is never homoskedastic. 
\end{exmp}

\begin{rem}
(Hypothesis Testing)
    \begin{enumerate}
        \item $H_0: \beta_1 = a$ and $H_1: \beta_1 \neq a$
        \item $T_n = \frac{\hat\beta_1 - \beta_1^{H_0}}{\text{SE}(\hat\beta_1)}$
        \item Same as before
    \end{enumerate}
\end{rem}

\begin{exmp}
    (Hypothesis Test) Test whether $\beta_1 = 1$ or $\beta_1 \neq 1$ at $\alpha 0.05$ We know that 
    \[T = \frac{\hat\beta_1 - \beta_1}{\hat\sigma_1} = \frac{0.6350  -1}{0.0214} = -17.05\] If sample is larger, then $t_\alpha = 1.96$ and we definitely reject.
\end{exmp}

\begin{rem}
(Log Level Regression)
    Recall the Maclaurin expansions:
    \[e^x = 1 + x + \frac{1}{2}x^2 + \cdots \approx 1 + x, \qquad x<<1\]
    \[\log(1+x) = 0 + x + O(x) \approx x, \qquad x<<1.\] Thus, if $Y = \exp\{\beta_0 + \beta_1X + U\},$ then $\Log(Y) \approx \beta_0 + \beta_1 X + U$
    We know that 
    \[\beta_1 = \frac{d\log Y}{dX} = \frac{1}{Y} \frac{dY}{dX} \approx \frac{\Delta Y}{\Delta X}\frac{1}{Y}\] And hence 
    \[\frac{\Delta Y}{Y} \approx \beta_1 \Delta X.\] Thus, 
    \[\boxed{\% \Delta Y \approx 100 \beta_1 \Delta X}\]
\end{rem}

\begin{rem}
    (Log Log Model)
    Suppose $\log(Y) = \beta_0 + \beta_1 \Log(X) + U$ and thus 
    \[\beta_1 = \frac{d\Log Y}{d\Log X} = \frac{1}{Y} \frac{1}{\frac{1}{X}}\frac{dY}{dX}\approx \frac{\% \Delta Y}{\% \Delta X}\] Thus, 
    \[\boxed{\%\Delta Y \approx \beta_1 \%\Delta X}\]
\end{rem}
\begin{rem}
    (Level Log Model) Similarly to before, if 
    $ Y = \beta_0 + \beta_1 \log X + U$, then 
    \[\boxed{\Delta Y \approx \frac{\beta_1}{100}\% \Delta X}\]
\end{rem}


\newpage
\subsection{Monday, June 30: Vector Statistics}






\begin{rem}
    Recall that $A^{-1}$ exists if $\det(A) \neq 0$ or if the columns of $A$ are linearly independent or the rows are. Recall that a vector $\textbf{x} = (x_1, \dots, x_n)^T$ is linearly dependent if there exists scalars $\textbf{c} = (c_1, \dots, c_n)$ such that 
    \[\textbf{c}\textbf{x} = c_1 x_1 + \dots c_n x_n = 0.\]

Suppose $X$ is a random vector such that 
\[X = \begin{pmatrix}x_{11} & x_{12}\\ x_{21} & x_{22}\end{pmatrix}\]
Then $\bbE[X]$ is the expected value of each of its entries. We have that \[\Var(X) = \bbE[(X - \bbE[X])(X - \bbE[X])^T]\] and the covariance matrix is 
\[\begin{bmatrix}
    \Var(X_1) &\Cov(X_1, X_2) & \cdots\\
    \vdots
\end{bmatrix}\] and thus $\Var(X)$ is symmetric. As an example, suppose $X_{n\times 1}$ is a r.v, and $A_{m \times n}$
 is a matrix of constants and $b_{m\times 1}$ is a column, then $\Var(A X + b) = A\Var(X)A^T$
 \end{rem}

\begin{thm}
\begin{center}
\fbox{
  \parbox{0.9\textwidth}{
    \textbf{(The big 4)} Suppose $X_1, \dots, X_n \sim X_{{k \times 1}}$ are i.i.d. Then the following hold, 
    \begin{enumerate}
        \item (\textbf{WLLN}) We have 
        \[\bar X \xrightarrow[\bbP]{}\bbE[X]\]\\
        \item (\textbf{CMT}) Suppose $X_n \xrightarrow[\bbP]{}x$ and $Y_n \xrightarrow[\bbP]{}y,$ and $g$ is continuous then 
        \[g(X_n, Y_n) \xrightarrow[\bbP]{}g(x,y)\]
        \item (\textbf{CLT}) Suppose the second moment of each element in $X$ is finite. Then 
        \[\sqrt{n}(\bar X - \bbE[X])\sim N(0, \Var(X))\]
        \item (\textbf{Slutsky's}) If $X_n \xrightarrow{d}X$ where $X$ is a random matrix and $Y_n \xrightarrow{\bbP} y$ is a constant matrix. Then 
    \begin{enumerate}
        \item $X_nY_N \xrightarrow{d}Xy$ when $Xy$ is defined.
        \item $X_n + Y_n \xrightarrow{d}X + y$ when $X + y$ is defined. 
        \item $X_n Y_n^{-1} \xrightarrow{d} X y^{-1}$ when $Xy$ is defined and $\det(y) \neq 0.$
    \end{enumerate}
    \end{enumerate}
    }}

\end{center}
\end{thm}

\begin{rem}
    If $X_{m\times 1} \sim \mathcal{N}(\bbE[X]_{m\times 1}, \Var(X)_{m\times m})$ then $A X + b$ is also multivariate normal with 
    \[AX  + b \sim \mathcal{N}(A \bbE[X] + b, A \Var(X)A^T)\]
\end{rem}



\begin{thm}
    If $X_{m \times 1} \sim \mathcal{N}(0_{m \times 1}, y_{m\times m})$ and $\det(y)\neq 0.$ Then $g(X, y)= X^T y^{-1} X \sim \chi_{\dim X}^2$. Moreover, suppose $X_n \xrightarrow X_{m\times 1} \sim \mathcal{N}(0, y_{m\times m})$ with $y$ invertible and $y_n \xrightarrow{\bbP} y_{m \times m}.$ Then 
    \[X_n^T y_n^{-1}X_n \xrightarrow{d} X^T y^{-1} X \sim \chi^2_{\dim(X)}\]
\end{thm}
\begin{rem}
    Suppose 
    \[Y = \beta_0 + \beta_1 X_1 + \cdots +\beta_kX_k + U\] Define 
    \[X_{k + 1\times 1} = (1, X_1, \dots, X_k)^T\] and 
    \[\beta_{k + 1 \times  1} = (\beta_0, \beta_1, \dots, \beta_k)^T.\] Then 
    \[Y = X^T \beta + U_{|X|}\]
\end{rem}
\begin{rem}
    \begin{enumerate}
    Again, there are three interpretations for the SLR:
        \item (Linear) Assume $\bbE[Y \mid X] = X^T \beta.$ Define $U = Y - \bbE[Y \mid X] = Y - X^T \beta.$ Then $Y = X^T \beta + U = \bbE[Y \mid X] + U$. Then the $\beta$ are not casual. But then 
        \[\bbE[U \mid X] = \bbE[Y - \bbE[Y \mid X] \mid X] = 0\] and thus $U$ is mean independent of $X.$ Moreover, 
        \[\bbE[X U] = \bbE[\bbE[XU \mid X]] = \bbE[X \bbE[U \mid X]] = 0.\] Note that this is enough (from PSET) to say that 
        \[\bbE[U] = 0.\]
        \item (BLP) The BLP $(Y \mid X)$ is the function that solves 
        \[\min_{b\in \bbR^{k+1}}\bbE[(Y - X^T b)^2]\] which can be shown to be equivalent to 
        \[\min_{b\in \bbR^{k+1}}\bbE[(\bbE[Y \mid X] - X^T b)^2].\] Then, once we find $\text{BLP}(Y \mid X) = X^T\beta,$ we define $U = Y - X^T\beta.$ Rewriting the minimization problem, we have that 
        \[\min_{b \in \bbR^{k+1}} \bbE[(Y - X^T b)^2]\] Taking derivative with respect to $b,$ we see that 
        \[\text{FOC}_b: \qquad -2\bbE[(Y- X^T \beta)X^T] = 0\] applying the transpose and ignoring the $-2,$ we see that (since $Y - X^T\beta$ is a scalar and is therefore its own transpose),
        \[\bbE[X(Y - X^T \beta)^T] = \bbE[X(Y - X^T \beta)] = \bbE[XU]= 0.\] So we get for free that $\bbE[XU] = 0,$ and thus $\bbE[U] = 0$ and $\Cov(X_j,U) = 0$ for any $j \in [k]$.
        \item (Causal Model) Assume 
        \[Y = g(X, U),\] where $X$ are the observed covariates of $Y$ and $U$ are the unobserved covariates. That is, if $g(X,U) = X^T\beta + U,$ then $Y = X^T\beta + U,$ where $\beta_j = \frac{\partial Y}{\partial X_j}$ is the causual effect of $X_j$ on $Y$, holding $X_{-j}$ and $U$ constant. Thus, 
        \[Y = \beta_0 + X_{-0}^T\beta_{-0}+ U = (\beta_0 + \bbE[U]) + X_0^T\beta_{-0} + (U - \bbE[U]) = \beta_0' + X_{-0}^T \beta_{-0} + U'\] Hence, 
        \[\bbE[U'] = 0 \qquad \Cov(X_j, U) \neq 0\]
    \end{enumerate}
\end{rem}


\newpage
\subsection{Monday, July 7: Interactions}
\begin{defn}
    (Notation) We notate
    \[X_{-j} = (1, X_1, \dots, X_{j-1}, X_{j+1}, \dots, X_k)^T\]
\end{defn}
\begin{exmp}
    (Non Linear) 
    Suppose 
    \[Y_i = \beta_0 + \beta_1 X_1 + \beta_2 X_2^2 + \beta_1X_1^3 + \beta_4 X_2 + U_i\]
    Hence, 
    \[\frac{\partial \text{BLP}}{\partial X_1} = \beta_1 + 3\beta_1 \] can be interpreted as the effect of $X_1$ on $Y,$ here $\beta_1$ is the effect when $X _1= 0,$ $\beta_2$ is the sensitivity of the $Y$ with respect to $X_1$ (if positive, then $X_1$ has an increasing effect on $Y$).  
\end{exmp}
\begin{exmp}
    (Interactions) 
    \begin{enumerate}
        \item (Dummy + Cont) Suppose $Y = \beta_0 + \beta_1 X_1 + \beta_2 X_2 + U,$ where $X_1$ is $1$ or $0$ and $X_2$ is a continuum and let's assume a causal model. Then $\beta_1$ is the effect of $X_1$ on $Y$ regardless of $X_2.$ And vice-versa for $\beta_2.$ The problem is that there is no way of measuring the interaction between $X_1$ and $X_2.$ Consider now 
        \[Y = \beta_0 + \beta_1X_1 + \beta_2 X_2 + \beta_3 X_1X_2 + U.\] Suppose $X_1 = 0,$ then $Y = \beta_0' + \beta_2' X_2 + u.$ If $X_1 = 1,$ then $Y = (\beta_0 + \beta_1) + (\beta_2 + \beta_3)X_2 + U,$ which is great because this shows there is a different intercept $(\beta_0 + \beta_1)$ and slope $(\beta_2 + \beta_3)$ for different $X_1.$ This is able to capture the interaction much better. 
        \item (Cont + Cont) Suppose $Y = \beta_0 + \beta_1X_1 + \beta_2 X_2 + \beta_3 X_1X_2 + U.$ Then 
        \[\frac{\partial\text{BLP}}{\partial X_1} = \beta_1 + \beta_3 X_2\] Then $\beta_3$ is the sensitivity of $X_1$ on $Y$ with respect to $X_2.$
        \item (Dummy + Dummy/Difference in Differences) Suppose 
        \[Y = \beta_0 + \beta_1 X_1 + \beta_2 X_2 + \beta_3 X_1 X_2 + U,\] where both $X_1$ and $X_2$ are binary

\begin{table}[H]
    \centering
    \begin{tabular}{cccl}
       $X_1/X_2$&$0$& $1$ &Diff\\
         $0$&  $\beta_0$&  $\beta_0  + \beta_2$&$\beta_2$\\
         $1$&  $\beta_0 + \beta_1$&  $\beta_0 + \beta_1+ \beta_2 + \beta_3$&$\beta_2 + \beta_3$\\
 Diff& $\beta_1$& $\beta_1 + \beta_3 $&$\beta_3$\\
    \end{tabular}
    \caption{Interactions between Dummies}
    \label{tab:my_label}
\end{table}

$\beta_3$ is known as the difference between differences coefficient.
    \end{enumerate}
\end{exmp}


\begin{defn}
    We say that $X_{k+1 \times 1}$ is \textbf{perfectly colinear} if there exists some $\textbf{c} = (c_1, \dots, c_{k+1})^T \neq \textbf{0}$ such that 
    \[\textbf{c}X = 0\]
\end{defn}
\begin{lem}
    Suppose $X$ is not perfectly colinear, then $\bbE[XX^T]$ is invertible. 
\end{lem}
\begin{proof}
    Suppose not. Then there exists some $\textbf{c}\neq 0 $ such that 
    \begin{align*}
        0 &= \bbE[XX^T]\textbf{c}\\
        &= \textbf{c}^T \bbE[XX^T]\textbf{c}\\
        &= \bbE[\textbf{c}^T XX^T\textbf{c}]\\
        &= \bbE[(\textbf{c}X)^2]
    \end{align*} Implying that $\textbf{c}X= 0$ and thus $X$ is perfectly colinear, a contradiction.
\end{proof}

\begin{rem}
    Let $X_1, X_2$ be binary. Then if $X$ contains $X_1$ and $X_2,$ then $X$ is perfectly colinear, as $0 = 1-(X_1 + X_2)$.     If $X_1$ and $X_2$ are colinear. DO NOT build a regressionn with 
    \[Y = \beta_0 + \beta_1 X_1 + \beta_2 X_2 + U\], instead do difference in means 
    \[Y = \beta_0 + \beta_1 X_2 + U\] where $\beta_0$ and $\beta_1$ are the difference in means
    
    or do 
    \[Y = \beta_1 X_1 + \beta_2 X_2 + U\]
\end{rem}




\newpage
\subsection{Wednesday, July 9: $\beta$ Theory}
\begin{thm}
\begin{center}
\fbox{
  \parbox{0.9\textwidth}{
    \textbf{(Deriving $\beta$)} let $Y_{1\times 1}, X_{k +1\times 1 }, U_{k+1\times 1}$ be R.V.s with $Y = X^T \beta + U$ such that
    \begin{enumerate}
        \item $\bbE[XU] = 0$ (which implies $\bbE[U] = 0, \Cov(X_j, U) = 0$). 
        \item No perfect co-linearity in $X.$ 
        \item $\bbE[XX^T] < \infty$ (which implies $\bbE[X_j^2] < \infty$ and $\bbE[X_jX_s] < \infty$)
    \end{enumerate}
    Let $(Y^1, (X^1)^T), \dots, (Y^n, (X^n)^T)\sim (Y, X^T)$ i.i.d. Then 
    \begin{align}
\beta = \bbE[XX^T]^{-1}\bbE[XY]        
    \end{align}
    
    }}
\end{center}

\end{thm}
\begin{proof}
    From (a), we see that 
    \begin{align*}
        0 &= \bbE[XU]\\
        &= \bbE[X(Y - X^T\beta)]\\
        &= \bbE[XY] - \bbE[XX^T]\beta
    \end{align*}
    Rearranging, we see that $\bbE[XX^T]\beta = \bbE[XY],$ and thus we use Lemma 4 to conclude.
\end{proof}

\begin{thm}
\begin{center}
\fbox{
  \parbox{0.9\textwidth}{
    (\textbf{Frisch-Waugh-Lovell}) With the same assumption as in Theorem 9, define 
    \begin{enumerate}
        \item $Y^* := Y - \text{BLP}(Y \mid X_{-j})$
        \item $X_j^* := X_j - \text{BLP}(X_j \mid X_{-j})$
    \end{enumerate} If
    \[Y^* = \beta_0^* + \beta_j^*X_j^* + U^*,\] then $\beta_j^*=  \beta_j$
    
    }}
\end{center}

\end{thm}


\begin{proof}
    From the PSET,
    \begin{itemize}
        \item  $\Cov(X_j^*, X_\ell) = 0$ for all $\ell \neq j$. 
        \item $\Cov(X_j^*, X_j) = \Var(X_j^*)$ (decompose $X_j$ with BLP)
        \item $\Cov(X_j^*, U) = 0$ open up $X_j^*$
    \end{itemize}
    Denoting $\text{BLP}(X_j \mid X_{-j}) = \alpha_0 + \alpha_1 X_1 + \cdots + \alpha_{j-1}X_{j-1} + \alpha_{j+1}X_{j+1} + \cdots + \alpha_k X_k.$ Then computing,
    \begin{align*}
      \beta_j^* &= \frac{\Cov(X_j^*, Y^*)}{\Var(X_j^*)} \\
      &= \frac{\Cov(X_j^*, Y) - \Cov(X_j^*, \text{BLP}(Y \mid X_{-j}))}{\Var(X_j^*)}\\
      &= \frac{\Cov(X_j^*, Y)}{\Var(X_j^*)}\qquad \qquad \qquad \qquad\qquad\text{by (a)}\\
      &= \frac{\Cov(X_j^*, \beta_jX_j + X_{-j}^T \beta_{-j} + U)}{\Var(X_j^*)}\\
      &= \frac{\beta_j \Cov(X_j^*, X_j) + \Cov(X_j^*, X_{-j}^T \beta_{-j}) + \Cov(X_j^*, U)}{\Var(X_j^*)}\\
      &= \beta_j
    \end{align*}
    
\end{proof}
\begin{rem}
    By the FWL thm, we interpret $\beta_j$ as the $\text{BLP}(Y \mid X)$ as partial statistical association between $X_j$ and $Y,$ controlling for $X_{-j}.$ We loosely call this partial correlation:
    \[Z_y = \frac{Y - \mu_y}{\sigma_Y}, \quad Z_{X_j} = \frac{X_j - \mu_{X_j}}{\sigma_{{X_j}}},\] and 
    \[Z-y= \beta_1 Z_{X_1} + \cdots + \beta_k Z_{X_k} + U,\] where $\beta_1$ is the partial correlation between $X_1$ and $Y.$\\

    Moreover, FWL \underline{also works in the sample}. That is, definiting all equal to the above but in the sample, 
    \[\hat\beta_j^* = \hat\beta_j\]
\end{rem}

\begin{thm}
\begin{center}
\fbox{
  \parbox{0.9\textwidth}{
    (\textbf{Estimating $\beta$}) With the assumptions of Theorem 9, the OLS estimator of $\beta$ is given by 
    \begin{align}
        \hat\beta = \left(\frac{1}{n}\sum X^i (X^i)^T\right)^{-1} \left(\frac{1}{n}\sum X^i Y^i\right)
    \end{align}
    }}
\end{center}

\end{thm}
\begin{proof}
    We seek 
    \[\min_{b \in \bbR^{k+1}}\frac{1}{n}\sum (Y^i - (X^i)^Tb)^2 = 0.\] Taking derivative, we see that the first order conditions imply 
    \begin{align*}
        (0)^T &= \left(\sum  (Y^i - (X^i)^T\beta)(X^i)^T\right)^T\\
        &= \sum X^i(Y^i - (X^i)^T\beta)\\
        &= \sum X^i Y^i - \beta\sum X^i(X^i)^T
    \end{align*}
    Rearranging we get the result.  
\end{proof}


\begin{rem}
    In the step when we apply the transpose, we can see that 

\begin{align}
    \frac{1}{n} \sum X^i \hat{U}^i = 0
\end{align}
     which are the mechanical equations of the OLS. This of course implies that 
\begin{align}
    \frac{1}{n} \sum \hat U^i = 0
\end{align}
This is similar to how $\bbE[XU] = \bbE[U] = 0.$
\end{rem}

\begin{defn}
    We say that the \textbf{fitted/predicted} value is 
    \[\hat Y^i := (X^i)^T \hat\beta.\]  We define the \textbf{residual} is 
    \[\hat U^i := Y^i - \hat Y^i = Y^i - (X^i)^T \hat\beta\]
\end{defn}
\begin{rem}
    We still define 
    \[R^2 = \frac{\text{ESS}}{\text{TSS}} = 1 - \frac{SSR}{TSS}\] with $R^2 \in (0,1).$ However, we claim that $R^2$ never decreases with the inclusion of a new regressor. That is, $\text{SSR}$ never increases with this inclusion. Thus, we motivate the definition of the \textbf{adjusted $R^2$} 
    \[\overline{R}^2 := 1- \frac{(n-1)}{(n-k-1)} \frac{\text{SSR}}{\text{TSS}}\]
\end{rem}



\begin{thm}
\begin{center}
\fbox{
  \parbox{0.9\textwidth}{
    \textbf{(Deriving $\beta$)} let $Y_{1\times 1}, X_{k +1\times 1 }, U_{k+1\times 1}$ be R.V.s with $Y = X^T \beta + U$ such that
    \begin{enumerate}
        \item $\bbE[XU] = 0$ (which implies $\bbE[U] = 0, \Cov(X_j, U) = 0$). 
        \item No perfect co-linearity in $X.$ 
        \item $\bbE[XX^T] < \infty$ (which implies $\bbE[X_j^2] < \infty$ and $\bbE[X_jX_s] < \infty$)
    \end{enumerate}
    Let $(Y^1, (X^1)^T), \dots, (Y^n, (X^n)^T)\sim (Y, X^T)$ i.i.d. The OLS estimator for $\beta$ satisfies the following:
    \begin{enumerate}
        \item (\textbf{Unbiased}) If $\bbE[U \mid X] = 0,$ then \[\bbE[\hat\beta] = \beta\]
        \item (\textbf{Consistent}) If $\bbE[Y_j^2] < \infty$ for all $j = 1,2,\dots, k+1,$ then 
        \[\hat \beta \xrightarrow[\bbP]{}\beta\]
        \item (\textbf{Asymptotic Distribution}) If $\bbE[X_j^4] < \infty$ and $\bbE[Y_j^4] < \infty,$ then 
        \[\sqrt{n}(\hat \beta - \beta) \sim N(0, \Sigma)\] where 
        \[\Sigma = \bbE[XX^T]^{-1}\Var(XU)\bbE[XX^T]^{-1}\]
    \end{enumerate}
    
    }}
\end{center}

\end{thm}

\begin{proof}
    \begin{enumerate}
        \item We have that 
        \begin{align*}
            \hat{\beta} &= \left(\frac{1}{n}\sum X^i (X^i)^T\right)^{-1} \left(\frac{1}{n}\sum X^i Y^i\right)\\
            &= \left(\frac{1}{n}\sum X^i (X^i)^T\right)^{-1} \left(\frac{1}{n}\sum X^i ((X^i)^T\beta + U^i)\right)\\
            &= \beta + \left( \frac{1}{n}\sum X^i(X^i)^T\right)^{-1} \frac{1}{n}\sum X^iU^i
        \end{align*}
        Taking conditional expectation of 
        \begin{align}
            \hat{\beta} = \beta + \left( \frac{1}{n}\sum X^i(X^i)^T\right)^{-1} \frac{1}{n}\sum X^iU^i,
        \end{align} 
        we see that if we use (a) on (12), 
        \[\bbE[\hat \beta \mid X_1, \dots, X_n] = \beta_1 +\left( \frac{1}{n}\sum X^i(X^i)^T\right)^{-1} \frac{1}{n}\sum X^i\bbE[U^i \mid X_1, \dots, X_n]  = \beta_1\] Conclude with LIE.
        \item By the WLLN, we have the convergence of 
        \[\frac{1}{n}\sum X^i (X^i)^T \xrightarrow[\bbP]{}\bbE[XX^T]\]
        \[\frac{1}{n}\sum X^i Y^i \xrightarrow[\bbP]{} \bbE[XY]\] so then using the CMT with $g(A,B) = A^{-1}B$ and Lemma 4, we conclude that 
        \[\hat{\beta} = g(\frac{1}{n}\sum X_i X_i^T, \frac{1}{n}\sum X_i Y_i)\xrightarrow[\bbP]{}g(\bbE[X_iX_i^T], \bbE[X_iY_i])  = \beta\]
        \item From equation (12) we use the WLLN, CLT, CMT, and Slutsky to see that
        \begin{align*}
            \sqrt{n}(\hat \beta - \beta) &=  \left( \frac{1}{n}\sum X^i(X^i)^T\right)^{-1} \frac{1}{\sqrt{n}}\sum X^iU^i\\
            &\xrightarrow[\bbP]{}\bbE[XX^T]^{-1} \frac{1}{\sqrt{n}}\sum X^iU^i\\
            &\xrightarrow[\cal D]{}\bbE[XX^T]^{-1}N(0, \Var(XU))\\
            &= N\left(0, \bbE[XX^T]^{-1}\Var(XU)\bbE[XX^T]^{-1}\right)
        \end{align*}
        Where the last equality is due to $\bbE[XX^T]$ being invertible and thus symmetric. 
    \end{enumerate}
\end{proof}

\begin{rem}
    \textbf{(Omitted Variable Bias)}
    Suppose 
    \[Y = \beta_0 + \beta_1 X_1 + \beta_2 X_2 + U\] satisfies the conditions of Theorem 12 (b). Suppose it is difficult to measure $X_2$ directly, so consider estimating with OLS the new 
    \[Y = \alpha_0+ \alpha_1 X_1  + V.\] We know by Proposition 7 that 
    \begin{align*}
    \hat\alpha_1 \xrightarrow[\bbP]{}\alpha_1 &= \frac{\Cov(X_1,Y)}{\Cov(X_1)}\\
    &= \frac{\Cov(X_1, \beta_0 + \beta_1 X_1 + \beta_2 X_2 + U)}{\Var(X_1)}\\
    &= \beta_1 + \beta_2 \frac{\Cov(X_1, X_2)}{\Cov(X_2)}
    \end{align*}
    \begin{center}
\fbox{
  \parbox{0.9\textwidth}{
    \textbf{(OVB)}   \begin{table}[H]
        \centering
        \begin{tabular}{cccc}
            $$\beta_2$$/($\Cov(X_1, X_2$)& $+$ & $-$ &$0$ \\
            $+$ & $+$ & $-$  & $0$\\
            $-$ &  $-$& $+$ & $0$\\
             $0$&  $0$&  $0$& $0$\\
        \end{tabular}
        \caption{Effects of OVB}
    \end{table}
    }}
\end{center}
\end{rem}

\begin{exmp}
    Suppose a causal MLR of
    \[\text{muscle mass} = \beta_0 + \beta_1 \text{gymtime} + \beta_2 \text{genes} + U\]
    Genes are hard to measure, so we consider estimating the following with OLS
    \[\text{muscle mass}= \alpha_0 + \alpha_1 \text{gymtime} + V.\] By OLS, We require that $\Cov(V, \text{gymtime}) = 0.$  The following step would be to investigate $\beta_2$ and $\Cov(X_1,X_2)$ and use Table 2.
\end{exmp}

\begin{rem}
    (\textbf{Measurement Error})
    \begin{itemize}
        \item A measurement error in $Y$ is benign, and doesn't cause OVB, only increases $\Var(\hat\beta_j)$
        \item Measurement error in $X$ is bad. Common case is classical measurement error (CME). 
        To see this, suppose 
        \[Y = \beta_0 + \beta_1 X + U,\] but a researcher estimates \[Y = \alpha_0 + \alpha_1 X^* + V\] with OLS, where $X^*$ is a ill-measured $X$ with $X^* = X + Z,$ where $\bbE[Z] = \Cov(X,Z) = \Cov(U,Z) = 0.$ We know that 
        \begin{align*}
            \hat \alpha_1 \xrightarrow[\bbP]{}\alpha_1 &= \frac{\Cov(X^*, Y)}{\Cov(X^*)}\\
            &= \frac{\Cov(X + Z, \beta_0 + \beta_1 X + U)}{\Var(X + Z)}\\
            &= \frac{\beta_1\Var(X) + \Cov(X,U) + \beta_1\Cov(Z,X) + \Cov(Z,U)}{\Var(X) + \Var(Z) + 2\Cov(X,Z)}
            \end{align*}
            Using various assumptions, we have derived 
        \begin{center}\fbox{
  \parbox{0.9\textwidth}{
    \textbf{(Attenuation Bias)}   
    \[
\hat{\alpha_1} \xrightarrow[\bbP]{} \beta_1\underbrace{ \frac{\Var(X)}{\Var(X) + \Var(Z)}}_{\text{attenuation bias, $\leq 1$}}
\]
Measurement error always pulls $\hat{\alpha}_1$ towards $0.$ 

    }}
\end{center}
    \end{itemize}
\end{rem}

\begin{prop}
If $U$ is homoskedastic, then 
    \[\Sigma = \bbE[XX^T]^{-1}\Var(U).\]
    \end{prop}
\begin{proof}
Since $U$ is homoskedastic, then $\bbE[U \mid X] = 0$ and $\Var(U \mid X) = \Var(U).$ Hence, 

    \begin{align*}
        \Var(X U)  &= \bbE[XX^T U^2] - \bbE[XU]\bbE[XU]^T\\
        &= \bbE[XX^T \bbE[U^2 \mid X]]\\
        &= \bbE[XX^T \Var(U \mid X)]\\
        &= \bbE[XX^T \Var(U)] \qquad \text{(homosk)}\\
        &= \Var(U)\bbE[X X^T]
    \end{align*}
    Thus, 
    \begin{align*}
        \Sigma &= \bbE[XX^T]^{-1} \Var(U)\bbE[XX^T] \bbE[XX^T]^{-1} = \Var(U)\bbE[XX^T]^{-1}
    \end{align*}
\end{proof}

\begin{rem}
    If  $U$ is not homosk, then using the analogy principle, 
    \begin{align}
\hat \Sigma = \left[\frac{1}{n}\sum X^i (X^i)^T\right]^{-1} \left[\frac{1}{n} \sum  (\hat U^i) ^2 X^i(X^i)^T\right]\left[\frac{1}{n}\sum X^i (X^i)^T\right]^{-1}        
    \end{align}

\end{rem}

\newpage
\subsection{Friday, July 11: Hypothesis Testing}

\begin{lem}
    Suppose $X_n \xrightarrow[\cal D]{}N(0, y_n)$ and $y_n \xrightarrow[\bbP]{}y.$ Then 
    \[X_n ^Ty^{-1}X_n \xrightarrow[\cal D]{}X^Ty^{-1}X \sim \chi_{\dim (X)}^2\]
\end{lem}

\begin{rem}
    \textbf{(Hypothesis Testing for Linear Combinations of $\beta$)}
    \begin{enumerate}
        \item[(1)] $H_0: R\beta = r,$ $H_a: R\beta \neq r,$ where, usually $r = \textbf{0}$ and $R$ is the matrix testing the linear combinations. 
        \item[(2)] Since $\sqrt{n}(\hat\beta - \beta) \xrightarrow[\cal D]{} N(0, \Sigma),$ then by Slutsky's Lemma
        \[\sqrt{n}(R \hat\beta - R\beta)\xrightarrow[\cal D]{}N(0, R\Sigma R^T).\] As usual, $\Sigma$ is unattainable, so we use (13) as a consistent estimator. From Lemma 5,
        \begin{center}\fbox{
  \parbox{0.9\textwidth}{
    \textbf{(Wald's Statistic)}   
    \[
T_n = n(R\hat\beta - R\beta)^T (R\hat\Sigma R^T)^{-1} (R\hat\beta - R\beta) \xrightarrow[\cal D]{} \chi_{\dim(R\beta)}^2
\]

    }}
\end{center}
\item [(3)] The usual end for hypothesis testing.
    \end{enumerate}
\end{rem}

\begin{exmp}
    
\end{exmp}



\begin{exmp}
Suppose $Y = \beta_0 + \beta_1 X_1 + \beta_2 + U,$ with 
    \[H_0:\beta_1 + 2\beta_2 = 0 \quad H_a: \beta_1 + 2\beta_2 \neq 0.\] Then 
    \[R = \begin{bmatrix}
        0 & 1 & 2
    \end{bmatrix}\] and $r = \textbf{0}.$
\end{exmp}

    \begin{exmp}
       \[ H_0: \beta_1, \beta_2 = 0 \quad H_a: \beta_1 \neq 0 \cup \beta_2 \neq 0\], then 
       \[R = \begin{bmatrix}
           0 &1 & 0\\
           0 & 0 & 1
       \end{bmatrix} \implies R\beta = \begin{bmatrix}
           \beta_1 \\ \beta_2
       \end{bmatrix}\] Under $H_0,$ $R\beta = r,$ where $r = \begin{bmatrix}
            0 \\ 0
       \end{bmatrix}$. Recall that 
       \[\sqrt{n}(\hat{\beta} - \beta) \to N(0, \Sigma) \implies \sqrt{n}(R\hat{\beta} - R\beta) \to N(0, R \Sigma R^T)\] we can measure the scalar of how far away the data is from the values:
       \[n(R\hat\beta - R\beta)^T(R\sum R^T)^{-1}(R\hat\beta - R\beta)\]
    \end{exmp}

    Let 
    \[T_n = n(R\hat\beta - R\beta^{H_0})^T(R \hat\Sigma R^T)^{-1}(R\hat\beta - R\beta) \xrightarrow[d]{}\chi^2_{\dim (R\beta)}\]



\begin{rem}
(\textbf{F-Statistic Hypothesis Testing})
\underline{Assume $U$ is homoskedastic}., and consider the SLR
\[Y^i = \beta_0 + \beta_1 X_{1}^i + \beta_2 X_{2}^i + U^i \quad (\text{unrestricted model}).\] 
\begin{enumerate}
    \item [(1)] \[H_0: \beta_1, \beta_2 = 0 \qquad H_a: \beta_1 \neq 0\cup \beta_2 \neq 0\]
    \item [(2)] Under $H_0:$ \[Y^i = \beta_0  + U^i \quad (\text{restricted model})\] Compute 
    \[F_n = \frac{\frac{1}{q}\left(\text{SSR}_r - \text{SSR}_{ur}\right)}{ \frac{1}{(n - k_{ur} - 1)}\text{SSR}_{ur}} = \frac{\frac{1}{q}\left(R^2_{ur} - R^2_r\right)}{ \frac{1}{(n - k_{ur} - 1)}(1-R^2_{ur})}, \] where $q$ is the number of constraints, and $K_{ur}$ is the number of regressor in the $ur$ model.
    \item [(3)] $F$ is distrbuted as $F_{q, \,n-K_{ur}- 1}$
\end{enumerate}


\newpage
\subsection{Monday, July 14: Instrument Variables in SLR}
\begin{defn}We say that $X_j$ is \textbf{endogenous} if $\Cov(X_j, U) \neq 0.$ Else, we say that $X_j$ is \textbf{exogenous.}
\end{defn}

Until otherwise stated, we consider 
\[Y = X^T\beta + U,\] where there is at least one endogenous variable. From Remark 22, (c), we note that $\bbE[U] = 0.$ how do we estimate $\beta$ in this scenario?
\begin{rem}
    There are three main sources of endogeneity:
    \begin{enumerate}
        \item OVB (when $U$ is alive, see Example 1.9)
        \item Measurement Error (When $X$ is not quite right, see Remark 28)
        \item Simultaneity Bias (See Example 1.19)
    \end{enumerate}
\end{rem}

\begin{defn}
    Suppose $Y = X^T\beta + U.$ We say that a r.v. $Z$ is an \textbf{instrument} if 
        \begin{itemize}
            \item $\Cov(Z,U) = \bbE[ZU]= 0$ (instrument exogeneity)
            \item $\Cov(X,Z) \neq 0$ (instrument relevance)
        \end{itemize} 
\end{defn}
\begin{exmp}
    (Civil conflict in Africa) Consider 
    \[\text{conflict}_i = \beta_0 + \beta_1 \text{growth}_i + U_i,\] and consider 
    \[Z_i: \text{rainfall} \text{ s.t. } \Cov(X,Z) \neq 0, \Cov(Z,U) = 0\]
\end{exmp}
\hline
\begin{rem}
    (\textbf{Case 1: SLR}) 
\begin{defn}
    Suppose $Y = \beta_0 + \beta_1 X + U.$ We say that a r.v. $Z$ is an \textbf{instrument} if 
        \begin{itemize}
            \item $\Cov(Z,U) = \bbE[ZU]= 0$ (instrument exogeneity)
            \item $\Cov(X,Z) \neq 0$ (instrument relevance)
        \end{itemize} 
\end{defn}    
Suppose $Y= \beta_0 + \beta_1 X + U,$ where $X$ is endogenous. 
    
    We derive
    \begin{align*}
        0 &= \bbE[U]\\
          &= \bbE[Y] - \beta_0 - \beta_1 \bbE[X]
    \end{align*}
    gives us 
    \begin{align}
        \beta_0 = \bbE[Y] - \beta_1 \bbE[X]
    \end{align}
    Plugging in (14)
    \begin{align*}
        0 &= \bbE[ZU]\\
        &= \bbE[Z(Y - \beta_0 - \beta_1 X)]\\
        &= \bbE[Z(Y - \bbE[Y] + \beta_1 \bbE[X] - \beta_1 X)]\\
        &= \bbE[Z(Y - \bbE[Y])] - \beta_1 \bbE[Z(X - \bbE[X])]\\
        &= \Cov(Z,Y) - \beta_1 \Cov(Z,X)
    \end{align*} give us 
    \begin{align}
        \beta_1 &= \frac{\Cov(Z,Y)}{\Cov(Z,X)}
    \end{align}
    Note that when $X$ is binary, this nicely simplifies to 
    \begin{center}\fbox{
  \parbox{0.9\textwidth}{
    \textbf{(Local Average Treatment)}   
    \[
 \beta_1 = \frac{\bbE[Y \mid Z = 1] - \bbE[Y \mid Z = 0]}{\bbE[X \mid Z = 1] - \bbE[X \mid Z = 0]} = \frac{\pi_1}{\alpha_1}
\]

    }}
\end{center}
    
    where $\pi_1$ and $\alpha_1$ are the coefficients from regressing $Y$ and $X$ on $Z,$ respectively. 
\begin{thm}
\begin{center}
\fbox{
  \parbox{0.9\textwidth}{
    \textbf{(Estimation with SLR Instrument Variables)} let $Y,X,U,Z$ be R.V.s with $Y = \beta_0 +  \beta_1X + U$ such that
    \begin{enumerate}
        \item $\bbE[U] = 0$. 
        \item $\bbE[ZU] = 0$ (Exogeneity).
        \item $\Cov(X,Z) \neq 0$
    \end{enumerate}
    Let $(Y^1, X^1, Z^1), \dots, (Y^n, X^n, Z^n)\sim (Y, X, Z)$ i.i.d. Then the IV estimator for $\beta$ given by 
\begin{align}
    \hat\beta_1^{IV} &= \frac{\hat \sigma_{ZY}}{\hat \sigma_{ZX}}\\
    \hat\beta_0^{IV} &= \bar Y - \hat\beta_1^{IV} \bar X
\end{align}
satisfies
    
    \begin{enumerate}
    \item The sample equivalents of (a) and (b):
    \begin{align*}
        0 &= \frac{1}{n} \sum (Y^i - \hat\beta_0^{IV} - \hat \beta_1^{IV} X^i) = \frac{1}{n}\sum \hat U^i\\
        0 &= \frac{1}{n}\sum Z^i(Y^i - \hat\beta_0^{IV} - \hat\beta_1^{IV} X^i) = \frac{1}{n}\sum Z^i \hat U^i
    \end{align*}
        \item (\textbf{Consistent}) If $\bbE[Y^2], \bbE[Z^2], \bbE[X^2] < \infty,$ then both $\hat\beta_0^{IV}$ and $\hat\beta_1^{IV}$ are consistent.
        \item (\textbf{Asymptotic Distribution}) If $\bbE[X^4], \bbE[Y^4], \bbE[Z^4] < \infty$, then 
        \[\sqrt{n}(\hat \beta_1^{IV} - \beta_1) \sim N(0, \sigma_{1,IV}^2)\] where 
        \[\sigma_{1, IV}^2 = \frac{\Var((Z - \bbE[Z])U)}{\Cov^2(X,Z)}\]
    \end{enumerate}
    
    }}
\end{center}

\end{thm}

\begin{proof}
    \begin{enumerate}
        \item From the first equation, we can derive $\hat\beta_0^{IV}.$
        \begin{align*}
            0 &= \frac{1}{n}\sum(Y^i - \hat\beta_0^{IV} - \hat\beta_1^{IV} X^i)\\
            &= \bar Y - \hat\beta_0^{IV} - \hat\beta_1^{IV} \bar X
        \end{align*} Similarly, we plug in (17) into the second equation to see how 
        \begin{align*}
            0 &= \frac{1}{n}\sum Z^i(Y^i - \hat\beta_0^{IV} - \hat\beta_1^{IV} X^i)\\
            &= \frac{1}{n}\sum Z^i(Y^i - (\bar Y - \hat\beta_1^{IV} \bar X) - \hat\beta_1^{IV} X^i)\\
            &= \frac{1}{n}\sum (Y^i - \bar Y)Z^i - \hat\beta_1^{IV} \frac{1}{n}\sum (X^i - \bar X)Z^i
        \end{align*}
        yields the answer.
        \item We know that $\hat\sigma_{YZ} \xrightarrow[\bbP]{}\sigma_{YZ}$ and $\hat \sigma_{XZ}\xrightarrow[\bbP]{}\sigma_{XZ}.$  use continuous mapping theorem to conclude.
        \item We can write
        \begin{align}
            \hat\beta_1 ^{IV}&= \frac{\sum (Z^i  - \bar Z)Y^i}{\sum (Z^i - \bar Z)X^i}=\frac{\sum (Z^i  - \bar Z)(\beta_0 + \beta_1X^i + U^i)}{\sum (Z^i - \bar Z)X^i} = \beta_1  + \frac{\sum(Z^i - \bar Z)U^i}{\sum (Z^i - \bar Z)X^i}
        \end{align}
        From (18) we see that, by using a combination of WLLN, CLT, then WLLN, then Slutsky's Lemma, 
        \begin{align*}
            \sqrt{n}(\hat\beta_1^{IV} - \beta_1) &= \frac{\frac{1}{\sqrt{n}}\sum(Z^i - \bar Z)U^i}{\frac{1}{n}\sum(Z^i - \bar Z)X^i}\\
            &\xrightarrow[\cal \bbP]{} \frac{\frac{1}{\sqrt{n}}\sum(Z^i - \bbE[Z])U^i}{\frac{1}{n}\sum(Z^i - \bbE[Z])X^i + \frac{1}{n}\sum (\bbE[Z] - \bar Z)X^i}\\
            &\xrightarrow[ \bbP]{} \frac{\frac{1}{\sqrt{n}}\sum(Z^i - \bbE[Z])U^i}{\Cov(Z,X)}\\
            &\xrightarrow[\cal D]{}\frac{N(0, \Var((Z - \bbE[Z])U))}{\Cov(Z,X)}
        \end{align*}
        As usual, we usually estimate this variance by 
            \[\hat\sigma_{1, IV}^2 = \frac{\sum (Z_i - \bar Z)\hat U_i}{\sum (Z_i - \bar Z)(X_i - \bar X)}\]
    \end{enumerate}
\end{proof}

\begin{rem}
    (\textbf{Biased}) What happens to the bias? In order to establish unbiasness, it is clear from (18) that we would need $\bbE[U^i \mid X^i, Z^i]= 0.$ But then using the LIE it is clear that $\bbE[X^iU^i] = 0.$ But this then implies that $\Cov(X^i, U^i) = 0,$ a contradiction to $X^i$ being endogenous.
\end{rem}

\begin{exmp}
    (\textbf{Simultaneity Problem})
Let 
                \[Q^d(p) = \beta_0 + \beta_1 p + U^d\]
                \[Q^s(p) = \alpha_0 + \alpha_1 p + \alpha_2 Z + U^s\]
assume $\beta_1 < 0, \alpha_1 >0,$ and $\Cov(U^d, U^s) = 0$ and $Z$ is a supply shifter with $\Cov(Z,U^s) = \Cov(Z,U^d) = 0$ and $\Var(Z)>0.$ Suppose further that $\alpha_2 \neq 0.$ 


Solving for $P$ in $Q(P) = Q(P) = Q$ gives 
\[\frac{1}{\alpha_1 - \beta_1}(\beta_0 - \alpha_0 - \alpha_2 Z + U^d - U^s),\] and hence $P$ is endogenous in both $Q^d(P)$ and $Q^s(P)$ with 
\[\Cov(P,U^d) = \frac{\Var(U^d)}{\alpha_1-\beta_1}>0 ,\qquad \Cov(P,U^d) = \frac{-\Var(U^s)}{\alpha_1-\beta_1} <0\] From OVB analysis, the above imply that (Example 1.9) the BLP coefficients overestimate $\beta_1$ and underestimate $\alpha_1.$\\

It can be worked out that $\Cov(Z,P) \neq 0$ given the assumptions, implying that $Z$ is an instrument variable and we can estimate a consistent estimator!
\end{exmp}
\end{rem}

\hline


\begin{rem}
    (\textbf{Case 2: MLR})    For the following case, we will consider the model $Y = X^T\beta + U,$ where $X_1$ is endogenous and $X_{-1}$ are exogenous and (WLOG) $\bbE[U] = 0.$
\begin{defn}
    Suppose $Y = X^T\beta + U.$ We say t We say that a r.v. $Z$ is an \textbf{instrument} if 
        \begin{itemize}
            \item (instrument exogeneity) $\Cov(Z,U) = \bbE[ZU]= 0$ 
            \item (instrument relevance) Letting $W = (1, Z, W_2, \dots, W_k)$ not be perfectly colinear and $\pi = (\pi_0, \pi_1, \dots, \pi_k)$ such that $\pi_1 \neq 0,$ then 
            \[X_1 = \text{BLP}(X_1 \mid W) + V = W^T\pi\]
        \end{itemize} 
\end{defn}

\begin{prop}
   Suppose $Y = X^T\beta + U$ and $X_1$ is the only endogeneous variable. Then the following are equivalent:
   \begin{enumerate}
       \item $Z$ is relevant;
       \item $\bbE[WX^T]$ is invertible;
       \item $\bbE[WW^T]^{-1}\bbE[WX^T]$ is invertible
   \end{enumerate}
\end{prop}
\begin{proof}
    ($a\iff c$) Suppose $Z$ is relevant. Consider regressing $X$ by $W.$ then the BLP coefficient is given by (see Theorem 9) \[\alpha = \bbE[WW^T]^{-1}\bbE[WY] = \begin{pmatrix}
    1 & \pi_0& 0 &\cdots & 0\\
    0 & \pi_1& 0 &\cdots & 0\\
    0 & \pi_2& 1 & \cdots & 0\\
    \vdots & \vdots & \vdots & \ddots & \vdots\\
    0 & \pi_k & 0 & 0 & 1
\end{pmatrix}\] which is clearly if and only if invertible when $\pi \neq 0.$ 

($c\iff b$) This is a result from linear algebra. More generally, if $A$ is an invertible matrix, then $B$ is invertible if and only if $A^{-1}B$ is invertible. For the forward direction we see that since $A$ and $B$ are invertible, then $B^{-1}A$ is the inverse of $A^{-1}B.$ For the backward direction, we note that products of invertible matrices are invertible, so then $B = A(A^{-1}B)$ is invertible. 
\end{proof}

With the assumptions above, we derive
\[0 = \bbE[WU] = \bbE[W(Y - X^T\beta)] = \bbE[WY] - \bbE[WX^T]\beta\] and thus 
\begin{align}
    \beta = \bbE[WX^T]^{-1}\bbE[WY]
\end{align}

\begin{thm}
\begin{center}
\fbox{
  \parbox{0.9\textwidth}{
    \textbf{(Estimation with MLR Instrument Variables)} let $Y_{1\times 1}, X_{k +1\times 1 }, U_{k+1\times 1}, Z_{k+1 \times 1}$ be R.V.s with $Y = \beta_0 +  \beta_1X + U$ and $W= (1, Z, W_1, X_2, X_k)$ such that
    \begin{enumerate}
        \item $\bbE[WU] = 0$ (Exogeneity)
        \item $\bbE[WX^T]$ is invertible (Relevance).
        \item $W$ has no perfect co-linearity
    \end{enumerate}
    Let $(Y^1, X^1, Z^1), \dots, (Y^n, X^n, Z^n)\sim (Y, X, Z)$ i.i.d. Then the IV estimator for $\beta$ given by 
\begin{align}
    \hat\beta^{IV} &= \left(\frac{1}{n}\sum W^i (X^i)^T\right)^{-1} \frac{1}{n}\sum W^iY^i
\end{align}
satisfies
    \begin{enumerate}
    \item The sample equivalents of (a):
    \begin{align*}
        0 &= \frac{1}{n}\sum W^i(Y^i - (X^i)^T\hat\beta^{IV}) = \frac{1}{n}\sum W^i \hat U^i
    \end{align*}
        \item (\textbf{Consistent}) If $\bbE[Y^2], \bbE[Z^2], \bbE[X^2] < \infty,$ then $\hat\beta^{IV}$ is consistent.
        \item (\textbf{Asymptotic Distribution}) If $\bbE[X^4], \bbE[Y^4], \bbE[Z^4] < \infty$, then 
        \[\sqrt{n}(\hat \beta^{IV} - \beta) \sim N(0, \Sigma_{IV})\] where 
        \[\Sigma_{IV} = \bbE[WX^T]^{-1}\Var(WU)(\bbE[WX^T]^{-1})^T\]
    \end{enumerate}
    
    }}
\end{center}

\end{thm}
\begin{proof}
    \begin{itemize}
        \item From the equation in (a), it is a simple rearrangement to recuperate (20).
        \item Using the WLLN, 
        \[\frac{1}{n}\sum W^i(Y^i)^T \xrightarrow[\bbP]{}\bbE[WY^T], \qquad \frac{1}{n}\sum W^i Y^i \xrightarrow[\bbP]{}\bbE[WY]\] Using the continuous mapping theorem along with (b), we arrive at the result.
        \item Expanding (20), 
        \begin{align*}
            \hat\beta^{IV} &= \left(\frac{1}{n}\sum W^i (X^i)^T\right)^{-1} \frac{1}{n}\sum W^iY^i\\
            &= \left(\frac{1}{n}\sum W^i (X^i)^T\right)^{-1} \frac{1}{n}\sum W^i((X^i)^T\beta + U^i)
        \end{align*}
        and thus 
        \begin{align}
            \hat\beta^{IV}&= \beta + \left(\frac{1}{n}\sum W^i (X^i)^T\right)^{-1} \frac{1}{n}\sum W^iU^i
        \end{align}
        Rearranging (21),
        \begin{align*}
            \sqrt{n}(\hat\beta^{IV} - \beta) 
            &=  \left(\frac{1}{n}\sum W^i (X^i)^T\right)^{-1} \sqrt{n}\sum W^iU^i\\
            &\xrightarrow[\bbP]{} \bbE[WX^T]^{-1} N(0,\Var(WU))\\
            &= N(0, \bbE[WX^T]^{-1}\Var(WU)(\bbE[WX^T]^{-1})^T)
        \end{align*}
        As usual (since we have that $\Var(WU) = \bbE[WW^T U^2]$), we estimate this variance by 
        \[\hat\Sigma_{IV} = \left(\frac{1}{n}\sum W^i (X^i)^T \right)^{-1}\left(\frac{1}{n}\sum W^i(W^i)^T\hat U^i\right)\left(\left(\frac{1}{n}\sum W^i (X^i)^T\right)^{-1}\right)^T\]
    \end{itemize}
\end{proof}
\begin{rem}
    (\textbf{Testing Relevance}) One can run an OLS on 
    \[X_1 = W^T \pi = \pi_0 + \pi_1 Z +\cdots + \pi_k X_k\] to test $H_0: \pi_1 = 0$ and $H_a: \pi_0 \neq 0.$ Use an $F$ statistics where the rule of thumb is that $F >10$ implies a relevant instrument, while $F\leq 10$ is an weak instrument that can inflate $SE(\hat\beta^{IV})$
\end{rem}

\end{rem}

\hline
\begin{rem}
    \textbf{(2SLS)} Applying Remark 35 to the SLR case, we consider $Y = X^\beta + U,$ where $X$ is endogenous, and $Z$ is an instrument such that 
    \[X = \hat{\text{BLP}}(X \mid Z) + V = \underbrace{\hat \pi_0 + \hat \pi_1Z}_{\hat X} + \hat \varepsilon,  \qquad \text{ (first stage)}\] Because this is an OLS, then $X^*$ is exogenous. Thus, we can run the OLS
    \[Y = \text{BLP}(Y \mid \hat X) + U^*= \hat \beta_0^{2SLS} + \beta_1^{2SLS}\hat X + \hat U \qquad \text{(second stage)}.\] With the assumptions of Theorem 13, estimating $\beta$ gives
\[\hat\beta_1^{SLS} = \frac{\Cov(\hat{X}, Y)}{\Var(\hat X)} \qquad \hat\beta_1^{SLS}= \bar Y - \hat\beta_1^{SLS}\bar {\hat X}\]
    with consistency results: 
    \[\hat\beta_0^{2SLS}\xrightarrow[\bbP]{}\beta_0 \qquad \hat\beta_1^{2SLS}\xrightarrow[\bbP]{}\beta_0\]
    and 
    \[\hat\beta_1^{SLS} = \hat \beta_1^{IV}\]
\end{rem}


\newpage
\subsection{Friday, July 16: Causal Inference}

\begin{rem}
    (\textbf{Rubin Causal Model}) 
    \begin{itemize}
        \item Assignment mechanism is first come first serve,
        \item Define $y_0^i$ to be the \underline{potential} outcome of individual $i$ if it does not receive the treatment.
        \item Define $y_1^i$ as above.
        \item Let 
        \[D^i = \begin{cases}1 \quad i \text{ did received treatment}
        \\
        0 \quad \text{else}
        \end{cases}\]
        \item Define $\tau^i = y_1^i - y_0^i$ to be the potential treatment difference between person $i.$ \begin{center}
\fbox{
  \parbox{0.9\textwidth}{
    \textbf{(Fundamental Problem fo Causal Inference)} $\tau^i$ cannot be observed since we cannot clone people.
    }}
\end{center}

        \item Define the observed outcome to be 
        \[Y^i = y^i_0 + D^i(y^i_1 - y^i_0)\]
        \item Define the \textbf{average treatment effect} is defined by 
    \[ATE= \bbE[y^i_0 - y^i_1]\]
    \item Define the the \textbf{average treatment effect on treated} is 
    \[ATT = \bbE[y^i_1 - y^i_0 \mid D^i = 1]\]
    \item Define the \textbf{average treatment effect on untreated is}
    \[ATU = \bbE[y^i_1 - y^i_0 \mid D^i = 0]\]
    \item Define the 
    naive treatment effect is 
    \[\theta = \bbE[Y^1 \mid D^i = 1 ] - \bbE[Y^i \mid D^i = 0] = \bbE[y^i_1 \mid D_i = 1] - \bbE[y^i_0 \mid D_i = 0]\]
    \end{itemize}
Clearly, we can estimate 
\[\hat\theta = \bar Y_T - \bar Y_C\xrightarrow[\bbP]{}\theta,\] but $\theta \neq $ any AT($\cdot$) above!

\begin{rem}
    (\textbf{Selection Bias an Treatment Effects})
Notice that 
\begin{align*}
    \theta &= \bbE[y^i_1 \mid D_i = 1] - \bbE[y^i_0 \mid D_i = 0]\\
    &= \bbE[y^i_1 \mid D_i = 1] - \bbE[y^i_0 \mid D_i = 1] + \bbE[y^i_0 \mid D_i = 1] - \bbE[y^i_0 \mid D_i = 0]\\
    &= ATT + \underbrace{\bbE[y^i_0 \mid D_i = 1] - \bbE[y^i_0 \mid D_i = 0]}_{SB_0\text{ selection bias in $y_0$}}
\end{align*} Similarly, 
\begin{align}
\boxed{\theta = \begin{cases}
    ATT + SB_0\\
    ATU + SB_1
\end{cases}}    
\end{align}
By (22), we see that if $SB_0, SB_1>0,$ then $\theta >ATU, ATU.$ Also by 22, we see that 
\[ATT + SB_0 = ATU + SB_1\]
\begin{exmp}
\textbf{(The Golden Standard in RCM)}
    In an experiment with randomization such that $y_0^i, y_1^i \perp D^i,$ we see that 
    \[SB_0 = \bbE[y^i_0 \mid D_i = 1] - \bbE[y^i_0 \mid D_i = 0] = \bbE[y_0^i] - \bbE[y_0^i] = 0\] and same for $SB_1.$

    Moreover, 
    \begin{align*}
    ATE &= \bbE[y_1^i - y_0^i]\\
    &= \bbE[\bbE[y_1^i - y_0^i \mid D^i]]\\
    &= p (ATT) + (1-p)(ATU)\\
    &= p(\theta - SB_0)  + (1-p)(\theta - SB_1)\\
    &= \theta
\end{align*}
Thus, in a randomized experiment, 
\boxed{ATT = ATU= ATE = \theta }
\end{exmp} 
\end{rem}
\end{rem}
\end{rem}

\begin{rem}
    (\textbf{Difference in Differences Model})
We consider the model 
\[Y^i_t = \beta_0 + \beta_1 \text{D}^i + \beta_2 \text{Post}_t + \beta_3 (\text{D}^i \times \text{Post}_t) + U^{i}\] where 
\begin{itemize}
    \item $Y_{it}$: Outcome for unit $i$ at time $t$
    \item $\text{D}^i$: Treatment group indicator (1 if treated, 0 otherwise)
    \item $\text{Post}_t$: Post-treatment period indicator
    \item $\beta_3$: DiD estimator (treatment effect)
\end{itemize}
\begin{align*}
\text{ATT} &= E[Y_{1i} - Y_{0i} \mid \text{Treat}_i = 1] \\
\beta_3 &= \underbrace{(\text{Treatment}_{\text{Post}} - \text{Treatment}_{\text{Pre}})}_{\text{Treatment group change}} - \underbrace{(\text{Control}_{\text{Post}} - \text{Control}_{\text{Pre}})}_{\text{Control group change}}
\end{align*}
These two are equal if:
\begin{enumerate}
    \item \textbf{Parallel Trends}: Control group represents counterfactual trend
    \item \textbf{No Anticipation}: Treatment doesn't affect pre-period outcomes
    \item \textbf{SUTVA}: No interference between units
\end{enumerate}
\begin{figure}[H]
    \centering
    \includegraphics[width=0.5\linewidth]{Images/DiD.png}
    \caption{PTA Visualized}
\end{figure}
\end{rem}


\section{Commandments of Econometrics}
\fbox{
  \parbox{0.9\textwidth}{
    \textbf{(First Commandment)} Never assume homoskedasticity, always compute the robust 
    \[SE(\hat\beta)\]
    }}
Reason: almost never is, and hard to see!

\fbox{
  \parbox{0.9\textwidth}{
    \textbf{(Second Commandment)} Never build a model with perfect colinearity in $\textbf{X}.$
    }}
Reason: $\hat\beta$ won't exist!

    \fbox{
  \parbox{0.9\textwidth}{
    \textbf{(Third Commandment)} Never use OLS to estimate supply and demand. Use instrument variables instead. 
    }}
Reason: $P$ is endogenous!
\end{document}