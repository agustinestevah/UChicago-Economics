\documentclass[11pt]{article}


% NOTE: Add in the relevant information to the commands below; or, if you'll be using the same information frequently, add these commands at the top of paolo-pset.tex file. 
\newcommand{\name}{Agustín Esteva \heart}
\newcommand{\email}{aesteva@uchicago.edu}
\newcommand{\classnum}{20210}
\newcommand{\subject}{Econometrics \heart}
\newcommand{\instructors}{Murilo Ramos}
\newcommand{\assignment}{Problem Set 4}
\newcommand{\semester}{Summer 2025}
\newcommand{\duedate}{\today}
\newcommand{\bA}{\mathbf{A}}
\newcommand{\bB}{\mathbf{B}}
\newcommand{\bC}{\mathbf{C}}
\newcommand{\bD}{\mathbf{D}}
\newcommand{\bE}{\mathbf{E}}
\newcommand{\bF}{\mathbf{F}}
\newcommand{\bG}{\mathbf{G}}
\newcommand{\bH}{\mathbf{H}}
\newcommand{\bI}{\mathbf{I}}
\newcommand{\bJ}{\mathbf{J}}
\newcommand{\bK}{\mathbf{K}}
\newcommand{\bL}{\mathbf{L}}
\newcommand{\bM}{\mathbf{M}}
\newcommand{\bN}{\mathbf{N}}
\newcommand{\bO}{\mathbf{O}}
\newcommand{\bP}{\mathbf{P}}
\newcommand{\bQ}{\mathbf{Q}}
\newcommand{\bR}{\mathbf{R}}
\newcommand{\bS}{\mathbf{S}}
\newcommand{\bT}{\mathbf{T}}
\newcommand{\bU}{\mathbf{U}}
\newcommand{\bV}{\mathbf{V}}
\newcommand{\bW}{\mathbf{W}}
\newcommand{\bX}{\mathbf{X}}
\newcommand{\bY}{\mathbf{Y}}
\newcommand{\bZ}{\mathbf{Z}}
\newcommand{\Vol}{\text{Vol}}
\newcommand{\MSE}{\text{MSE}}
\newcommand{\Bias}{\text{Bias}}
\newcommand{\Var}{\text{Var}}
\newcommand{\Cov}{\text{Cov}}
\newcommand{\Corr}{\text{Corr}}



%% blackboard bold math capitals
\newcommand{\bbA}{\mathbb{A}}
\newcommand{\bbB}{\mathbb{B}}
\newcommand{\bbC}{\mathbb{C}}
\newcommand{\bbD}{\mathbb{D}}
\newcommand{\bbE}{\mathbb{E}}
\newcommand{\bbF}{\mathbb{F}}
\newcommand{\bbG}{\mathbb{G}}
\newcommand{\bbH}{\mathbb{H}}
\newcommand{\bbI}{\mathbb{I}}
\newcommand{\bbJ}{\mathbb{J}}
\newcommand{\bbK}{\mathbb{K}}
\newcommand{\bbL}{\mathbb{L}}
\newcommand{\bbM}{\mathbb{M}}
\newcommand{\bbN}{\mathbb{N}}
\newcommand{\bbO}{\mathbb{O}}
\newcommand{\bbP}{\mathbb{P}}
\newcommand{\bbQ}{\mathbb{Q}}
\newcommand{\bbR}{\mathbb{R}}
\newcommand{\bbS}{\mathbb{S}}
\newcommand{\bbT}{\mathbb{T}}
\newcommand{\bbU}{\mathbb{U}}
\newcommand{\bbV}{\mathbb{V}}
\newcommand{\bbW}{\mathbb{W}}
\newcommand{\bbX}{\mathbb{X}}
\newcommand{\bbY}{\mathbb{Y}}
\newcommand{\bbZ}{\mathbb{Z}}

%% script math capitals
\newcommand{\sA}{\mathscr{A}}
\newcommand{\sB}{\mathscr{B}}
\newcommand{\sC}{\mathscr{C}}
\newcommand{\sD}{\mathscr{D}}
\newcommand{\sE}{\mathscr{E}}
\newcommand{\sF}{\mathscr{F}}
\newcommand{\sG}{\mathscr{G}}
\newcommand{\sH}{\mathscr{H}}
\newcommand{\sI}{\mathscr{I}}
\newcommand{\sJ}{\mathscr{J}}
\newcommand{\sK}{\mathscr{K}}
\newcommand{\sL}{\mathscr{L}}
\newcommand{\sM}{\mathscr{M}}
\newcommand{\sN}{\mathscr{N}}
\newcommand{\sO}{\mathscr{O}}
\newcommand{\sP}{\mathscr{P}}
\newcommand{\sQ}{\mathscr{Q}}
\newcommand{\sR}{\mathscr{R}}
\newcommand{\sS}{\mathscr{S}}
\newcommand{\sT}{\mathscr{T}}
\newcommand{\sU}{\mathscr{U}}
\newcommand{\sV}{\mathscr{V}}
\newcommand{\sW}{\mathscr{W}}
\newcommand{\sX}{\mathscr{X}}
\newcommand{\sY}{\mathscr{Y}}
\newcommand{\sZ}{\mathscr{Z}}


\renewcommand{\emptyset}{\O}

\newcommand{\abs}[1]{\lvert #1 \rvert}
\newcommand{\norm}[1]{\lVert #1 \rVert}
\newcommand{\sm}{\setminus}



\newcommand{\sarr}{\rightarrow}
\newcommand{\arr}{\longrightarrow}

% NOTE: Defining collaborators is optional; to not list collaborators, comment out the line below.
%\newcommand{\collaborators}{Alyssa P. Hacker (\texttt{aphacker}), Ben Bitdiddle (\texttt{bitdiddle})}

% Copyright 2021 Paolo Adajar (padajar.com, paoloadajar@mit.edu)
% 
% Permission is hereby granted, free of charge, to any person obtaining a copy of this software and associated documentation files (the "Software"), to deal in the Software without restriction, including without limitation the rights to use, copy, modify, merge, publish, distribute, sublicense, and/or sell copies of the Software, and to permit persons to whom the Software is furnished to do so, subject to the following conditions:
%
% The above copyright notice and this permission notice shall be included in all copies or substantial portions of the Software.
% 
% THE SOFTWARE IS PROVIDED "AS IS", WITHOUT WARRANTY OF ANY KIND, EXPRESS OR IMPLIED, INCLUDING BUT NOT LIMITED TO THE WARRANTIES OF MERCHANTABILITY, FITNESS FOR A PARTICULAR PURPOSE AND NONINFRINGEMENT. IN NO EVENT SHALL THE AUTHORS OR COPYRIGHT HOLDERS BE LIABLE FOR ANY CLAIM, DAMAGES OR OTHER LIABILITY, WHETHER IN AN ACTION OF CONTRACT, TORT OR OTHERWISE, ARISING FROM, OUT OF OR IN CONNECTION WITH THE SOFTWARE OR THE USE OR OTHER DEALINGS IN THE SOFTWARE.

\usepackage{fullpage}
\usepackage{enumitem}
\usepackage{amsfonts, amssymb, amsmath,amsthm}
\usepackage{mathtools}
\usepackage[pdftex, pdfauthor={\name}, pdftitle={\classnum~\assignment}]{hyperref}
\usepackage[dvipsnames]{xcolor}
\usepackage{bbm}
\usepackage{graphicx}
\usepackage{mathrsfs}
\usepackage{pdfpages}
\usepackage{tabularx}
\usepackage{pdflscape}
\usepackage{makecell}
\usepackage{booktabs}
\usepackage{natbib}
\usepackage{caption}
\usepackage{subcaption}
\usepackage{physics}
\usepackage[many]{tcolorbox}
\usepackage{version}
\usepackage{ifthen}
\usepackage{cancel}
\usepackage{listings}
\usepackage{courier}

\usepackage{tikz}
\usepackage{istgame}

\hypersetup{
	colorlinks=true,
	linkcolor=blue,
	filecolor=magenta,
	urlcolor=blue,
}

\setlength{\parindent}{0mm}
\setlength{\parskip}{2mm}

\setlist[enumerate]{label=({\alph*})}
\setlist[enumerate, 2]{label=({\roman*})}

\allowdisplaybreaks[1]

\newcommand{\psetheader}{
	\ifthenelse{\isundefined{\collaborators}}{
		\begin{center}
			{\setlength{\parindent}{0cm} \setlength{\parskip}{0mm}
				
				{\textbf{\classnum~\semester:~\assignment} \hfill \name}
				
				\subject \hfill \href{mailto:\email}{\tt \email}
				
				Instructor(s):~\instructors \hfill Due Date:~\duedate	
				
				\hrulefill}
		\end{center}
	}{
		\begin{center}
			{\setlength{\parindent}{0cm} \setlength{\parskip}{0mm}
				
				{\textbf{\classnum~\semester:~\assignment} \hfill \name\footnote{Collaborator(s): \collaborators}}
				
				\subject \hfill \href{mailto:\email}{\tt \email}
				
				Instructor(s):~\instructors \hfill Due Date:~\duedate	
				
				\hrulefill}
		\end{center}
	}
}

\renewcommand{\thepage}{\classnum~\assignment \hfill \arabic{page}}
\newcommand{\uconv}{\overset{}{\rightrightarrows}}
\makeatletter
\def\points{\@ifnextchar[{\@with}{\@without}}
\def\@with[#1]#2{{\ifthenelse{\equal{#2}{1}}{{[1 point, #1]}}{{[#2 points, #1]}}}}
\def\@without#1{\ifthenelse{\equal{#1}{1}}{{[1 point]}}{{[#1 points]}}}
\makeatother

\newtheoremstyle{theorem-custom}%
{}{}%
{}{}%
{\itshape}{.}%
{ }%
{\thmname{#1}\thmnumber{ #2}\thmnote{ (#3)}}

\theoremstyle{theorem-custom}

\newtheorem{theorem}{Theorem}
\newtheorem{lemma}[theorem]{Lemma}
\newtheorem{example}[theorem]{Example}

\newenvironment{problem}[1]{\color{black} #1}{}

\newenvironment{solution}{%
	\leavevmode\begin{tcolorbox}[breakable, colback=green!5!white,colframe=green!75!black, enhanced jigsaw] \proof[\scshape Solution:] \setlength{\parskip}{2mm}%
	}{\renewcommand{\qedsymbol}{$\heartsuit$} \endproof \end{tcolorbox}}

\newenvironment{reflection}{\begin{tcolorbox}[breakable, colback=black!8!white,colframe=black!60!white, enhanced jigsaw, parbox = false]\textsc{Reflections:}}{\end{tcolorbox}}

\newcommand{\qedh}{\renewcommand{\qedsymbol}{$\blacksquare$}\qedhere}

\definecolor{mygreen}{rgb}{0,0.6,0}
\definecolor{mygray}{rgb}{0.5,0.5,0.5}
\definecolor{mymauve}{rgb}{0.58,0,0.82}

% from https://github.com/satejsoman/stata-lstlisting
% language definition
\lstdefinelanguage{Stata}{
	% System commands
	morekeywords=[1]{regress, reg, summarize, sum, display, di, generate, gen, bysort, use, import, delimited, predict, quietly, probit, margins, test},
	% Reserved words
	morekeywords=[2]{aggregate, array, boolean, break, byte, case, catch, class, colvector, complex, const, continue, default, delegate, delete, do, double, else, eltypedef, end, enum, explicit, export, external, float, for, friend, function, global, goto, if, inline, int, local, long, mata, matrix, namespace, new, numeric, NULL, operator, orgtypedef, pointer, polymorphic, pragma, private, protected, public, quad, real, return, rowvector, scalar, short, signed, static, strL, string, struct, super, switch, template, this, throw, transmorphic, try, typedef, typename, union, unsigned, using, vector, version, virtual, void, volatile, while,},
	% Keywords
	morekeywords=[3]{forvalues, foreach, set},
	% Date and time functions
	morekeywords=[4]{bofd, Cdhms, Chms, Clock, clock, Cmdyhms, Cofc, cofC, Cofd, cofd, daily, date, day, dhms, dofb, dofC, dofc, dofh, dofm, dofq, dofw, dofy, dow, doy, halfyear, halfyearly, hh, hhC, hms, hofd, hours, mdy, mdyhms, minutes, mm, mmC, mofd, month, monthly, msofhours, msofminutes, msofseconds, qofd, quarter, quarterly, seconds, ss, ssC, tC, tc, td, th, tm, tq, tw, week, weekly, wofd, year, yearly, yh, ym, yofd, yq, yw,},
	% Mathematical functions
	morekeywords=[5]{abs, ceil, cloglog, comb, digamma, exp, expm1, floor, int, invcloglog, invlogit, ln, ln1m, ln, ln1p, ln, lnfactorial, lngamma, log, log10, log1m, log1p, logit, max, min, mod, reldif, round, sign, sqrt, sum, trigamma, trunc,},
	% Matrix functions
	morekeywords=[6]{cholesky, coleqnumb, colnfreeparms, colnumb, colsof, corr, det, diag, diag0cnt, el, get, hadamard, I, inv, invsym, issymmetric, J, matmissing, matuniform, mreldif, nullmat, roweqnumb, rownfreeparms, rownumb, rowsof, sweep, trace, vec, vecdiag, },
	% Programming functions
	morekeywords=[7]{autocode, byteorder, c, _caller, chop, abs, clip, cond, e, fileexists, fileread, filereaderror, filewrite, float, fmtwidth, has_eprop, inlist, inrange, irecode, matrix, maxbyte, maxdouble, maxfloat, maxint, maxlong, mi, minbyte, mindouble, minfloat, minint, minlong, missing, r, recode, replay, return, s, scalar, smallestdouble,},
	% Random-number functions
	morekeywords=[8]{rbeta, rbinomial, rcauchy, rchi2, rexponential, rgamma, rhypergeometric, rigaussian, rlaplace, rlogistic, rnbinomial, rnormal, rpoisson, rt, runiform, runiformint, rweibull, rweibullph,},
	% Selecting time-span functions
	morekeywords=[9]{tin, twithin,},
	% Statistical functions
	morekeywords=[10]{betaden, binomial, binomialp, binomialtail, binormal, cauchy, cauchyden, cauchytail, chi2, chi2den, chi2tail, dgammapda, dgammapdada, dgammapdadx, dgammapdx, dgammapdxdx, dunnettprob, exponential, exponentialden, exponentialtail, F, Fden, Ftail, gammaden, gammap, gammaptail, hypergeometric, hypergeometricp, ibeta, ibetatail, igaussian, igaussianden, igaussiantail, invbinomial, invbinomialtail, invcauchy, invcauchytail, invchi2, invchi2tail, invdunnettprob, invexponential, invexponentialtail, invF, invFtail, invgammap, invgammaptail, invibeta, invibetatail, invigaussian, invigaussiantail, invlaplace, invlaplacetail, invlogistic, invlogistictail, invnbinomial, invnbinomialtail, invnchi2, invnF, invnFtail, invnibeta, invnormal, invnt, invnttail, invpoisson, invpoissontail, invt, invttail, invtukeyprob, invweibull, invweibullph, invweibullphtail, invweibulltail, laplace, laplaceden, laplacetail, lncauchyden, lnigammaden, lnigaussianden, lniwishartden, lnlaplaceden, lnmvnormalden, lnnormal, lnnormalden, lnwishartden, logistic, logisticden, logistictail, nbetaden, nbinomial, nbinomialp, nbinomialtail, nchi2, nchi2den, nchi2tail, nF, nFden, nFtail, nibeta, normal, normalden, npnchi2, npnF, npnt, nt, ntden, nttail, poisson, poissonp, poissontail, t, tden, ttail, tukeyprob, weibull, weibullden, weibullph, weibullphden, weibullphtail, weibulltail,},
	% String functions 
	morekeywords=[11]{abbrev, char, collatorlocale, collatorversion, indexnot, plural, plural, real, regexm, regexr, regexs, soundex, soundex_nara, strcat, strdup, string, strofreal, string, strofreal, stritrim, strlen, strlower, strltrim, strmatch, strofreal, strofreal, strpos, strproper, strreverse, strrpos, strrtrim, strtoname, strtrim, strupper, subinstr, subinword, substr, tobytes, uchar, udstrlen, udsubstr, uisdigit, uisletter, ustrcompare, ustrcompareex, ustrfix, ustrfrom, ustrinvalidcnt, ustrleft, ustrlen, ustrlower, ustrltrim, ustrnormalize, ustrpos, ustrregexm, ustrregexra, ustrregexrf, ustrregexs, ustrreverse, ustrright, ustrrpos, ustrrtrim, ustrsortkey, ustrsortkeyex, ustrtitle, ustrto, ustrtohex, ustrtoname, ustrtrim, ustrunescape, ustrupper, ustrword, ustrwordcount, usubinstr, usubstr, word, wordbreaklocale, worcount,},
	% Trig functions
	morekeywords=[12]{acos, acosh, asin, asinh, atan, atanh, cos, cosh, sin, sinh, tan, tanh,},
	morecomment=[l]{//},
	% morecomment=[l]{*},  // `*` maybe used as multiply operator. So use `//` as line comment.
	morecomment=[s]{/*}{*/},
	% The following is used by macros, like `lags'.
	morestring=[b]{`}{'},
	% morestring=[d]{'},
	morestring=[b]",
	morestring=[d]",
	% morestring=[d]{\\`},
	% morestring=[b]{'},
	sensitive=true,
}

\lstset{ 
	backgroundcolor=\color{white},   % choose the background color; you must add \usepackage{color} or \usepackage{xcolor}; should come as last argument
	basicstyle=\footnotesize\ttfamily,        % the size of the fonts that are used for the code
	breakatwhitespace=false,         % sets if automatic breaks should only happen at whitespace
	breaklines=true,                 % sets automatic line breaking
	captionpos=b,                    % sets the caption-position to bottom
	commentstyle=\color{mygreen},    % comment style
	deletekeywords={...},            % if you want to delete keywords from the given language
	escapeinside={\%*}{*)},          % if you want to add LaTeX within your code
	extendedchars=true,              % lets you use non-ASCII characters; for 8-bits encodings only, does not work with UTF-8
	firstnumber=0,                % start line enumeration with line 1000
	frame=single,	                   % adds a frame around the code
	keepspaces=true,                 % keeps spaces in text, useful for keeping indentation of code (possibly needs columns=flexible)
	keywordstyle=\color{blue},       % keyword style
	language=Octave,                 % the language of the code
	morekeywords={*,...},            % if you want to add more keywords to the set
	numbers=left,                    % where to put the line-numbers; possible values are (none, left, right)
	numbersep=5pt,                   % how far the line-numbers are from the code
	numberstyle=\tiny\color{mygray}, % the style that is used for the line-numbers
	rulecolor=\color{black},         % if not set, the frame-color may be changed on line-breaks within not-black text (e.g. comments (green here))
	showspaces=false,                % show spaces everywhere adding particular underscores; it overrides 'showstringspaces'
	showstringspaces=false,          % underline spaces within strings only
	showtabs=false,                  % show tabs within strings adding particular underscores
	stepnumber=2,                    % the step between two line-numbers. If it's 1, each line will be numbered
	stringstyle=\color{mymauve},     % string literal style
	tabsize=2,	                   % sets default tabsize to 2 spaces
%	title=\lstname,                   % show the filename of files included with \lstinputlisting; also try caption instead of title
	xleftmargin=0.25cm
}

% NOTE: To compile a version of this pset without problems, solutions, or reflections, uncomment the relevant line below.

%\excludeversion{problem}
%\excludeversion{solution}
%\excludeversion{reflection}

\begin{document}	

	% Use the \psetheader command at the beginning of a pset. 
	\psetheader
\section*{Problem 1 \heart}


Stock and Watson, Exercises 6.6, 6.9\\

\begin{enumerate}
    \item (\textbf{6.6})
A researcher plans to study the causal effect of police on crime, using data from a random sample of U.S. counties. He plans to regress the county’s crime rate on the (per capita) size of the county’s police force.

\begin{enumerate}
    \item[(a)] Explain why this regression is likely to suffer from omitted variable bias. Which variables would you add to the regression to control for important omitted variables?
\begin{solution}
    Because his error term is full of life! Imagine how many variables go into crime other than policing force. Consider the income level of the counties or the race makeup of the county or unemployment levels. These should all be added to control. 
\end{solution}
    \item[(b)] Use your answer to (a) and the expression for omitted variable bias given in Equation (6.1) to determine whether the regression will likely over- or underestimate the effect of police on the crime rate. That is, do you think that 
    \[
    \hat{\beta}_1 > \beta_1 \quad \text{or} \quad \hat{\beta}_1 < \beta_1?
    \]
    \begin{solution}
        I think investigations will show the higher the poverty level, the higher the crime will be. Thus, $\Cov(U_i, X_i)>0,$ and thus the regression will overestimate the effect of police on crime. Hence, we should find more effective methods than policing to handle crime. 
    \end{solution}
\end{enumerate}

\item (\textbf{6.9}) Suppose $(Y_i, X_{1i}, X_{2i})$ satisfy the assumptions in Key Concept 6.4. You are interested in $\beta_1$, the causal effect of $X_1$ on $Y$. Suppose $X_1$ and $X_2$ are uncorrelated. You estimate $\beta_1$ by regressing $Y$ onto $X_1$ (so that $X_2$ is not included in the regression). Does this estimator suffer from omitted variable bias? Explain.
\begin{solution}
    No, this is answered by Key Concept 6.1 in the textbook! Also, from the slides, 
    \[\Bias = \frac{\beta_2 \Cov(X_1, X_2)}{\Var(X_1)} = 0.\]
\end{solution}
\end{enumerate}


\newpage
\section*{Problem 2 \heart}

Suppose
\[
Y = \beta_0 + \beta_1 X + U,
\]
where \( Y \) is a binary random variable. Suppose further that \( \mathbb{E}[U \mid X] = 0 \) and \( 0 < \operatorname{Var}[X] < \infty \).

\begin{enumerate}[label=(\alph*)]
    \item What is \( \mathbb{E}[Y \mid X] \)? What is \( \mathbb{P}(Y = 1 \mid X) \)?
    \begin{solution}
        Using LIE, 
        \begin{align*}
            \bbE[Y \mid X] &= \bbE[\beta_0 + \beta_1 X + U \mid X]\\
            &= \beta_0 + \beta_1X + \bbE[U \mid X]\\
            &= \beta_0 + \beta_1 X
        \end{align*}
        We have that since $Y$ is binary, then
        \begin{align*}
            \bbP(Y = 1 \mid X) &= \bbE[Y \mid X]
        \end{align*}
    \end{solution}
    \item What is \( \operatorname{Var}[Y \mid X] \)?
    \begin{solution}
    Since $Y$ is binary then $Y^2 = Y.$ 
    \begin{align*}
        \Var(Y \mid X) &= \bbE[Y^2 \mid X] - \bbE[Y \mid X]^2\\
        &=   \bbE[Y \mid X](1 - \bbE[Y \mid X])
    \end{align*}
    where we have seen what $\bbE[Y \mid X]$ is above.

    \end{solution}
    \item What is \( \operatorname{Var}[U \mid X] \)? Is the model homoskedastic or heteroskedastic?
    \begin{solution}
        Computing, 
        \begin{align*}
            \Var(U \mid X) &= \Var(Y - \beta_0 - \beta_1X \mid X)\\
            &=\Var(Y \mid X)
        \end{align*}
        and thus $U$ is heterodestatic because $\Var(Y \mid X)$ depends on $X$ from parts (a), (b)
    \end{solution}
    \item Let \( (Y_1, X_1), \dots, (Y_n, X_n) \) be an i.i.d. sample from \( (Y, X) \). In addition to the assumptions above, suppose that \( \mathbb{E}[X^4] < \infty \). Assume that the sample size \( n \) is large.
    \begin{enumerate}[label=(\roman*)]
        \item How would you test the null hypothesis that \( \beta_1 = 0 \) versus the alternative that \( \beta_1 \neq 0 \) at the 5\% significance level?
        \begin{solution}
            Since $U$ is hetero, 
            then we consider 
            \[T = \frac{\hat\beta_1}{\text{SE}(\beta_1)} \sim N(0,1),\] and reject if $T > 1.96$
        \end{solution}
        \item How would you compute the p-value for the test in part (i)?
        \begin{solution}
            From the previous problem, it is clear that $p = 2(1-\Phi(T))$
        \end{solution}
        \item How would you construct a (two-sided) confidence interval for \( \beta_1 \) at the 5\% significance level?
        \begin{solution}
            Light work, 
            \[\beta \in [\hat{\beta}_1 \pm 1.96\text{SE}(\hat\beta_1)]\]
        \end{solution}
    \end{enumerate}
\end{enumerate}

\newpage
\section*{Problem 3 \heart}

Consider the following regression model in the population:
\[
Y_i = X_i' \beta + U_i,
\]
where \( Y_i \) is a scalar (1x1 vector) that represents the outcome for observation \( i \), \( X_i = (1, X_{1i}, X_{2i}, \ldots, X_{ki})' \) is a \( (k+1) \times 1 \) vector that contains the \( k \) regressors plus one intercept (represented by the value 1), and \( U_i \) is a scalar that represents the error term. Moreover, \( \beta = (\beta_0, \beta_1, \ldots, \beta_k)' \) is a \( (k+1) \times 1 \) vector that contains all the population parameters in the BLP \( \mathbb{E}[Y \mid X] \).

\begin{enumerate}[label=(\alph*)]
    \item Show that the equation above is equivalent to
    \[
    Y_i = \beta_0 + \beta_1 X_{1i} + \cdots + \beta_k X_{ki} + U_i.
    \]
    \begin{solution}
        This is just matrix multiplication, 
        \[Y_i = X^T\beta + U_i  = \begin{pmatrix}
    1 & X_1 & X_2 & \cdots &X_k
        \end{pmatrix}\begin{pmatrix}
            \beta_0 \\ \beta_1\\ \vdots \\\beta_k
        \end{pmatrix} + U_i = \beta_0 + \beta_1 X_1 + \cdots + \beta_kX_k + U_i\]
    \end{solution}
    
    \item Multiply the original model by the vector \( X_i \), such that
    \[
    X_i Y_i = X_i X_i' \beta + X_i U_i.
    \]
    Perform the multiplication of the matrices to show each element that is inside the vectors: \( X_i Y_i \), \( X_i X_i' \), and \( X_i U_i \).
    \begin{solution}
        \begin{align*}
            X_i Y_i &=\begin{pmatrix}
                1 \\ X_1 \\\vdots \\ X_k
            \end{pmatrix} Y_i = \begin{pmatrix}
                Y_i \\ Y_iX_1 \\\vdots \\ Y_iX_k
            \end{pmatrix} \\
            X_iX_i^T &= \begin{pmatrix}
                1 \\ X_1 \\\vdots \\ X_k
            \end{pmatrix}\begin{pmatrix}
    1 & X_1 & X_2 & \cdots &X_k
        \end{pmatrix} = \begin{pmatrix}
1 \\
X_{1i} \\
\vdots \\
X_{ki}
\end{pmatrix}
\begin{pmatrix}
1 & X_{1i} & \cdots & X_{ki}
\end{pmatrix}
=
\begin{pmatrix}
1 & X_{1i} & \cdots & X_{ki} \\
X_{1i} & X_{1i}^2 & \cdots & X_{1i} X_{ki} \\
\vdots & \vdots & \ddots & \vdots \\
X_{ki} & X_{1i} X_{ki} & \cdots & X_{ki}^2
\end{pmatrix}
        \end{align*}
        For the last it's literally the same as the first just replace $Y_i$ with $U_i$
    \end{solution}
    \item Now assume \( \mathbb{E}[X_i U_i] = 0 \). Notice that this is a set of \( k+1 \) equations. Show that it implies that \( \mathbb{E}[U_i] = 0 \) and \( \operatorname{Cov}(X_{ji}, U_i) = 0 \), when \( j = 1, \ldots, k \).
    \begin{solution}
        We are assuming 
        \begin{align*}
            \begin{pmatrix}
                \bbE[U_i]\\
                \bbE[X_1 U_i]\\
                \bbE[X_2 U_i]\\
                \vdots\\
                \bbE[X_k U_i]
            \end{pmatrix} = \begin{pmatrix}
                0\\0\\0\\\vdots\\0
            \end{pmatrix}
        \end{align*}
        and we immediately see that $\bbE[U_i] = 0$ and $\bbE[X_{j}, U_i] = 0$ for all $j\geq 0$  We also know that 
        \[\Cov(X_{ji}, U_i) = \bbE[X_{ji}, U_i] - \bbE[X_{ji}]\bbE[U_i] = 0 - 0 = 0.\]
    \end{solution}
    
    \item Show that if \( \mathbb{E}[X_i U_i] = 0 \), then this system of equations has the solution
    \[
    \beta = \left( \mathbb{E}[X_i X_i'] \right)^{-1} \mathbb{E}[X_i Y_i].
    \]
    Which additional assumption have you used in this question?
    \begin{solution}
        Taking expectations, we see that 
        \[\bbE[X_i Y_i] = \bbE[X_i X^T]\beta + \bbE[X_i U_i] = \bbE[X_i X_i^T]\beta.\] Assuming that $\bbE[X_iX^T]$ is invertible, we multiply by the inverse on both sides and conclude.
    \end{solution}
\end{enumerate}





    \end{document}